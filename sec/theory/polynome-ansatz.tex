\subsection{Polynome Ansatz}

Briefly, we are going to take a closer look at the class of \emph{Finite
Elements} approach where the \emph{Finite Volume} and the \emph{Discontinuous
Galerkin} are concrete implementations are. At first the very physical domain
$\Omega$ is divided into a \emph{mesh} of adjunct self-contained sub-domains
$\Omega_l$ ($l \in \mathbb{N})$ with concisely defined boundaries. For the rest
of this text we omit the element index $l$. So $\Omega$ is now the domain within
an element. The unknown solution $\vec{U}$ is replaced by polynomes of order $N_p$
constructed from linear combinations of orthogonal basis functions
$\vec{\Psi}^j$.

TODO: explain multi-index

\begin{equation}
    U_{\vec{I}}(t,x,y,z) \approx p_{\vec{I}}(t,x,y,z) = \sum^{N_p}_{\vec{J}=0} U^{\vec{J}}_{\vec{I}}(t) \, \Psi^{\vec{I}}(x,y,z)
\end{equation}
\remark Usally, the polynomes of every element are transformed to a reference
space $\hat{\Omega} = [-1,1]^3$ where the actual interpolation takes place. This
meassure massively increases efficiency since the basis functions are equal
among all elements. For the sake of simplicity this step is ommitted here.

Remembering the general weak formulation of the solution integral, derived in ...
The treatment is the same for all five conservative variables of the Euler
equation hence we ignore the index $i$.
\begin{align}
    &\int_\Omega \left(\sum^{N_p}_{j=0} (\partial_t U^j(t)) \, \Psi^j(\vec{x})\right)\,\phi(\vec{x}) d^3x 
        + \int_\Omega S(t) \, \phi(\vec{x}) d^3x = \nonumber \br
         &\int_{\partial\Omega} \left[F(t)\,\phi(\vec{x})\,n_x + G(t)\,\phi(\vec{x})\,n_z + H(t)\,\phi(\vec{x})\,n_z \right] d^2x, \br
        -&\int_\Omega \left[
              \left(\sum^{N_p}_{j=0} F^j(t)\,\Psi^j(\vec{x})\right)\,\partial_x \phi(\vec{x}) 
            + \left(\sum^{N_p}_{j=0} G^j(t)\,\Psi^j(\vec{x})\right)\,\partial_y \phi(\vec{x})
            + \left(\sum^{N_p}_{j=0} H^j(t)\,\Psi^j(\vec{x})\right)\,\partial_z \phi(\vec{x}) \right] d^3x \nonumber
\end{align}

\paragraph{Lagrange Polynome}

If we associate the basis functions $\Psi^{\vec{J}} := L^{\vec{J}}$ and the test functions $\phi := L^{\vec{I}}$ with
\vip{Lagrange} polyonmes of equal order $N_p$, we can formulate an interpolation and
integration scheme (\emph{collocation}) over the domain $\Omega$. The polynome in one dimension reads as follows
\begin{equation}
l_j(x) = \prod^p_{k=0,k\neq j} \frac{x-x_k}{x_j-x_k}, \ j = 0,\dots,p,
\end{equation}
whith the \vip{Kronecker} property $l_j(x_i) = \delta_{ij}$.

For illustrative purposes we begin with the simplest case: one-dimensional
\textsc{Lagrange} interpolation.
\image{0.8}{lagrange-interpolation-1d.png}{One-dimensinal Lagrange
interpolation with four sample nodes.}
The Lagrange polynome of third order needs four anchor nodes with their
associated values in order to interpolate any other point in between. This
works very well and does not cause any headaches. Unfortunately, considering
previous section, extrapolation is necessary, too. Lagrange polynomes tend to
explode going further away from the outer anchor nodes. In pathological
interpolation cases this effect yields erroneous results. The end of the
document shows specific examples.

One gets to three-dimensional formulation via the \emph{Tensor Product Ansatz}.
\newcommand{\LI}{L_{\vec{i}}(\vec{x})}
\begin{equation}
\label{eq-tensor-ansatz}
    \LI = L_{ijk}(x,y,z) = l_i(x) \cdot l_j(y) \cdot l_k(z)
\end{equation}

We write the thee-dimensional polynome as:
\newcommand{\PI}{P^{\vec{i}}(t,\vec{x})}
\newcommand{\FI}{F_{\vec{i}}(t)}
\newcommand{\sumI}{\sum^{N_p}_{\vec{i}=0}}
\begin{equation}
\label{eq-tensor-ansatz}
    \PI = \sumI \FI \LI = \sum^{N_p}_{i,j,k = 0} f_{ijk}(t) \cdot l_i(x) \cdot l_j(y) \cdot l_k(z)
\end{equation}
Note, that the time varying part of the polynome lives exclusively in the coefficients.

\paragraph{Galerkin Method}

Replacing $\Psi^{\vec{J}}$ and $\phi$ appropiately, we get an explicit equation

\newcommand{\FJ}{F_{\vec{j}}(t)}
\newcommand{\GJ}{G_{\vec{j}}(t)}
\newcommand{\HJ}{H_{\vec{j}}(t)}

\newcommand{\STX}{S(t,\vec{x})}

\newcommand{\UJ}{ \dot{U}_{\vec{j}}(t)}

\newcommand{\LJ}{L_{\vec{j}}(\vec{x})}
\newcommand{\sumJ}{\sum^{N_p}_{\vec{j}=0}}
\newcommand{\nx}{n_x(\vec{x})}
\newcommand{\ny}{n_y(\vec{x})}
\newcommand{\nz}{n_z(\vec{x})}

\begin{align}
     &\int_\Omega \left(\sumJ \UJ \, \LJ \right) \, \LI d^3x + \int_\Omega \STX \, \LI d^3x = \nonumber \br
     &\int_{\partial\Omega} \bigg[ F(t) \, \nx + G(t) \, \ny + H(t)\, \nz \bigg]\,\LI\, d^2x \br
    -&\int_\Omega \left[
          \left(\sumJ \FJ \, \LJ \right) \, \partial_x \LI 
        + \left(\sumJ \HJ \, \LJ \right) \, \partial_y \LI
        + \left(\sumJ \GJ \, \LJ \right) \, \partial_z \LI \right] d^3x \nonumber
\end{align}

\newcommand{\sumK}{\sum^{N_p}_{\vec{k}=0}}
\newcommand{\wK}{\omega_{\vec{k}}}

\newcommand{\STXK}{S(t,\vec{x}_{\vec{k}})}
\newcommand{\LIK}{L_{\vec{i}}(\vec{x}_{\vec{k}})}
\newcommand{\LJK}{L_{\vec{j}}(\vec{x}_{\vec{k}})}

\newcommand{\LXIK}{L^{(x)}_{\vec{i}}(\vec{x}_{\vec{k}})}
\newcommand{\LYIK}{L^{(y)}_{\vec{i}}(\vec{x}_{\vec{k}})}
\newcommand{\LZIK}{L^{(z)}_{\vec{i}}(\vec{x}_{\vec{k}})}

Evaluating above formular at discrete nodes $\vec{x}_{\vec{i}}$ and introducing
\emph{integration weigths}
\begin{equation}
    \omega_{\vec{i}} = \int_{\Omega} \LI d^3 x
\end{equation}
we can discretize the continuous integrals.
\begin{align}
     &\sumK \left(\sumJ \UJ \, \LJK \right) \, \LIK \,\wK + \sumK \STXK \, \LIK \wK = \nonumber \br
     &\bigg[ F^*(t) + G^*(t) + H^*(t)\bigg]\,\LI \br
    -&\sumK \left[
          \left(\sumJ \FJ \, \LJK \right) \, \LXIK 
        + \left(\sumJ \HJ \, \LJK \right) \, \LYIK
        + \left(\sumJ \GJ \, \LJK \right) \, \LZIK \right] \wK \nonumber
\end{align}
The surface term got replaced by flux functions who exchange mass, momentum and
energy between element boundaries. The partial differential
$\partial_x$ got transformed into a discrete linear operator.
\newcommand{\DXKI}{D^{(x)}_{\vec{k}\vec{i}}}
\newcommand{\DYKI}{D^{(y)}_{\vec{k}\vec{i}}}
\newcommand{\DZKI}{D^{(z)}_{\vec{k}\vec{i}}}
\begin{equation}
    L^{(x)}_{\vec{j}}(\vec{x}) = \partial_x \LJ = (\partial_x l_{j_1}(x)) \cdot l_{j_2}(y) \cdot l_{j_3}(z)
\end{equation}
\begin{equation}
    D^{(x)}_{\vec{i}\vec{j}} = D^{(x)}_{i_1i_2i_3j_1j_2j_3} = l^{(x)}_{j_1}(x_{i_1})\cdot l_{j_2}(y_{i_2}) \cdot l_{j_3}(z_{i_3})
\end{equation}
The differential operator for the y- and z-dimension are constructed in analog
manner.  
\begin{align}
     &\sumK \left(\sumJ \UJ \, \LJK \right) \, \LIK \,\wK + \sumK \STXK \, \LIK \wK = \nonumber \br
     &\bigg[ F^*(t) + G^*(t) + H^*(t)\bigg]\,\LI \br
    -&\sumK \left[
          \left(\sumJ \FJ \, \LJK \right) \, \DXKI 
        + \left(\sumJ \HJ \, \LJK \right) \, \DYKI
        + \left(\sumJ \GJ \, \LJK \right) \, \DZKI \right] \wK \nonumber
\end{align}

If the polynomial order is set to one $N_p = 1$ the formulation reduces to the
first order FV method.
