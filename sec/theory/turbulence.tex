\subsection{Turbulences}

Turbulences are very common phenomena in nature. They can be desired as well as
unsoliceted. In astrophysics turbulence are suspected to play a major role in
star formation in interstellar clouds. Hence, a good understanding of the
underlying mechanics is crucial in order to model them correctly in numerical
simulations. While turbulences in incompressible media has been thoroughly
studied in the past, there is still an on-going debate about what additional
dynamics compressibility brings especially in supersonic setups where shocks
emerge.

Say something about fluctuation property around mean
see Hydrodynamic and MHD Turbulent Flows Chapter 1

\paragraph{Energy Cascade}

see Kolmogorov-Burgers Model for Star-forming Turbulence
    - inertial range -> Kolmogorov scaling since large scales
    - dissipative range -> Burgers scale
    - theoretical ground

see SCALING RELATIONS OF SUPERSONIC TURBULENCE IN STAR-FORMING MOLECULAR CLOUDS
    - numerical validation of theory in above paper

\paragraph{Density and Velocity Distribution}
    - skewness, log, log-log scale ...


