\subsection{Turbulences}

Turbulences are very common phenomena in nature. They can be desired as well as
unsoliceted. In astrophysics turbulence are suspected to play a major role in
star formation in interstellar clouds. Hence, a good understanding of the
underlying mechanics is crucial in order to model them correctly in numerical
simulations. While turbulences in incompressible media has been thoroughly
studied in the past, there is still an on-going debate about what additional
dynamics compressibility brings especially in supersonic setups where shocks
emerge.

Grid point requirement
\begin{equation}
    N \varpropto Re^{9/4}
\end{equation}
can be relaxed since most dissipation takes places 5 to 15 times the
Kolmogorov length scale $\eta$. See Moin and Mahesh, 1998

There is a wide range of time scales in a turbulent flow, so the system of
equations is stiff. Implicit time advancement and large time steps are
routinely used for stiff systems in general-purpose CFD, but these are
unsuitable in DNS because complete time resolution is needed to describe
the energy dissipation process accurately. Specially designed implicit and
explicit methods have been developed to ensure time accuracy and stability
(see e.g. Verstappen and Veldman, 1997).

Reynolds (in Lumley, 1989) noted that it is essential to have accurate time
resolution of all the scales of turbulent motion. The time steps must be3.10 SUMMARY
113
adjusted so that fluid particles do not move more than one mesh spacing.
Moin and Mahesh (1998) demonstrated the strong influence of time step size
on small-scale amplitude and phase error.


Laminar flow becomes unstable above a certain Reynolds number $R = u L/\nu$

Say something about fluctuation property around mean
see Hydrodynamic and MHD Turbulent Flows Chapter 1

Three solving strategies: direct numerical simulation
large eddy simulation and RANS (Reynolds-averaged Navier-Stokes) simulation.

Turbulences is characterized by a random fluctuation of flow variable in time
meassured at a fixed point in space. Hence, the time evolution can be described
via the \emph{Helmholtz decomposition}.
\begin{equation}
    q(t) = \avg{q}_t + \tilde{q}(t), \ \ \avg{\tilde{q}} \equiv 0
\end{equation}
where 
\begin{equation}
    \avg{q}_t = \frac{\int_{t_0}^{t1} q(t) dt}{t_1 - t_0}
\end{equation}
is the time-average of the flow property $q(t)$. A turbulent flow can be
globally described in terms of the mean values of the flow properties like
density, velocity, pressure, etc.

Turbulent flows always show a three-dimensional character and they lead to
rotational flow structures, called turbulent eddies, with a wide range of
length and energy scales.

Fluid particles which were initially separated by a large distance can be
brought close together by eddying motions. Consequently, mass, heat and momentum
are very effectively exchanged. It is an established fact that this
property has a profound influence in birth of stars, solar systems and cosmic
structures.

The largest turbulent eddies interact and transfer energy from the mean flow.
Since large eddies are of the same order as the characteristic length and
velocity scale the flow is inviscid and their dynamics are dominated by
inertial effects. They transfer kinetic energy down to smaller eddies via
\emph{vortex stretching}. This way an energy cascade emerges from largest to
smallest scales.

The spatial wavenumber $k$ of an eddy of wavelength $\lambda$ is defined as
\begin{equation}
    k = \frac{2\pi}{\lambda}
\end{equation}
and the \emph{spectral energy} $E(k)$ is a function of $k$.
\paragraph{Energy Cascade}

see Kolmogorov-Burgers Model for Star-forming Turbulence
    - inertial range -> Kolmogorov scaling since large scales
    - dissipative range -> Burgers scale
    - theoretical ground

see SCALING RELATIONS OF SUPERSONIC TURBULENCE IN STAR-FORMING MOLECULAR CLOUDS
    - numerical validation of theory in above paper


\image{0.8}{energy-cascade-schematic.png}{
Schematic representation of the K41 picture of turbulence showing the spatial
energy spectrum as an example. Source: \cite{MueBiskl2003}, p6}


The smallest length scales $\eta$ of motion are dominated by viscous effects
where their Reynolds number is equal to 1. $Re_{\eta} = \vels \eta / \shearvisc = 1$,
so inertia and viscous effects are of equal strength. The kinetic energy
gets dissipated and converted into thermal energy. These so called
\vip{Kolmogorov} \emph{microscales} are therefore a meassure for the spatial
resolution characteristics and the diffusive order of numerical schemes of equal
h-refinement.

A detailed discussion of turbulences can be found in:
H K Versteeg and W Malalasekera: An Introduction to Computational Fluid Dynamics


AKlRA YOSHIZAWA: HydrodynaIllic and MagnetohydrodynaIllic Turbulent Flows Modelling and
Statistical Theory


\paragraph{Density and Velocity Distribution}
    - skewness, log, log-log scale ...




