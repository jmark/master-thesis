\subsection{Shocks}

Shocks, or in more technical terms singular compression waves, are escalating
highly localized spikes in density and pressure due to nonlinear dynamics
encoded in the Euler equation. When a shock wave is emerging the velocity
behind the wave front is higher than in front of it. The media gets highly
compressed untill an unphysical state is reached. The velocity characteristica
begin to cross. Nature solves this dilemma by introducing additional physics
like extreme heat radiation, explosions, bangs or detachement of media in case
of surface waves. Either way, it involves an entropy increase in the system.
A numerical solver has to capture this kind of physics in order to prevent
unphysical solutions.

Solving Riemann problem.

Due to the nonlinearity of the Euler equations 

\paragraph{Method of Characteristics}

The prototype for all ODEs of second order is the \vip{Burger}'s equation.

\begin{equation}
    U_t + U \cdot U_x = 0
\end{equation}

The commonly used approach to analyze shock waves due to non-linear equations
is by \emph{method of characteristics}
\begin{equation}
    x = x_0 + U(t,x_0) \cdot t
\end{equation}

When we set following initial condition:
\begin{equation}
    U(0,x) = 
    \begin{cases}
        1 \ \ \text{if} \ \ x < 0 \\
        0 \ \ \text{if} \ \ 0 \leq x < 1 \\
        2 \ \ \text{if} \ \ 1 \leq x < 2 \\
        0 \ \ \text{if} \ \ x > 2    
    \end{cases}
\end{equation}

\image{0.8}{burger-characteristics.png}{
Source: https://calculus7.org/2015/11/27/rarefaction-colliding-with-two-shocks/
}

The picture above shows multiple interesting features of shocks.

Rarification wave is pulling both shocks together so the eventually collide

\paragraph{Entropy Increase}

The second law of thermodynamics requires that entropy must increase across a normal shock
wave.

\begin{equation}
    \Delta s = c_v \ln\left[\frac{\pres_2}{\pres_1}\left(\frac{\dens_1}{\dens_2}\right)\right]
\end{equation}
