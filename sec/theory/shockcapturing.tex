\subsection{Shock Capturing}

In hypersonic simulations the solver has to deal with strong shocks in an
accurate and robust manner. The utilized shock capturing strategy consists of
two parts: sensoring and capturing.

This endeavor, however, is far from trivial because of two main reasons.
The first is that the exact solution of (nonlinear) purely convective problems
develops discontinuities in finite time; the second is that these solutions might
display a very rich and complicated structure near such discontinuities. Thus,
when constructing numerical methods for these problems, it must be 
guaranteed that the discontinuities of the approximate solution are the physically
relevant ones. Also, it must be ensured that the appearance of a discontinuity
in the approximate solution does not induce spurious oscillations that spoil
the quality of the approximation; on the other hand, while ensuring this, the
method must remain sufficiently accurate near that discontinuity in order to
capture the possibly rich structure of the exact solution.



\subsubsection{Sensoring}

Many shock sensors have been developed. Within \FLEXI following indicators
are available.

\begin{description}
    \item[Jameson]\cite{jameson1981}
    \item[Ducros] \cite{ducros1999}
    \item[Persson] \cite{persson2006}
\end{description}

Based on the \textsc{Persson} indicator we develop a \emph{smoothness} sensor.
The basic idea is to find a meassure for the variance of the highest
frequencies in modal space of the polynome.

First we express the solution of order $p$ within each eleement in terms of
an orthogonal basis as
\begin{equation}
    u = \sum^{N(p)}_{i=1} u_i \psi_i,
\end{equation}
where $N(p)$ is the total number of terms in the expansion and $\psi_i$ are the
\textsc{Legendre} basis functions.
\image{0.6}{expansion-modes.png}{Expansion modes for the Lagrange and the Legendre Basis.}


Now we only consider the terms up to order $p-1$,
that is
\begin{equation}
    \hat{u} = \sum^{N(p)}_{i=1} u_i \psi_i,
\end{equation}
Whithin each element $\Omega$ we define the following \emph{smoothness} indicator
\begin{equation}
    s = \log_{10} \frac{\langle u - \hat{u}, u - \hat{u} \rangle}{\langle u, u \rangle},
\end{equation}
where $\langle \cdot,\cdot \rangle$ is the standard inner product in $L_2(\Omega)$.

The smaller the indicator $s$, the smoother is the approximating solution.  By
setting a specific threshold for $s$ one can decide when to switch between DG
and FV mode. This procedure is done at every timestep hence the elements in FV mode
should follow along the shock waves throughout the domain.

\subsubsection{Capturing}



\begin{description}
    \item[Entropy Stable Flux] \cite{derigs2016}
    \item[Local Finite Volume] 
    \item[Split Schemes] See \cite{gassner2016} for further details.
    \item[Artificial Viscosity] ...
\end{description}

\begin{equation}
\label{eq:persson-indicator-gov-equ}
    \frac{\partial\vec{U}}{\partial t} + \nabla \cdot \vec{F} = \nabla \cdot (\epsilon \nabla U)
\end{equation}

The amount of viscosity varies for each element depending on the current shock
strength. We have to consider two cases.

If the element is in FV mode the AV is set quadratic proportional to the
maximal RMS $v_{rmsv}$ within the element.
\begin{equation}
    \epsilon = \epsilon_0 \cdot \max(v_{rmsv})^2    
\end{equation}
This step is necessary for FV schemes who do not handle strong shocks via their
flux schemes.

In case of DG mode the amount of AV is based on the \emph{Persson Indicator}
introduced above.
\begin{equation}
    \epsilon = \begin{cases}
        0 & \text{if}\;\;\; s < s_0 - \kappa \\
        \epsilon_0 & \text{if}\;\;\; s > s_0 + \kappa \\
        \frac{\epsilon_0}{2} \left ( 1 + \sin\frac{\pi(s-s_0)}{2\kappa} \right ) & \text{else}
    \end{cases}
\end{equation}
The parameters $\epsilon_0$ and $\kappa$ are chosen empirically.  Since we do
not have any natural viscosity the artifical one must be as small as possible;
just enough for diffusing velocity spikes in the presence of shocks. Typical
values are around $\epsilon_0 \propto 10^{-10}$.
