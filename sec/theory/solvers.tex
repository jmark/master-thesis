\subsection{Finite Element Scheme}

Solving PDEs numerically comes down to discretizing an original continuous
problem. Most approaches consist of in essence three major generally
interchangeable modules.

a) Divide the problem domain into adjunct self-contained sub-domains, called
elements or cells.

b) Apply an averaging function or polynomial over every sub-domain.

c) Define a flux function through which ever cell communicates with its
adjacent neighbors.

REA Algorithm

Briefly, we are going to take a closer look at the class of \emph{Finite
Elements} approach where the \emph{Finite Volume} and the \emph{Discontinuous
Galerkin} are concrete implementations are. At first the very physical domain
$\Omega$ is divided into a \emph{mesh} of adjunct self-contained sub-domains
$\Omega_l$ ($l \in \mathbb{N})$ with concisely defined boundaries. For the rest
of this text we omit the element index $l$. So $\Omega$ is now the domain within
an element. The unknown solution $\vec{U}$ is replaced by polynomes of order $N_p$
constructed from linear combinations of orthogonal basis functions
$\vec{\Psi}^j$.

multi-index

\begin{equation}
    U_{\vec{I}}(t,x,y,z) \approx p_{\vec{I}}(t,x,y,z) = \sum^{N_p}_{\vec{J}=0} U^{\vec{J}}_{\vec{I}}(t) \, \Psi^{\vec{I}}(x,y,z)
\end{equation}
\remark Usally, the polynomes of every element are transformed to a reference
space $\hat{\Omega} = [-1,1]^3$ where the actual interpolation takes place. This
meassure massively increases efficiency since the basis functions are equal
among all elements. For the sake of simplicity this step is ommitted here.

Remembering the general weak formulation of the solution integral, derived in ...
The treatment is the same for all five conservative variables of the Euler
equation hence we ignore the index $i$. T
\begin{align}
    &\int_\Omega \left(\sum^{N_p}_{j=0} (\partial_t U^j(t)) \, \Psi^j(\vec{x})\right)\,\phi(\vec{x}) d^3x 
        + \int_\Omega S(t) \, \phi(\vec{x}) d^3x = \nonumber \br
         &\int_{\partial\Omega} \left[F(t)\,\phi(\vec{x})\,n_x + G(t)\,\phi(\vec{x})\,n_z + H(t)\,\phi(\vec{x})\,n_z \right] d^2x, \br
        -&\int_\Omega \left[
              \left(\sum^{N_p}_{j=0} F^j(t)\,\Psi^j(\vec{x})\right)\,\partial_x \phi(\vec{x}) 
            + \left(\sum^{N_p}_{j=0} G^j(t)\,\Psi^j(\vec{x})\right)\,\partial_y \phi(\vec{x})
            + \left(\sum^{N_p}_{j=0} H^j(t)\,\Psi^j(\vec{x})\right)\,\partial_z \phi(\vec{x}) \right] d^3x \nonumber
\end{align}
If we associate the basis functions $\Psi^{\vec{J}} := L^{\vec{J}}$ and the test functions $\phi := L^{\vec{I}}$ with the
\vip{Lagrange} polyonmes of equal order $N_p$, we can formulate an interpolation and
integration scheme (\emph{collocation}) over the domain $\Omega$.
\begin{equation}
l^j(x) = \prod^p_{k=0,k\neq j} \frac{x-x_k}{x_j-x_k}, \ j = 0,\dots,p,
\end{equation}
whith the \vip{Kronecker} property $l^j(x_i) = \delta_{ij}$. One gets to three-dimensional
formulation via the \emph{Tensor Product Ansatz}.
\begin{equation}
\label{eq-tensor-ansatz}
    L^{\vec{I}}(\vec{x}) = L^{ijk}(x,y,z) = \sum^{N_p}_{i,j,k = 0} f^{ijk} \cdot l^i(x) \cdot l^j(y) \cdot l^k(z)
\end{equation}






\paragraph{Flux Functions}

Roe, Rusanov, Lax-Wendroff, ES, ...
