\subsection{Finite Element Scheme}

Solving PDEs numerically comes down to discretizing an original continuous
problem. Most approaches consist of in essence three major generally
interchangeable modules.

REA Algorithm

a) Divide the problem domain into adjunct self-contained sub-domains, called
elements or cells.

b) Apply an averaging function or polynomial over every sub-domain.

c) Define a flux function through which ever cell communicates with its
adjacent neighbors.

\begin{description}
\item[Mesh] A mesh consists of either cells or elements. The mesh can be
structured or unstructured. It contains the necessary information where to find
cells/elements and what their (spatial) relationship to neighbors are.

\item[Grid] Regular/Irregular, grid spaces, array of points/nodes.

\item[Cell] The atomic container type of a grid. They contain the actual data
which can be a scalar, arrays of scalars, vectors, tensors, etc. What cells
distinguish from points is that they have an expanse. Hence, one must specificy
if the data is defined in the cell-center, at their corners or at their faces.

\item[Element] Elements are spatially extended objects like cells. However,
they group a list of points called \emph{nodes} on which the data is pinned on.
When an element interacts with the outside world it must extract the necessary
values from these nodes via polynomial interpolation.
\end{description}
