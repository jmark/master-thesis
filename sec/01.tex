\section{Interpolation Test Cases}

This section displays a variety of test cases in order to confirm the
correctness of the implementation. The two-dimensional interpolation is
implemented in the same manner as the three-dimensinal. As expected functions
of sufficient low polynomial order get interpolated exactly.

\subsection{2D Interpolation in one element}

Here we see the interpolation of two-dimensional functions within an element
consisting of four nodes as usual. If not specified otherwise, the anchor
points are four-times-four BCG nodes.

\image{1.0}{lagrange-interpolation-2d-plane.png}{Third-order 2d interpolation
of the ascending plane.\\ rms = 5.393455 | rms error = 0.000000 | relative rms
error = 0.000000\%}

\image{1.0}{lagrange-interpolation-2d-polynom.png}{Third-order 2d interpolation
of a second-order polynome.\\ rms = 23.398272 | rms error = 0.000000 | relative
rms error = 0.000000\%}

\image{1.0}{lagrange-interpolation-2d-sinus1.png}{Third-order 2d interpolation
of a moderate sine-cosine superposition.\\ rms = 1.000000 | rms error =
0.028277 | relative rms error = 2.827652\%}

\image{1.0}{lagrange-interpolation-2d-sinus2.png}{Third-order 2d interpolation
of a more difficult sine-cosine superposition.\\ rms = 0.974679 | rms error =
0.110654 | relative rms error = 11.352840\%}

\image{1.0}{lagrange-interpolation-2d-sinus3.png}{Third-order 2d interpolation
of an incommensurable sine-cosine superposition. \\
rms = 2.280351 | rms error = 3.157907 | relative rms error = 138.483376\%}

\image{1.0}{lagrange-interpolation-2d-sampling-comparison-1.png}{Comparison of
third-order 2d interpolation with different sampling grids.}

\image{1.0}{lagrange-interpolation-2d-sampling-comparison-2.png}{Comparison of
third-order 2d interpolation with different sampling grids.}

\subsection{3D Interpolation over the grid}

Following compares the original data on the FLASH grid (BCG) to the
back-interpolated data of FLEXI (BCG!).

\image{1.0}{lagrange-interpolation-3d-plane.png}{
Ascending plane in x-direction: Third-order 3d interpolation of \textsc{flash} data to \textsc{flexi} data.\\
rms = 0.404423 | rms error = 0.000000 | relative rms error = 0.000000\%}

\image{1.0}{lagrange-interpolation-3d-gaussian.png}{
3D Gaussian: Third-order 3d interpolation of \textsc{flash} data to \textsc{flexi} data.\\
rms = 0.125497 | rms error = 0.000000 | relative rms error = 0.000000\%}

\image{1.0}{lagrange-interpolation-3d-gaussian-ray-nvisu-4.png}{
3D Gaussian - ray through the center: Third-order 3d interpolation of
\textsc{flash} data to \textsc{flexi} data.}

\image{1.0}{lagrange-interpolation-3d-plane-wiggle.png}{
Ascending plane in all directions superposed by sines and cosines. Third-order 3d
interpolation of \textsc{flash} data to \textsc{flexi} data.\\
rms = 1.377285 | rms error = 0.000000 | relative rms error = 0.000000\%}

\image{1.0}{lagrange-interpolation-3d-plane-wiggle-ray-nvisu-8.png}{
Ascending plane in all directions superposed by sines and cosines - ray through
the center: Third-order 3d interpolation of \textsc{flash} data to
\textsc{flexi} data.}

\image{1.0}{lagrange-interpolation-3d-steps.png}{
Stress test: 3D superposed sines and cosines with random noise: Third-order 3d interpolation
of \textsc{flash} data to \textsc{flexi} data. \\
rms = 2.731093 | rms error = 0.409321 | relative rms error = 14.987457\%}

The bottom line is that for nice and smooth setups the transformation scheme is
exact up to machine precision. Nasty discontinuities cause the Lagrange
polynomial to break out but within bearable magnitude. The global solution
still remains stable and resembles the overall structure. More on this in the
next chapter. Remark: Considering
figure \ref{fig-lagrange-interpolation-3d-plane-wiggle-ray-nvisu-8.png}, how can this
setup still yield zero error estimate? The green curve depicts more points than
just the (back-interpolated) BCG nodes. The error estimate only considers the
values at the same position as the FLASH data which coincide. 
