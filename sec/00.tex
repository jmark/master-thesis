In our last meeting we discussed how to proceed after I got FLEXI successfully
running. Our main goal is to compare initally identical MHD turbulence setups
in FLASH and in FLEXI. One option is to transfer a fully developed turbulence
stored in a FLASH snapshot file to a format FLEXI understands. The other option
is using an initally smooth velocity field which leads to turbulence during
simulation. Interpolation is in order. So before that the
transfer/interpolation scheme from FLASH all the way through FLEXI has to be
tested on sensible test cases. The focus lies on the correctnes of handling
smooth setups and the resolution of shocks. Following document testifies just
that, shows many plots, reveals encountered problems and puts the result into
perspective.

\section{Definitions and Preliminary Remarks}

\subsection{Grids, Cells and Elements}
Since there is no precise nomenclature yet for grids, cells and elements, I
will propose one for this document. The following definitions reflect the
common way of implementation in actual code as well and shall be framework
agnostic. This is open for debate of course.

\begin{description}
\item[Grid] A grid consists of either cells or elements. The grid can be
structured or unstructured. It contains the necessary information where to find
cells/elements and what their (spatial) relationship to neighbors are.

\item[Cell] The atomic container type of a grid. They contain the actual data
which can be a scalar, arrays of scalars, vectors, tensors, etc. What cells
distinguish from points is that they have an expanse. Hence, one must specificy
if the data is defined in the cell-center, at their corners or at their faces.

\item[Element] Elements are spatially extended objects like cells. However,
they group a list of points called \emph{nodes} on which the data is pinned on.
When an element interacts with the outside world it must extract the necessary
values from these nodes via polynomial interpolation.
\end{description}

Specifically for this document, it is important to keep in mind that cells
refer to the \emph{finite volumes (FV)} of the \textsc{FLASH} grid and elements
to \textsc{HOPR} mesh scheme. Remark: No nodal information is stored in the
\textsc{HOPR} mesh file. \emph{FLEXI} on the other hand just stores the data as
arrays of scalars refering to an element in the mesh. The exact assignment of
values to the nodes is based on convention only.

\subsection{Grid Spaces and Transformation}
In this work, four grid spaces are of importance: \emph{Face-centered grid
(FCG)}, \emph{body-centered grid (BCG)}, \emph{Gauss nodal grid (GNG)} and
\emph{Gauss-Lobatto nodal grid (LNG)}. A visual representation can be found in
figure \ref{fig-grid-spaces.png}. This figure shows a one-dimensional grid of
eight cells (dashed lines) or alternatively a grid with two elements (thick
lines) each consisting of four nodes which implies a polynomial order of three.
Major part of the work is the transformation back and forth between these
grid spaces via interpolation. Another significant aspect is the relationship
of elements to cells. First one overlays both grids. Ideally, they are of the
same shape and cover the same physical domain. When transforming between both
grid types, cells get grouped together in numbers equal to the nodal number of
the superincumbent element. Interpolation happens in each group/element
independently.

\image{0.6}{grid-spaces.png}{Nodes in a two element grid each consisting of
four cells. From bottom to top: face-centered, body-centered, Gauss nodes (n =
3), Gauss-Lobatto nodes (n = 3).}

\subsection{1D Lagrange Interpolation}

For illustrative purposes we begin with the simplest case: one-dimensional
\textsc{Lagrange} interpolation.
\image{0.8}{lagrange-interpolation-1d.png}{One-dimensinal Lagrange
interpolation with four sample nodes.}
The Lagrange polynome of third order needs four anchor nodes with their
associated values in order to interpolate any other point in between. This
works very well and does not cause any headaches. Unfortunately, considering
previous section, extrapolation is necessary, too. Lagrange polynomes tend to
explode going further away from the outer anchor nodes. In pathological
interpolation cases this effect yields erroneous results. The end of the
document shows specific examples.

\subsection{Multidimensional Lagrange Interpolation}

Since we operate in three-dimensional space we need to generalize
the one-dimensional Lagrange interpolation to higher dimensions. One
approach is called: \emph{Multivariate Lagrange Interpolation}. Their
exists extensive amout of literature about this. Interestingly, this
method needs lesser anchor nodes than the following one, we use: \emph{Tensor
Ansatz Lagrange Interpolation}. The ansatz is straigtforward. For the three
dimensional case it reads:

\begin{equation}
\label{eq-tensor-ansatz}
    p(x,y,z) = \sum^{n_x,n_y,n_z}_{i,j,k = 0} f_{ijk} \cdot l_i(x) \cdot l_j(y) \cdot l_k(z)
\end{equation}

Life is easy, so we set $n = n_x = n_y = n_z$. If we go crazy we could
introduce tensor notation.

\begin{equation}
\label{eq-tensor-ansatz}
    p(x,y,z) = \mathbb F \odot (\vec{l(x)} \otimes \vec{l(y)} \otimes \vec{l(z)})
\end{equation}

where $\odot$ is the \emph{relentless} contraction operator. If the target
nodes $x_q, y_r, z_s$ are known beforehand, $(\vec{l(x)} \otimes \vec{l(y)} \otimes \vec{l(z)})$
can be precomputed and stays the same for all interpolations within an element
and it's cell group.

\subsection{Transformation from \textsc{FLASH} to \textsc{FLEXI}}

All following examples/figures in this document are produced by the very same
procedure where we set the interpolation order $n = 3$. First the intial values
are generated (or read) in the FLASH grid format. The grid is split into groups
of $4^3 = 64$ neighboring cells. These groups get mapped to the associated
element of the HOPR grid. Looping over all group-element pairs a
three-dimensional Lagrange Polynome gets constructed according to the
body-centered values of the cells. Then the new values at the
Gauss/Gauss-Lobatto nodes get interpolated and stored according to the ordering
convention of FLEXI.

In order to do useful analysis and visualization, the FLEXI data gets again
back-interpolated to BCG. This is similar to what the visualization routines in
FLEXI do.

\subsection{Interpolation Error Estimate}

As a meassure of interpolation error of an original function $f$ and its
interpolated counterfeit $\widetilde{f}$ I opted for the relative
root-means-sqare variance. Taking the absolute error would of course be another
valid option.

\begin{equation}
    \text{rms} = \sqrt{\frac{1}{N} \sum^N_i f^2_i} 
\end{equation}

\begin{equation}
    \text{rmse} = \sqrt{\frac{1}{N} \sum^N_i (f_i - \widetilde{f_i})^2} 
\end{equation}

\begin{equation}
    \text{relative rmse} = \frac{\text{rmse}}{\text{rms}} 
\end{equation}

One has to ensure that $f_i$ and $\widetilde{f_i}$ live in the same grid space.
