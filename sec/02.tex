\section{3D Interpolation of Shocks / Real world data}

\image{1.0}{ekin-over-time-b5-128-256.png}{Total kinetic energy over crossing time.}

\subsection{First approach for $128^3$}
\paragraph{13.10.2016}
Following figures depict the very same as the previous section but with real
world FLASH data of a fully developed turbulence.

\image{1.0}{lagrange-interpolation-3d-turbulence.png}{
Fully developed turbulence: Third-order 3d interpolation
of \textsc{flash} data to \textsc{flexi} data. \\
rms = 3.162914 | rms error = 0.879976 | relative rms error = 27.821685\%}

\image{1.0}{lagrange-interpolation-3d-turbulence-nvisu-4.png}{
Fully developed turbulence - ray through the center: Third-order 3d
interpolation of \textsc{flash} data to \textsc{flexi} data. The vertical
dashed lines depict the boundaries of an element.}

\image{1.0}{column-density-128.png}{Summed columns along z-axes through the box. Right
side is interpolated two twice the grid size.}

\paragraph{rms error}
rms = 3.162914 | rms error = 1.240110 | relative rms error = 39.207842\%

As one can see the result is not satisfiying. Shocks do not get resolved very
well with this scheme. Suprisingly, the huge jumps within an element do not
cause any harm. This turbulence snapshot has a resolution of $128^3$ nodes in
BCG space. Hence, higher resolution like $256^3$, which is necessary for
serious simulation anyway, in the first place would mitigate interpolation
errors considerably.
 
The bigger problem is the necessary extrapolation near the element boundaries
during conversion from BCG space to nodal space. As noted in the beginning
Lagrange polynomes tend to explode outside their anchor nodes. In conjunction
with the influence of the other dimensions this might explain the shift of the
interpolant at shocks. To exemplify this phenomena following artificial shock
cases were generated.

\image{1.0}{lagrange-interpolation-3d-steps-ray-nvisu-8.png}{
Superposed sines and cosines - ray through the center: Third-order 3d interpolation of
\textsc{flash} data to \textsc{flexi} data. The vertical
dashed lines depict the boundaries of an element.}

\image{1.0}{lagrange-interpolation-3d-steps-ray-nvisu-8-1.png}{ Superposed
sines and cosines - ray through the center: Third-order 3d interpolation of
\textsc{flash} data to \textsc{flexi} data. The vertical dashed lines depict
the boundaries of an element.}

Bearing figure \ref{fig-grid-spaces.png} (comparison of grid spaces) in mind
extrapolation from BCG to nodal space clearly yields erroneous values.  Remedy
would be to enforce the Lagrange polynome to consider the via a different
method obtained values at element boundaries. If one could achieve that I would
bet for considerably better shock resolution. Suggestions are welcome.

\subsection{Second approach for $256^3$ and neighboring cells}

\paragraph{07.11.2016} ...
As proposed above two approaches yielded considerably better results. First,
introducing the neighboring cells. Second, going to higher resolution. The data
used for all figures below is not filtered yet for unphysical values like
negative density.

\image{1.0}{ray-density-256_a.png}{Ray through the box.}
\image{1.0}{ray-density-256_b.png}{Ray through the box.}
\image{1.0}{ray-density-256_c.png}{Ray through the box.}

\image{1.0}{slice-density-256.png}{Slice through the box.}

\image{1.0}{column-density-256.png}{Summed columns along z-axes through the box. Right
side is interpolated two twice the grid size.}

\subsubsection{Error estimates and conserved quantities}

\paragraph{rms error}
rms = 3.153845 | rms error = 0.503812 | relative rms error = 15.974545\%

\paragraph{total error}
tot = 16777216 | total error = 1377048.424 | relative total error = 8.207848\%

\paragraph{rms error (negative filter)}
rms = 3.153845 | rms error = 0.489323 | relative rms error = 15.515110\%

\paragraph{total error (negative filter)}
tot = 16777216 | total error = 1321598.559 | relative total error = 7.877341\%

\paragraph{total mass}
mass flash = 1.0 | mass flexi = 0.9999913 | $\Delta M$ = 8.63261 $10^{-6}$

\paragraph{total mass (negative filter)}
mass flash = 1.0 | mass flexi = 1.003296 | $\Delta M$ = 3.29643 $10^{-3}$
