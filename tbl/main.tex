%======================================================================
%	Vorlage
%======================================================================
%	$Id$
%	Matthias Kupfer
%======================================================================
%	Documentclass
%======================================================================
\documentclass[
	11pt,			% 
	a4paper,		% DIN A4
]{scrartcl}		% Zweiseitig

\usepackage[utf8]{inputenc}

\usepackage{
	calc,			% Erweiterung der arithmetischen Funktionen in 
	color,			% im Laufenden Text einfach mit \color{Farbe) zwischen den 
	float,			% Positionierung von Gleitobjekten genau an der Stelle, wo man
	mdwlist,		% compact list: itemize* ..
	%scrdate,		% \todaysname 
	%scrtime,		% \thistime
    supertabular,		% Tabellen > 1 Seite
	%tabularx,		% Blocksatzspalten
	marvosym,		% Euro etc.
}
\usepackage{amsmath}
%\usepackage{amsthm}
\usepackage{amssymb}
\usepackage[amsmath,thmmarks,amsthm]{ntheorem}

%\newtheorem{proof}{Beweis}

\usepackage[]{units}
\usepackage{booktabs}
\usepackage{tabularx}
\usepackage{lscape}
\usepackage{bbold}
\usepackage{mathtools}

%======================================================================
%	Bilder, Links
%======================================================================

\usepackage{hyperref}
\usepackage{geometry}
\geometry{textwidth=15cm}

\usepackage{lmodern}
\usepackage[T1]{fontenc}
\usepackage[ngerman]{babel}
\usepackage[autostyle=true,german=quotes]{csquotes}

\usepackage[
    justification=justified,
    singlelinecheck=false,
    font=small,
    format=plain,
    labelfont=bf,
    up,
    aboveskip=2pt,
    center
]{caption}
\usepackage{graphicx}

\usepackage{listings}
\usepackage{parskip}
\usepackage{gnuplot-lua-tikz}

\lstset{ %
    breaklines=true,
    basicstyle=\scriptsize\sffamily,
    captionpos=b,
    frame=single,
    framexleftmargin=1.5em,
    xleftmargin=16pt,
    keepspaces=true,
    aboveskip=0.5cm,
    belowskip=0.3cm,
    lineskip=4pt,
    %keywordstyle=\bf,
    language=Mathematica,
    numbers=left,
    numbersep=5pt,
    numberstyle=\color{black},
    rulecolor=\color{black},
    morekeywords={Quiet},
    alsoletter={\#},
    %emph={map,jacobian,initV,steps,orbit,initQ,initR,diagsR},
    %emphstyle={\color{green}},
    emph={[2]\#},emphstyle={[2]\color{green}}
}
\usepackage{dcolumn}
% \renewcommand{\lstlistingname}{Code}

% \setcounter{tocdepth}{3}

% \fussy			% viele Worttrennungen, "schönere" Wortabstände

\usepackage{stmaryrd}
\usepackage{float}
\usepackage{algpseudocode}

\newtheorem{mydef}{Definition}[section]
\newtheorem{lemma}[mydef]{Lemma}
\newtheorem{satz}[mydef]{Satz}
\newtheorem{kor}[mydef]{Korollar}
\newtheorem{prop}[mydef]{Proposition}
\newcommand{\inner}[2]{\left\langle #1, #2 \right\rangle}

\usepackage{siunitx}

\lstset{ %
    breaklines=true,
    basicstyle=\scriptsize\sffamily,
    captionpos=b,
    frame=single,
    framexleftmargin=1.5em,
    xleftmargin=16pt,
    keepspaces=true,
    aboveskip=0.5cm,
    belowskip=0.3cm,
    lineskip=4pt,
    language=C++,
    numbers=left,
    numbersep=5pt,
    numberstyle=\color{black},
    rulecolor=\color{black},
    morekeywords={Quiet},
    alsoletter={\#},
    emph={[2]\#},emphstyle={[2]\color{green}}
}

\renewcommand{\lstlistingname}{Code}


\newcommand{\IMAGE}[4]
{
    \begin{figure}[H]
    \centering
    \includegraphics[width=#1\textwidth]{./img/#2}
    \caption{#4}
    \label{#3}
    \end{figure}
}
% example:
% \IMAGE{1.0}{./img/blubb.png}{This is the caption.}{fig-label-blubb}

\newcommand{\HRule}{\rule{\linewidth}{0.5mm}}

\renewcommand{\vec}[1]{\underline{#1}}
\newcommand{\mat}[1]{\underline{\underline{#1}}}
%\renewcommand{\qed}[0]{$\square$}
\newcommand{\norm}[1]{\left|\left|#1\right|\right|}

\newcommand{\ceil}[1]{\lceil #1 \rceil}
\newcommand{\floor}[1]{\left\lfloor #1 \right\rfloor}

\newcommand{\Gaense}[1]{\textquotedblleft #1\textquotedblright}

\setcounter{section}{0}

\raggedbottom
\begin{document}

\newcommand{\expNr}{04}
\newcommand{\expTitle}{}
\newcommand{\releaseDate}{\today}

\begin{titlepage}
\begin{center}

\includegraphics[width=0.8\textwidth]{./frp/uni_koeln_logo.jpg}\\[1.0cm]  

\textsc{\LARGE Master Thesis}
\textsc{\Large }\\[0.8cm]
\textsc{\Large Theoretical Astrophysics}\\[0.5cm]

% Title
\HRule\\[0.7cm]
{ \LARGE \bfseries \textquotedblleft Modelling turbulent gases with Finite Volume \\[0.4cm] and Discontinuous Galerkin methods \textquotedblright}\\[0.4cm]
\HRule \\[0.7cm]

\vspace{1cm}
\large \emph{submitted by}\\[0.4cm]
\textsc{\LARGE Johannes Markert}\\[1cm]
\vspace{0.2cm}

\large \emph{in fulfilment of the requirements for the academic degree}\\[0.4cm]
\textsc{\LARGE Master of Science}
\vspace{2.0cm}

\includegraphics[width=0.4\textwidth]{./frp/logo-astrophysik.png}\\
\textsc{\large I. Physikalisches Institut}\\[1cm]

\vfill
K\"oln - \releaseDate 

\end{center}
\end{titlepage}

% \newpage 
% \thispagestyle{empty}
% \quad 
% \newpage

\cleardoubleemptypage


%\pagenumbering{gobble}
%\tableofcontents

\cleardoublepage
\pagenumbering{arabic}
\setcounter{page}{1}

\paragraph{13.10.2016}
In our last meeting we discussed how to proceed after I got FLEXI successfully
running. Our main goal is to compare initally identical MHD turbulence setups
in FLASH and in FLEXI. One option is to transfer a fully developed turbulence
stored in a FLASH snapshot file to a format FLEXI understands. The other option
is using an initally smooth velocity field which leads to turbulence during
simulation. Interpolation is in order. So before that the
transfer/interpolation scheme from FLASH all the way through FLEXI has to be
tested on sensible test cases. The focus lies on the correctnes of handling
smooth setups and the resolution of shocks. Following document testifies just
that, shows many plots, reveals encountered problems and puts the result into
perspective.

\paragraph{07.11.2016}
Answer to the resolution problem of turbulences via 3D Lagrange interpolation.

\section{Definitions and Preliminary Remarks}

\subsection{Grids, Cells and Elements}
Since there is no precise nomenclature yet for grids, cells and elements, I
will propose one for this document. The following definitions reflect the
common way of implementation in actual code as well and shall be framework
agnostic. This is open for debate of course.

\begin{description}
\item[Mesh] A mesh consists of either cells or elements. The mesh can be
structured or unstructured. It contains the necessary information where to find
cells/elements and what their (spatial) relationship to neighbors are.

\item[Grid] Regular/Irregular, grid spaces, array of points/nodes.

\item[Cell] The atomic container type of a grid. They contain the actual data
which can be a scalar, arrays of scalars, vectors, tensors, etc. What cells
distinguish from points is that they have an expanse. Hence, one must specificy
if the data is defined in the cell-center, at their corners or at their faces.

\item[Element] Elements are spatially extended objects like cells. However,
they group a list of points called \emph{nodes} on which the data is pinned on.
When an element interacts with the outside world it must extract the necessary
values from these nodes via polynomial interpolation.
\end{description}

Specifically for this document, it is important to keep in mind that cells
refer to the \emph{finite volumes (FV)} of the \textsc{FLASH} grid and elements
to \textsc{HOPR} mesh scheme. Remark: No nodal information is stored in the
\textsc{HOPR} mesh file. \emph{FLEXI} on the other hand just stores the data as
arrays of scalars refering to an element in the mesh. The exact assignment of
values to the nodes is based on convention only.

\subsection{Grid Spaces and Transformation}
In this work, four grid spaces are of importance: \emph{Face-centered grid
(FCG)}, \emph{body-centered grid (BCG)}, \emph{Gauss nodal grid (GNG)} and
\emph{Gauss-Lobatto nodal grid (LNG)}. A visual representation can be found in
figure \ref{fig-grid-spaces.png}. This figure shows a one-dimensional grid of
eight cells (dashed lines) or alternatively a grid with two elements (thick
lines) each consisting of four nodes which implies a polynomial order of three.
Major part of the work is the transformation back and forth between these
grid spaces via interpolation. Another significant aspect is the relationship
of elements to cells. First one overlays both grids. Ideally, they are of the
same shape and cover the same physical domain. When transforming between both
grid types, cells get grouped together in numbers equal to the nodal number of
the superincumbent element. Interpolation happens in each group/element
independently.

\image{0.6}{grid-spaces.png}{Nodes in a two element grid each consisting of
four cells. From bottom to top: face-centered, body-centered, Gauss nodes (n =
3), Gauss-Lobatto nodes (n = 3).}

\subsection{1D Lagrange Interpolation}

For illustrative purposes we begin with the simplest case: one-dimensional
\textsc{Lagrange} interpolation.
\image{0.8}{lagrange-interpolation-1d.png}{One-dimensinal Lagrange
interpolation with four sample nodes.}
The Lagrange polynome of third order needs four anchor nodes with their
associated values in order to interpolate any other point in between. This
works very well and does not cause any headaches. Unfortunately, considering
previous section, extrapolation is necessary, too. Lagrange polynomes tend to
explode going further away from the outer anchor nodes. In pathological
interpolation cases this effect yields erroneous results. The end of the
document shows specific examples.

\subsection{Multidimensional Lagrange Interpolation}

Since we operate in three-dimensional space we need to generalize
the one-dimensional Lagrange interpolation to higher dimensions. One
approach is called: \emph{Multivariate Lagrange Interpolation}. Their
exists extensive amout of literature about this. Interestingly, this
method needs lesser anchor nodes than the following one, we use: \emph{Tensor
Ansatz Lagrange Interpolation}. The ansatz is straigtforward. For the three
dimensional case it reads:

\begin{equation}
\label{eq-tensor-ansatz}
    p(x,y,z) = \sum^{n_x,n_y,n_z}_{i,j,k = 0} f_{ijk} \cdot l_i(x) \cdot l_j(y) \cdot l_k(z)
\end{equation}

Life is easy, so we set $n = n_x = n_y = n_z$. If we go crazy we could
introduce tensor notation.

\begin{equation}
\label{eq-tensor-ansatz}
    p(x,y,z) = \mathbb F \odot (\vec{l(x)} \otimes \vec{l(y)} \otimes \vec{l(z)})
\end{equation}

where $\odot$ is the \emph{relentless} contraction operator. If the target
nodes $x_q, y_r, z_s$ are known beforehand, $(\vec{l(x)} \otimes \vec{l(y)} \otimes \vec{l(z)})$
can be precomputed and stays the same for all interpolations within an element
and it's cell group.

\subsection{Transformation from \textsc{FLASH} to \textsc{FLEXI}}

All following examples/figures in this document are produced by the very same
procedure where we set the interpolation order $n = 3$. First the intial values
are generated (or read) in the FLASH grid format. The grid is split into groups
of $4^3 = 64$ neighboring cells. These groups get mapped to the associated
element of the HOPR grid. Looping over all group-element pairs a
three-dimensional Lagrange Polynome gets constructed according to the
body-centered values of the cells. Then the new values at the
Gauss/Gauss-Lobatto nodes get interpolated and stored according to the ordering
convention of FLEXI.

In order to do useful analysis and visualization, the FLEXI data gets again
back-interpolated to BCG. This is similar to what the visualization routines in
FLEXI do.

\subsection{Interpolation Error Estimate}

As a meassure of interpolation error of an original function $f$ and its
interpolated counterfeit $\widetilde{f}$ I opted for the relative
root-means-sqare variance. Taking the absolute error would of course be another
valid option.

\begin{equation}
    \text{rms} = \sqrt{\frac{1}{N} \sum^N_i f^2_i} 
\end{equation}

\begin{equation}
    \text{rmse} = \sqrt{\frac{1}{N} \sum^N_i (f_i - \widetilde{f_i})^2} 
\end{equation}

\begin{equation}
    \text{relative rmse} = \frac{\text{rmse}}{\text{rms}} 
\end{equation}

One has to ensure that $f_i$ and $\widetilde{f_i}$ live in the same grid space.

Another error estimate could be the absolut total error:

\begin{equation}
    \text{tot} = \sum^N_i |f_i|
\end{equation}

\begin{equation}
    \text{rote} = \sum^N_i |f_i - \widetilde{f_i}|
\end{equation}

\begin{equation}
    \text{relative tote} = \frac{\text{tote}}{\text{tot}} 
\end{equation}

In most cases the total absolute error yields smaller results, since runaways
get weighted more in the rmse case. A proper handling of runaways especially in
interpolation of shocks is still an open question.

\section{Interpolation Test Cases}

This section displays a variety of test cases in order to confirm the
correctness of the implementation. The two-dimensional interpolation is
implemented in the same manner as the three-dimensinal. As expected functions
of sufficient low polynomial order get interpolated exactly.

\subsection{2D Interpolation in one element}

Here we see the interpolation of two-dimensional functions within an element
consisting of four nodes as usual. If not specified otherwise, the anchor
points are four-times-four BCG nodes.

\image{1.0}{lagrange-interpolation-2d-plane.png}{Third-order 2d interpolation
of the ascending plane.\\ rms = 5.393455 | rms error = 0.000000 | relative rms
error = 0.000000\%}

\image{1.0}{lagrange-interpolation-2d-polynom.png}{Third-order 2d interpolation
of a second-order polynome.\\ rms = 23.398272 | rms error = 0.000000 | relative
rms error = 0.000000\%}

\image{1.0}{lagrange-interpolation-2d-sinus1.png}{Third-order 2d interpolation
of a moderate sine-cosine superposition.\\ rms = 1.000000 | rms error =
0.028277 | relative rms error = 2.827652\%}

\image{1.0}{lagrange-interpolation-2d-sinus2.png}{Third-order 2d interpolation
of a more difficult sine-cosine superposition.\\ rms = 0.974679 | rms error =
0.110654 | relative rms error = 11.352840\%}

\image{1.0}{lagrange-interpolation-2d-sinus3.png}{Third-order 2d interpolation
of an incommensurable sine-cosine superposition. \\
rms = 2.280351 | rms error = 3.157907 | relative rms error = 138.483376\%}

\image{1.0}{lagrange-interpolation-2d-sampling-comparison-1.png}{Comparison of
third-order 2d interpolation with different sampling grids.}

\image{1.0}{lagrange-interpolation-2d-sampling-comparison-2.png}{Comparison of
third-order 2d interpolation with different sampling grids.}

\subsection{3D Interpolation over the grid}

Following compares the original data on the FLASH grid (BCG) to the
back-interpolated data of FLEXI (BCG!).

\image{1.0}{lagrange-interpolation-3d-plane.png}{
Ascending plane in x-direction: Third-order 3d interpolation of \textsc{flash} data to \textsc{flexi} data.\\
rms = 0.404423 | rms error = 0.000000 | relative rms error = 0.000000\%}

\image{1.0}{lagrange-interpolation-3d-gaussian.png}{
3D Gaussian: Third-order 3d interpolation of \textsc{flash} data to \textsc{flexi} data.\\
rms = 0.125497 | rms error = 0.000000 | relative rms error = 0.000000\%}

\image{1.0}{lagrange-interpolation-3d-gaussian-ray-nvisu-4.png}{
3D Gaussian - ray through the center: Third-order 3d interpolation of
\textsc{flash} data to \textsc{flexi} data.}

\image{1.0}{lagrange-interpolation-3d-plane-wiggle.png}{
Ascending plane in all directions superposed by sines and cosines. Third-order 3d
interpolation of \textsc{flash} data to \textsc{flexi} data.\\
rms = 1.377285 | rms error = 0.000000 | relative rms error = 0.000000\%}

\image{1.0}{lagrange-interpolation-3d-plane-wiggle-ray-nvisu-8.png}{
Ascending plane in all directions superposed by sines and cosines - ray through
the center: Third-order 3d interpolation of \textsc{flash} data to
\textsc{flexi} data.}

\image{1.0}{lagrange-interpolation-3d-steps.png}{
Stress test: 3D superposed sines and cosines with random noise: Third-order 3d interpolation
of \textsc{flash} data to \textsc{flexi} data. \\
rms = 2.731093 | rms error = 0.409321 | relative rms error = 14.987457\%}

The bottom line is that for nice and smooth setups the transformation scheme is
exact up to machine precision. Nasty discontinuities cause the Lagrange
polynomial to break out but within bearable magnitude. The global solution
still remains stable and resembles the overall structure. More on this in the
next chapter. Remark: Considering
figure \ref{fig-lagrange-interpolation-3d-plane-wiggle-ray-nvisu-8.png}, how can this
setup still yield zero error estimate? The green curve depicts more points than
just the (back-interpolated) BCG nodes. The error estimate only considers the
values at the same position as the FLASH data which coincide. 

\section{3D Interpolation of Shocks / Real world data}

Following figures depict the very same as the previous section but with real
world FLASH data of a fully developed turbulence.

\image{1.0}{lagrange-interpolation-3d-turbulence.png}{
Fully developed turbulence: Third-order 3d interpolation
of \textsc{flash} data to \textsc{flexi} data. \\
rms = 3.162914 | rms error = 0.879976 | relative rms error = 27.821685\%}

\image{1.0}{lagrange-interpolation-3d-turbulence-nvisu-4.png}{
Fully developed turbulence - ray through the center: Third-order 3d
interpolation of \textsc{flash} data to \textsc{flexi} data. The vertical
dashed lines depict the boundaries of an element.}

As one can see the result is not satisfiying. Shocks do not get resolved very
well with this scheme. Suprisingly, the huge jumps within an element do not
cause any harm. This turbulence snapshot has a resolution of $128^3$ nodes in
BCG space. Hence, higher resolution like $256^3$, which is necessary for
serious simulation anyway, in the first place would mitigate interpolation
errors considerably.
 
The bigger problem is the necessary extrapolation near the element boundaries
during conversion from BCG space to nodal space. As noted in the beginning
Lagrange polynomes tend to explode outside their anchor nodes. In conjunction
with the influence of the other dimensions this might explain the shift of the
interpolant at shocks. To exemplify this phenomena following artificial shock
cases were generated.

\image{1.0}{lagrange-interpolation-3d-steps-ray-nvisu-8.png}{
Superposed sines and cosines - ray through the center: Third-order 3d interpolation of
\textsc{flash} data to \textsc{flexi} data. The vertical
dashed lines depict the boundaries of an element.}

\image{1.0}{lagrange-interpolation-3d-steps-ray-nvisu-8-1.png}{ Superposed
sines and cosines - ray through the center: Third-order 3d interpolation of
\textsc{flash} data to \textsc{flexi} data. The vertical dashed lines depict
the boundaries of an element.}

Bearing figure \ref{fig-grid-spaces.png} (comparison of grid spaces) in mind
extrapolation from BCG to nodal space clearly yields erroneous values.  Remedy
would be to enforce the Lagrange polynome to consider the via a different
method obtained values at element boundaries. If one could achieve that I would
bet for considerably better shock resolution. Suggestions are welcome.

\section{Conclusion}

This document shows the correctness and stability of the current transformation
scheme from FLASH (BCG) to FLEXI (GNG/LNG) and back to BCG for various setups.
Shocks do not get resolved very well due to extrapolation errors inherent to
the Lagrange interpolation.

Smooth or not too harsh setups are transformed up to machine precision.


% \bibliography{references}{}
% \bibliographystyle{myplain}
\end{document}
