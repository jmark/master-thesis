\section*{Aufgabe 1}

\underline{Behauptung:} Sei nun $\rho =\lambda =1$. Es gilt, dass
\begin{align}
\label{eq:behauptung01}
\begin{pmatrix}
v_1\\
v_2\\
p
\end{pmatrix}
=
\begin{pmatrix}
2+\sin(2\pi(x-t))\\
4+\cos(2\pi(y-t))\\
6+\sin(2\pi(x-t))+\cos(2\pi(y-t))
\end{pmatrix}
\end{align}
die zweidimensionalen linearen Eulergleichungen (1) auf dem Gebiet
$\Omega=[0,1]\times[0,1]$ mit periodischen Randbedingungen exakt löst.

\begin{proof}
Unter den oben genannten Voraussetzungen ergibt sich:

\begin{minipage}[t]{0.5\textwidth}
\begin{align*}
\frac{dv_1}{dt}&=-2\pi \cos(2\pi(x-t)) \\
\frac{dv_1}{dx}&=2\pi \cos(2\pi(x-t)) \\
\frac{dv_2}{dt}&=2\pi \sin(2\pi(y-t)) \\
\frac{dv_2}{dy}&=-2\pi \sin(2\pi(y-t))
\end{align*}
\end{minipage}
\begin{minipage}[t]{0.5\textwidth}
\begin{align*}
\frac{dp}{dt}&=-2\pi \cos(2\pi(x-t))+2\pi \sin(2\pi(y-t)) \\
\frac{dp}{dx}&=2\pi \cos(2\pi(x-t)) \\
\frac{dp}{dy}&=-2\pi \sin(2\pi(y-t)) \\
\frac{dv_1}{dy}&=0,\quad\frac{dv_2}{dx}=0
\end{align*}
\end{minipage}

Eingesetzt in
\begin{align}
\begin{pmatrix}
v_1\\
v_2\\
p
\end{pmatrix}_t
+
\begin{pmatrix}
p\\
0\\
v_1
\end{pmatrix}_x
+
\begin{pmatrix}
0\\
p\\
v_2
\end{pmatrix}_y
=
\begin{pmatrix}
0\\
0\\
0
\end{pmatrix}
\end{align}
ergibt sich somit
\begin{align}
\begin{pmatrix}
-2\pi \cos(2\pi(x-t))\\
2\pi \sin(2\pi(y-t))\\
-2\pi \cos(2\pi(x-t))+2\pi \sin(2\pi(y-t))
\end{pmatrix}
+ \\[0.4cm]
\begin{pmatrix}
2\pi \cos(2\pi(x-t))\\
0\\
2\pi \cos(2\pi(x-t))
\end{pmatrix}
+
\begin{pmatrix}
0\\
-2\pi \sin(2\pi(y-t))\\
-2\pi \sin(2\pi(y-t))
\end{pmatrix}
=
\underline{\underline{\begin{pmatrix}
0\\
0\\
0
\end{pmatrix}}}
\end{align}
Damit wird die Behauptung, aufgestellt in Gleichung \eqref{eq:behauptung01}, bestätigt.
\end{proof}
