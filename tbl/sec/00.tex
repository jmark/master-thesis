In diesem Projekt soll das DGSEM Verfahren auf die zweidimensionalen
kompressiblen linearen Eulergleichungen (kompressiblen Akkustik-Gleichungen)
angewandt werden. Die Gleichungen sind gegegeben durch

\begin{align}
\begin{pmatrix}
v_1\\
v_2\\
p
\end{pmatrix}_t
+
\begin{pmatrix}
\frac{1}{\rho}p\\
0\\
\lambda v_1
\end{pmatrix}_x
+
\begin{pmatrix}
0\\
\frac{1}{\rho}p\\
\lambda v_2
\end{pmatrix}_y
=
\begin{pmatrix}
0\\
0\\
0
\end{pmatrix}
\end{align}

Dabei bezeichnet $v_1$, bzw. $v_2$ die Geschwindigkeit einer Akkustikwelle in
$x$-, bzw. $y$-Richtung. Der Druck der Welle ist durch $p$ gegeben.
Zusätzlich beschreiben die Konstanten $\rho > 0$ die Dichte und $\lambda > 0$ die
sogenannte „zweite Lame Konstante“.

Ziel dieses Projekts ist es, ein zweidimensionales DGSEM Verfahren zu
implementieren und zu verifizieren.  Verwendet werden im Folgenden stets das
Low- Storage-4th-Order Runge-Kutta-Verfahren des letzten Projektes zur
Zeitintegration. Der Zeitschritt wird stets adaptiv durch die CFL-Bedingung
gemäß Vorlesung bestimmt werden.
