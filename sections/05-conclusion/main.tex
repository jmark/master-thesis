\section{Conclusion \& Outlook}

A multitude of reasons are conceivable for what systematically prevents FLEXI's
schemes from pushing energy into small scales. First of all

The high-resolution finite methods discussed in these lectures can be used
very successfully for a wide range of problems. They are not foolproof, how-
however, and one should never accept computed results without a critical study
of their accuracy. This is often hard to assess for complex problems where
the exact solution is not known, but the following techniques can help:
- Investigate simple cases where exact solutions might be known, or at least
the correct qualitative behavior of the solution is well understood.
— Reduce the number of space dimensions by considering radially symmetric
solutions, for example. Then a fine-grid solution in one space dimension
can be used as a reference solution for the multi-dimensional solution.
- Perform grid refinement studies on the real problem of interest. If you
refine the grid does the solution remain basically the same? If not, then
you probably cannot trust either solution. (If so, both solutions may still
be completely incorrect. An error in the code that changes the equations
might lead to a method that converges very nicely to a solution of the
wrong equation.)
A number of specific difficulties that can arise in solving the Euler equa-
equations have already been mentioned, such as
— The use of a nonconservative method can give shocks that look reasonable
but which travel at the wrong speed (Sect. 4.3).
— Stiff source terms can lead to similar results (Sect. 5.4).
- The computed solution may not satisfy the entropy condition, leading to
discontinuities where there should be a smooth rarefaction (Sect. 4.6.4).

This same sort of artifact is also often seen when two shocks collide or when a
shock reflects off a solid wall. Consider two identical shocks approaching each
other with zero velocity in between (which also models reflection at a wall
halfway between the shocks — see Sect. 4.9.3). Each shock may have settled
down to some numerical traveling wave that does not appear to generate any
noise. But when the shocks collide, the result is two new out-going shocks
with a different state in between than before the collision, with higher density
and pressure but still zero velocity. During the interaction phase considerable
noise will be generated in the other families, and in particular a spurious
entropy wave will be generated which is then stationary in the zero-velocity
region between the out-going shocks. This wave yields a dip in the density.
The pressure, however, is nearly constant and so this dip in density results
in an increase in the temperature T — p/TZg. The gas appears to have been
heated at the point where the collision occurs.
In particular, in computing the reflection of a shock off a solid wall, this
spurious temperature rise occurs at the wall itself. This phenomenon is fre-
frequently observed in numerical simulations where shocks reflect off physical
boundaries, and is known as wall heating in the literature. See, for example,
[73], [175].


7.7 Grid-Aligned Shocks
In a multi-dimensional calculation it may seem advantageous to have a shock
aligned with the grid so that Riemann problems normal to the cell edges
are also in the physically correct direction. However, a shock that is nearly
aligned with the grid can also suffer certain numerical instabilities that have
no analog in one-dimensional calculations. Figure 7.3 shows density contours
at a sequence of times for a colliding flow problem. Initially g = p = 1,
v = 0 everywhere, while the a;-velocity is u = +20 on the left half of the
domain and u = -20 on the right. This colliding flow should give rise to two
symmetric shock waves propagating outwards. If the initial data is exactly
uniform then the calculation will yield a reasonable approximation to this.
For the calculation in Fig. 7.3, however, the density was initially perturbed

