\section{Conclusion \& Outlook}
The goal of this thesis was to investigate if it is feasible to apply a
recent third-order discontinuous Galerkin method (FLEXI) to astrophysical
turbulence simulations and how they compare to established first/second order
finite volume methods (FLASH). The objective is that DG methods are less
dissipative and better resolve small scale structures.

The primary focus was to accurately model isothermal turbulences up to Mach 10
which are, for example, important for simulating star formation rates in
molecular clouds.

As a first step FLEXI was augmented in order to support isothermal turbulence
setups in the same way as FLASH does. Modules for polytropic cooling, turbulent
forcing, bulk motion correction and shock capturing were implemented and
integrated into FLEXI (cf. \sec{computational-frameworks}). These
modifications were then tested and verified with the Sod Shock Tube problem
(cf. \sec{sod-shock-tube}).

Since pure Galerkin schemes cannot cope with discontinuities, shock
capturing routines try to dampen spurious oscillations and, as a last resort,
locally switch to second-order finite volumes. In theory, this hybrid scheme
should yield a convergence rate somewhere between 2 to 3, depending on the
ratio of DG elements to FV elements (cf. \sec{shock-capturing}).

The next step was to conduct driven turbulence simulations of Mach 2.5 with
three explicit time integration methods of ascending order (first order Euler,
second order Midpoint and third-order Runge-Kutta) each in combination with
hybrid and FV-only mode. The same simulation was also performed with three
solvers from FLASH: PPM, Bouchut3 and Bouchut5. They serve as reference
(cf. \sec{stirturb}).

In order to check the influence of the turbulent forcing, this study was
concluded by a decaying turbulence setup with identical initial conditions
among all solvers (cf. \sec{decayturb}).

Side-by-side comparison of column density snapshots at different times and
methods from turbulence statistics, namely density PDFs and powerspectra,
gained insights into the quality of the resulting turbulences.

We arrive at following conclusions:
\begin{itemize}
\item The evaluation of the simulation data is in good agreement with the
reference. Supersonic isothermal turbulent decay is modeled
equally by all analyzed solvers (cf. \sec{decayturb-column-plots} and
\sec{decayturb-PDF}). No significant difference among dissipation
rates (cf. \fig{plots/decayturb/evol-ekin-diss}) has been measured.

\item Isothermal turbulent decay, in the way applied here, is an unsuitable
setup to benchmark numerical schemes. Polytropic cooling might have a
``grinding'' effect which renders higher-order reconstruction and DG schemes
useless. The conclusion of \sec{decayturb} elaborates on this.

\item Turbulent driving for four turning times is too short to reach a
stationary state and get reliable statistics. (cf.
\fig{plots/stirturb/evolutions/sonic-mach.png}) This is especially true, when
comparing two totally different simulation frameworks. In fact, serious driven
turbulence studies drive more than 10 turning times (cf. \cite{konstandin2016mach}).

\item The driven turbulence simulations are substantially influenced by the
driver. This is, in particular, obvious for the results by the Midpoint
schemes. It renders our dissipation rates inconclusive. Apparently, turbulent
driving should be treated with caution as it can lead to unnatural
pseudo-turbulent flows (cf. \fig{plots/stirturb/snapshots/forcing-stop-02.png}).

\item A grid size of $512^3$ cells is still rather low in order to have a good
approximation of an isothermal turbulent flow. In the powerspectra we
identified the beginning of the dissipation range at $k_{diss} \approx 25$.
This means a small structure, say a shock, occupies $512/25 \approx 21$
cells\footnotemark in one dimension. That is 4\% of the box length. Since the
Reynolds number is, in theory, infinity (cf.  \sec{turbulence}), the turbulent
flow should go beyond $k_{diss}$.  Hence, the simulations are under-resolved.
\end{itemize}

The improvement in dissipation rates and resolution capacity, as we had hoped
for, was not achieved. To be honest, it would have been a surprise anyway. In
fact, a supersonic isothermal turbulent flow is one of the worst setups one can
choose to run higher-order Galerkin methods on. 

%A box full of swirling shocks is not a suitable habitat for polyonomes.

In essence, what we have shown is that discontinuous Galerkin methods can work
in strong shock environments. It makes sense, now, to look for astrophysical
problems where DG methods indeed have a chance to yield better results
without worrying that occasional shock conditions crash the simulation.

Protoplanetary disk simulations, for example, still have problems with
spurious angular momentum dissipation. Higher-order discontinuous
Galerkin methods, in combination with recent flux functions, might yield
promising results.

\footnotetext{Counting the cells under the peaks in
\fig{plots/sod/isothermal-supersonic/density} (isothermal Sod w. strong shock)
yields the same number, by the way.}

\newpage
