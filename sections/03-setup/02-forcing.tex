\subsection{Turbulent Forcing}
\label{sec:turbuforcing}

%% \begin{equation}
%%   d\widehat{\mathbf{f}}(\mathbf{k},t) = \frac{3}{\sqrt{1-2\zeta+3\zeta^2\,}}\;
%%         \left[-\hat{\mathbf{f}}(\mathbf{k},t)\frac{dt}{T_d} 
%%         + F_0 \; \sqrt{\frac{2\sigma^2(\mathbf{k})}{T_d}} \; \mathbf{P}(\mathbf{k})\cdot d\mathbf{W}_t\right],
%% \end{equation}


In equation \eqn{comp-euler-conservative}, a \emph{source} respectively
\emph{forcing} term was introduced into the Euler equations. At each time step
a varying force field $\vec{f}(t,x,y,z)$ perpetually injects kinetic energy at
largest scales (cf. \sec{turbulence}/\emph{Energy Cascade}). Based on
\cite{schmidt2009numerical}, we formulate an \vip{Ornstein-Uhlenbeck} process:
\begin{equation}
\label{eqn:ornstein-uhlenbeck}
  d\widehat{\vec{f}}(\vec{k},t) = F_0
        \left[-\hat{\vec{f}}(\vec{k},t)\,dt 
        + \big(\Theta(k_0 - |\vec{k}|) - \Theta(k_1 - |\vec{k}|)\big)\; \vec{P}(\zeta,\vec{k})\cdot d\vec{W}_t \right],
\end{equation}
where $\widehat{\vec{f}}$ is the Fourier transformed force field $\vec{f} =
(f_1,f_2,f_3)^T$, $\vec{k}$ the wave vector (cf.  \sec{theory-powerspectrum}),
$t$ is the physical time, $k_0,k_1 > 0$ and $F_0 > 0$ parameters, $\Theta(k)$ the
Heaviside step function and $d\vec{W}_t$ the \vip{Wiener} process. The
parameter $\zeta = [0,1]$ sets the ratio of \emph{compressive} (curl-free) to
\emph{solenoidal} (divergence-free) modes in the projection operator
$\vec{P}(\zeta,\vec{k})$
\begin{equation}
  (P_{ij})(\zeta,\vec{k}) = \zeta\,P_{ij}^\perp(\vec{k}) + (1-\zeta) P_{ij}^\parallel =\zeta\,\delta_{ij} + (1-2\,\zeta) \frac{k_i k_j}{k^2},\;\;\;\; i,j = \{1,2,3\}.
\end{equation}
For the sake of clarity, \eqn{ornstein-uhlenbeck} strongly simplifies the
actual implementation in FLEXI. The choice of runtime parameters in the driven
turbulence setup (cf. \sec{driven}) were set in such a way that the code
applies the force field as shown in \eqn{ornstein-uhlenbeck}.  Due to the
Heaviside step functions, a flat energy spectrum from $k_0$ to $k_1$ modes is
injected. 

Following parameters and their settings are of importance here:

\begin{description}
    \item [rmsv] Desired average \emph{root-mean-square-velocity} of the turbulence. When the
            specified threshold is reached, small but perpetual injections keep the turbulence
            in proximity of the \emph{rmsv} (cf. \sec{stirturb}). The driver modifies
            the force parameter $F_0$ via heuristics in order to avoid over- or undershoots.
    \item [kmin, kmax] Range of modes where to apply forcing (cf.
        \sec{turbulence}/\emph{Energy Cascade}). The range is set from $1$ to
        $3$. Stirring on only the first mode can be imagined as a force field with
        distinct features half the size of the box. Higher modes divide the box further
        down accordingly. Limiting forcing to only first three modes avoids imprinting
        a factitious powerspectrum on the system.
    \item [zeta] Parameter between 0 and 1 which sets the ratio of
            compressive to solenoidal modes 
            forcing. In this work, \emph{zeta} is set to $\zeta = 0.5$. That is,
            equal forcing on both driving mechanisms.
\end{description}

A depiction of the velocity field after energy injection for the first time on
an initially constant state is shown in
\fig{plots/forcing/rk3-hy-initial-forcing} as well as its associated velocity
powerspectrum (cf. \sec{turbulence}/\emph{Velocity Powerspectrum}) in
\fig{plots/forcing/powerspectrum}.  Obviously, the majority of kinetic energy
is crowded on the first three modes: $k = [1,3]$. Gradually, the energy moves
up to higher modes and creates the desired small scale structures desired for
driven turbulence simulations. An example of a fully developed turbulence
is shown in \fig{plots/forcing/b5-full-turbulence}.
\image{1.0}{plots/forcing/rk3-hy-initial-forcing}{Early-stage snapshot taken
immediately after first-time turbulent forcing on the first three modes. The
simulation was done in dimensionless units.}
\image{1.0}{plots/forcing/powerspectrum}{Shell-averaged velocity powerspectrum
(cf. \sec{theory-powerspectrum}) obtained from the velocity (or Mach number in
this case) field in \fig{plots/forcing/rk3-hy-initial-forcing}. The solvers
from FLEXI are shown.  Obviously, there is a small difference (consider log-log
scale!) between the FV-only and hybrid schemes. The distorted
spectra on larger modes might be a sign of already emerging oscillations.}
\image{1.0}{plots/forcing/b5-full-turbulence}{Fully developed Mach-10
turbulence snapshot, simulated with Bouchut5. Forcing was applied on the
first three modes and the picture was taken after four crossing times. The
formerly smooth large scale waves in \fig{plots/forcing/rk3-hy-initial-forcing}
compress and smash into each other, yielding tenuous filaments.}
