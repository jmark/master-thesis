\subsection{Turbulent Forcing}
\label{sec:turbuforcing}

In equation \eqn{comp-euler-conservative} a \emph{source} respectively
\emph{forcing} term was introduced into the Euler equations. At each time step
a varying force field $\vec{F}(t,x,y,z)$ perpetually injects kinetic energy at
largest scales (cf. \sec{turbulence}/\emph{Energy Cascade}). Based on
\cite{schmidt2009} we formulate
\begin{align}
  \hat{d\mathbf{f}}(\mathbf{k},t) &= \frac{3}{\sqrt{1-2\zeta+3\zeta^2\,}}\;\left[-\hat{\mathbf{f}}(\mathbf{k},t)\frac{dt}{T} + F_0 \left(\frac{2\sigma^2(\mathbf{k})}{T}\right)^{1/2}\mathbf{P}_\zeta(\mathbf{k})\cdot d\mathbf{W}_t\right]\\
  (P_{ij})(\mathbf{k}) &= \zeta\,P_{ij}^\perp(\mathbf{k}) + (1-\zeta) P_{ij}^\parallel =\zeta\,\delta_{ij} + (1-2\,\zeta) \frac{k_i k_j}{k^2}
\end{align}
The force field in Fourier space allows us to precisely specify at which
length scales we want to apply the forcing as well as the ratio of
\emph{compressive} and \emph{solenoidal} modes.

Following parameters are of importance:

\begin{description}
    \item [rmsv] Desired average \emph{root-mean-square-velocity} of the turbulence. When the
            specified threshold is reached, small but perpetual injections keep the turbulence
            in proxmity of the \emph{rmsv}.
    \item [kmin, kmax] Range of modes where to apply forcing (cf.
        \sec{turbulence}/\emph{Energy Cascade}). Usually, the range is set from $1$ to
        $3$. Stirring on only the first mode can be imagined as a force field with
        distinct features half the size of the box. Higher modes divide the box further
        down accordingly. Limiting forcing to only first three modes avoids imprinting
        a factitious powerspectrum on the system.
    \item [zeta] Parameter between 0 and 1 which sets the ratio of \emph{compressive} 
            to \emph{solenoidal} forcing. Many studies have shown a universality of 
            both stirring types ... In this work, \emph{zeta} is set to $0.5$.
\end{description}
A depiction of the velocity field after energy injection for the first time on
an initially constant state is shown in
\fig{plots/forcing/rk3-hy-initial-forcing} as well as its associated velocity
powerspectrum (cf. \sec{turbulence}/\emph{Velocity Powerspectrum}) in
\fig{plots/forcing/powerspectrum}.  Obviously, the majority of kinetic energy
is crowded on the first three modes: $k = [1,3]$. Gradually, the energy moves
up to higher modes and creates the desired small scale structures desired for
driven turbulence simulations. An example of a fully developed turbulence
is shown in \fig{plots/forcing/b5-full-turbulence}.
\image{1.0}{plots/forcing/rk3-hy-initial-forcing}{Early-stage snapshot taken
immediately after first-time turbulent forcing on the first three modes.}
\image{1.0}{plots/forcing/powerspectrum}{Shell-averaged three-dimensional
velocity powerspectrum obtained from the velocity (or Mach in this case) field
in \fig{plots/forcing/rk3-hy-initial-forcing}.}
\image{1.0}{plots/forcing/b5-full-turbulence}{Fully developed Mach-10
turbulence snapshot simulated with with Bouchut-5  and Girichidi's Stirred
Turbulence Module. Forcing was applied on the first three modes and the picture
was taken after four crossing times. The formerly smooth large scale waves
in \fig{plots/forcing/rk3-hy-initial-forcing} compress and smash into each
other yielding tenuous filaments.}
