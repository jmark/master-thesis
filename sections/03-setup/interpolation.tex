\subsection{Grid Spaces and Transformation}
%\label{sec:theory-interpolation}

In this work, four grid spaces are of importance: \emph{Face-centered grid
(FCG)}, \emph{body-centered grid (BCG)}, \emph{Gauss nodal grid (GNG)} and
\emph{Gauss-Lobatto nodal grid (LNG)}. A visual representation can be found in
figure \ref{fig-grid-spaces.png}. This figure shows a one-dimensional grid of
eight cells (dashed lines) or alternatively a grid with two elements (thick
lines) each consisting of four nodes which implies a polynomial order of three.
Major part of the work is the transformation back and forth between these
grid spaces via interpolation. Another significant aspect is the relationship
of elements to cells. First one overlays both grids. Ideally, they are of the
same shape and cover the same physical domain. When transforming between both
grid types, cells get grouped together in numbers equal to the nodal number of
the superincumbent element. Interpolation happens in each group/element
independently.

\image{0.6}{grid-spaces.png}{Nodes in a two element grid each consisting of
four cells. From bottom to top: face-centered, body-centered, Gauss nodes (n =
3), Gauss-Lobatto nodes (n = 3).}

\paragraph{Transformation from \textsc{FLASH} to \textsc{FLEXI}}

All following examples/figures in this document are produced by the very same
procedure where we set the interpolation order $n = 3$. First the intial values
are generated (or read) in the FLASH grid format. The grid is split into groups
of $4^3 = 64$ neighboring cells. These groups get mapped to the associated
element of the HOPR grid. Looping over all group-element pairs a
three-dimensional Lagrange Polynome gets constructed according to the
body-centered values of the cells. Then the new values at the
Gauss/Gauss-Lobatto nodes get interpolated and stored according to the ordering
convention of FLEXI.

In order to do useful analysis and visualization, the FLEXI data gets again
back-interpolated to BCG. This is similar to what the visualization routines in
FLEXI do.

\paragraph{Interpolation Error Estimate}

As a meassure of interpolation error of an original function $f$ and its
interpolated counterfeit $\widetilde{f}$ the relative root-means-sqare variance
represents one distance meassure.
\begin{equation}
    f_{\text{RMS}} = \sqrt{\frac{1}{N} \sum^N_i f^2_i} 
\end{equation}

\begin{equation}
    \errorrms = (f - \widetilde{f})_{\text{RMS}} = \sqrt{\frac{1}{N} \sum^N_i (f_i - \widetilde{f_i})^2} 
\end{equation}

\begin{equation}
    \errorrmsrel = \frac{\errorrms}{f_{\text{RMS}}} 
\end{equation}

One has to ensure that $f_i$ and $\widetilde{f_i}$ live in the same grid space.

Another error estimate could be the absolut mean error:
\begin{equation}
    f_{\text{ABS}} = \frac{1}{N}\sum^N_i |f_i| 
\end{equation}

\begin{equation}
    \errorabs = (f - \widetilde{f})_{\text{ABS}} = \frac{1}{N}\sum^N_i |f_i - \widetilde{f_i}| 
\end{equation}

\begin{equation}
    \errorabsrel = \frac{\errorrms}{f_{\text{ABS}}} 
\end{equation}
In most cases the total absolute error yields smaller results, since runaways
get weighted more in the rmse case. A proper handling of runaways especially in
interpolation of shocks is still an open question.

For analysis and defining initial conditions with DG based code interpolation
techniques for translating correct state values between different grid spaces
are necessary. See \sec{theory-interpolation} for theoretical disuccusion.
In this section we want to present observations and draw important conclusions
on how to correctly interpret the simulation data. Furthermore, the correctness
of the interpolation code is shown and hence its produced results can be trusted.

When all data points represent a smooth pathway and are sufficiently close
together an interpolating polyonme can resemble the original function very
precisely. \Fig{2d-third-order.png} gives an example of this well-behaving
case.
\image{1.0}{plots/interpolation/2d-third-order.png}{
Two-dimensional third-order interpolation of a second-order function.
relative rms error = 0.000000\%
}
However, since we want to work with hyper-sonic shocks, discontinuities are
present throughout the data. Spurious oscillations and severely erranous
approximations are the consequence. An artificial stress test
see \fig{2d-third-order-2.png}, reveals the strength of the DG method. Since,
the interpolation is always limited on small patches of the whole domain
the global features still get represented at the cost of considerable deviations
at smaller scales.
\image{1.0}{plots/interpolation/2d-third-order-2.png}{
Stress test: 3D superposed sines and cosines with random noise: Third-order 3d interpolation of
finite volume data to DG data (Gauss-Lobatto nodes).
relative rms error = 14.987457\%
}
When transferring real-world data from \FLASH (finite-volume) to \FLEXI (DG) we get
following results.




\image{1.0}{plots/interpolation/density-ray.png}{Foobar fo bar}
