\subsection{Polynomial Interpolation}

For analysis and defining initial conditions with DG based code interpolation
techniques for translating correct state values between different grid spaces
are necessary. See \sec{theory-interpolation} for theoretical disuccusion.
In this section we want to present observations and draw important conclusions
on how to correctly interpret the simulation data. Furthermore, the correctness
of the interpolation code is shown and hence its produced results can be trusted.

When all data points represent a smooth pathway and are sufficiently close
together an interpolating polyonme can resemble the original function very
precisely. \Fig{2d-third-order.png} gives an example of this well-behaving
case.
\image{1.0}{plots/interpolation/2d-third-order.png}{
Two-dimensional third-order interpolation of a second-order function.
relative rms error = 0.000000\%
}
However, since we want to work with hyper-sonic shocks, discontinuities are
present throughout the data. Spurious oscillations and severely erranous
approximations are the consequence. An artificial stress test
see \fig{2d-third-order-2.png}, reveals the strength of the DG method. Since,
the interpolation is always limited on small patches of the whole domain
the global features still get represented at the cost of considerable deviations
at smaller scales.
\image{1.0}{plots/interpolation/2d-third-order-2.png}{
Stress test: 3D superposed sines and cosines with random noise: Third-order 3d interpolation of
finite volume data to DG data (Gauss-Lobatto nodes).
relative rms error = 14.987457\%
}
When transferring real-world data from \FLASH (finite-volume) to \FLEXI (DG) we get
following results.




\image{1.0}{plots/interpolation/density-ray.png}{Foobar fo bar}
