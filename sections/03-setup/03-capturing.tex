\subsection{Shock Capturing}
\label{sec:shock-capturing}

In hypersonic simulations the solver has to deal with strong shocks in an
accurate and robust manner. We utilize shock capturing strategies in order to
alleviate the destabilizing impact of discontinuities on DG schemes.

This endeavor, however, is far from trivial because of two main reasons.  The
first is that the exact solution of (nonlinear) purely convective problems
develops discontinuities in finite time; the second is that these solutions
might display a very rich and complicated structure near such discontinuities.
Thus, when constructing numerical methods for these problems, it must be
guaranteed that the discontinuities of the approximate solution are the
physically relevant ones. Also, it must be ensured that the appearance of a
discontinuity in the approximate solution does not induce spurious
oscillations, see \fig{plots/burgers/ringing.png}, which spoil the quality of
the approximation; on the other hand, while ensuring this, the method must
remain sufficiently accurate near shocks in order to capture the possibly rich
structure of the exact solution (cf.  \sec{theory-shocks}).

Within the scope of this work three intertwining concepts are developed.  First
we must detect (\emph{sensoring}) a discontinuity and dampen (\emph{artificial
viscosity}) the appearing oscillations. In case this is not sufficient the
element switches to FV mode (\emph{switching}) and endures the troubling
phase till it can safely reverts back to DG mode.

\paragraph{Sensoring}
Based on the \textsc{Persson} indicator we develop a \emph{smoothness} operator
which meassures the variance of the highest frequencies in modal space of the
Galerkin polynome. In other words we build an oscillation detector.  First we
express the solution of order $N_p$ within each eleement in terms of an
orthogonal basis as
\begin{equation}
    u = \sum^{N_p}_{i=1} u_i \psi_i,
\end{equation}
where $N_p$ is the total number of terms in the expansion and $\psi_i$ are the
\textsc{Legendre} basis functions. See \fig{expansion-modes.png}.
\image{0.6}{expansion-modes.png}{First four expansion modes for the Lagrange
and the Legendre basis. They are related by a coordinate transformation from
\emph{nodal} to \emph{modal} space. The presence of higher modes in modal space
signifies a strongly oscillating interpolant in nodal space.}

Now we only consider the terms up to order $N_p-1$, that is
\begin{equation}
    u_{N_p-1} = \sum^{N_p-1}_{i=1} u_i \psi_i,
\end{equation}
Whithin each element $\Omega$ we define the following \emph{smoothness} sensor
\begin{equation}
\label{eqn:smoothness-indicator}
    s = \log_{10} \frac{\langle u - u_{N_p-1}, u - u_{N_p-1} \rangle}{\langle u, u \rangle},
\end{equation}
where $\langle \cdot,\cdot \rangle$ is the standard inner product in $L_2(\Omega)$.
The smaller the indicator $s$, the smoother is the approximated solution.
The state variable $u$ is mostly the pressure $\pres$, which gives the best
results, because pressure jumps are generally very sharp in shocks.

\paragraph{Artificial Viscosity}
Artificial viscosity $0 < \epsilon \lll 1$ is introduced to the Euler equations
by a very weakly interacting diffusion term.
\begin{equation}
\label{eqn:persson-indicator-gov-equ}
    \frac{\partial\vec{U}}{\partial t} + \nabla \cdot \mat{F}(\vec{U}) = \nabla \cdot (\epsilon \nabla \mat{\tau}),
\end{equation}
with $\mat{\tau}$ being the deviatoric stress tensor. The amount of viscosity
varies for each element and depends on the current shock strength.
We have to consider two cases.

If the element is in FV mode, $\epsilon$ is set quadratic proportional to the
maximum root-mean-square velocity $\vels_{rms}$ of the element.
\begin{equation}
    \epsilon_{FV} = \epsilon_{FV,0} \cdot \max(\vels_{rms})^2
\end{equation}
with $\epsilon_{FV,0}$ being an arbritrary, but sufficiently small constant.
This meassure should only have an effect on extreme velocity spikes otherwise
the governing equations would get viscous and violate the timestep constraints.

In case of an element in DG mode, the amount of $\epsilon$ is based on the
\emph{Persson indicator} introduced above.
\begin{equation}
    \epsilon_{DG} = \begin{cases}
        0 & \text{if}\;\;\; s < s_0 \\
        \epsilon_{DG,0} & \text{if}\;\;\; s > s_0 \\
        \frac{\epsilon_0}{2} \left ( 1 + \sin\frac{\pi(s-s_0)}{2\kappa} \right ) & \text{else}
    \end{cases}
\end{equation}
The parameters $\epsilon_{DG,0}$, $s_0$ and $\kappa$ are chosen empirically,
but again, $\epsilon_{DG,0}$ must be sufficiently small.

In this thesis, the viscosity constants are set very tiny on purpose:
$\epsilon_{DG,0} = \epsilon_{DG,0} = 10^{-10}$. The solely task of this module
is to catch extremely high velocity spikes, which happens from time to time,
and diffuse them into neighboring elements. Besides that, there is no measurable
effect on the simulation.

\paragraph{FV-DG Mode Switching}
By setting a specific threshold for the \emph{smoothness indicator} $s$, one can
decide when to switch back and forth between DG and FV mode. This procedure is
done at every timestep, hence the elements in FV mode should follow along the
shock waves throughout the domain.  We propose two modifications of the Persson
indicator, perform a comparing study and clarify the advantages over the other.
Based on \eqn{smoothness-indicator} the first variation, called \emph{indicator
A}, reads
\begin{equation}
\label{eqn:indicator-A}
    s_A = \log_{10} \max \left(
        \frac{\fnorm{u^2 - u_{N_p-1}^2}^2}{\fnorm{u^2}^2},
        \frac{\fnorm{u_{N_p-1}^2 - u_{N_p-2}^2}^2}{\fnorm{u^2 - u_{N_p-1}^2}^2}
    \right),
\end{equation}
where $\fnorm{\cdot} = \sqrt{\avg{\cdot,\cdot}}$.  The second version, called
\emph{indicator B}, reads
\begin{equation}
    s_B = \log_{10} \max_M \frac{\fnorm{u - u_{N_p-M}}^2}{\fnorm{u_{N_p-M}}^2}
\end{equation}
\Fig{indicator-slice.png} shows a fully developed turbulence with shocks and
their associated heat maps of indicator values calculated from the pressure.  A
normalized distribution (PDF) of indicator values $s$ is plotted in
\fig{indicator-pdf.png}.  The most prominent difference is the higher
sensitivity of indicator A and its limitation to only negative values. The
latter is a result of the normalizing fractions in \eqn{indicator-A}. Both
variants trail the shock fronts with sufficient accuracy. Considering the PDF
in \fig{indicator-pdf.png}, one can observe spikes in both curves in the
interval between -1 and 0. They herald the realm of strong shocks. The broad
hill around -2.5 contains the majority of stressed elements which were affected
by a recently pervading shock front.  The little spike at -4.5 stems from a
bias introduced by the indicator A. Numerous tests revealed an empirical
threshold of $s_0$ = -4.5 where the switching takes place. Clearly, indicator A
unambiguously signalizes elements with unresolvable discontinuities. This opens
the possibility to further strengthen the DG scheme for a broader coverage of
the domain above $s_0 = -4.5$.
%\image{0.8}{indicator-column.png}{Snapshot of hyper-sonic turbulence.}
\image{0.8}{indicator-slice.png}{Pressure and Mach Number slices of a Mach-10
driven turbulence similar to \fig{plots/forcing/b5-full-turbulence}. Below are
the \emph{smoothness} indicators $s_A$ (\eqn{indicator-A} and $s_B$
(\eqn{indicator-B} derived from the pressure.}
\image{0.9}{indicator-pdf.png}{Probability Distribution of \emph{smoothness}
indicators $s_A$ (\eqn{indicator-A} and $s_B$ (\eqn{indicator-B} in a Mach-10
driven turbulence shown in \fig{indicator-slice.png}. Due to the high spike
of indicator A, it is well-suited for distinguishing strong shocks in the
pressure field. A more detailed explaination is provided in the text.}
