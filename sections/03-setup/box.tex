\subsection{Periodic Box and Initial Conditions}

The turbulence simulations live within a three-dimensional periodic box of
equal length in all dimensions. It means the state is a closed system where
mass, momentum and energy are conserved. Via active forcing kinetic energy
enters the medium on large scales and leaves it as internal energy via active
cooling on small scales. This way a stationary flow of momentum gets pumped
through the system. In conjunction with \emph{bulk motion correction}
hyper-sonic turbulence emerges which presents a stress test for every numerical
scheme.

include some fancy schmancy 3d picture 

Based on considerations in \sec{governing-equations} we initialize the state
as listed in \tbl{initial-state}. 
\begin{table}[H]
%\fontsize{3mm}{3mm}\selectfont
%\captionsetup{width=.6\textwidth}
\caption{Overview of initial values set for all turbulence simulations.}
\centering
\begin{tabular}{llc|llc}
\toprule
\multicolumn{3}{c}{\FLASH} &
\multicolumn{3}{c}{\FLEXI} \\
\midrule
Name & Symbol & Value & Name & Symbol & Value\\
\midrule
density         & $\rho_0$            & 1.0 & density         & $\rho_0$            & 1.0 \\ 
x-velocity      & $\vels_{x,0}$       & 0.0 & x-momentum      & $p_{x,0}$           & 0.0 \\ 
y-velocity      & $\vels_{y,0}$       & 0.0 & y-momentum      & $p_{y,0}$           & 0.0 \\ 
z-velocity      & $\vels_{z,0}$       & 0.0 & z-momentym      & $p_{z,0}$           & 0.0 \\  
pressure        & $\pres_{x,0}$       & 0.6 & total energy    & $\ener_{x,0}$       & 0.9 \\  
\bottomrule
\end{tabular}
\label{tab:initial-state}
\end{table}\remark Some numerical schemes actually solve the magneto-hydrodynamics
equations. By setting the initial magnetic density flux field $\magdensv_0 = 0$,
the MHD equations resemble the compressible Euler equations.
