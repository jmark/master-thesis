\subsection{Turbulent Forcing}

The theoretical framework has been given in section \ref{Stirred Turbulence}. The turbulent
stirring module \emph{StirGirichids} by P. Girichids was ported to \FLEXI and tested. The
module can be configured by a multitude of parameters at runtime. What follows is an overview
of the most important one.
\begin{description}
    \item [rmsv] Desired average \emph{root-mean-square-velocity} of the turbulence. When the
            specified threshold is reached small but perpetual injections keep the turbulence
            in proxmity of the \emph{rmsv}.
    \item [kmin, kmax] Range of modes where to apply forcing. Commonly, the range
            is set between $1$ and $7$. Stirring on only the first mode can be translated to a
            force field with distinct features half-size the box. Higher modes divide the box further
            down accordingly. All simulations presented here are limited to only the
            first mode since one wants to avoid imprinting an artificial powerspectrum
    \item [zeta] Parameter between 0 and 1 which sets the ratio of \emph{compressive} and \emph{solenoidal}
            forcing. Many studies have shown a universality of both stirring types ...
            In this work \emph{zeta} is set to $0.5$.
\end{description}
A depiction of the velocity field after energy injection for the first time on
a constant state is is shown in \fig{example-forcing.png} and its associated
three-dimensional powerspectrum in \fig{example-forcing-powerspectrum.png}. In
both figures the nearly equal distribution of kinetic energy on the first seven modes
is very clear.

\image{0.8}{example-forcing.png}{Velocity immediately after first-time
turbulent forcing on the first seven
modes.}
\image{1.0}{example-forcing-powerspectrum.png}{Shell-averaged three-dimensional velocity
powerspectrum obtained from the velocity field shown in \fig{example-forcing.png}.}
