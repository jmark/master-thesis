\subsection{CFD Frameworks}

For the turbulence simulations two numerical frameworks for computational fluid
dynamics are compared.

\paragraph{\FLASH}
Cite from the homepage.

The FLASH code, currently in its 4th version, is a publicly available high
performance application code which has evolved into a modular, extensible
software system from a collection of unconnected legacy codes. FLASH consists
of inter-operable modules that can be combined to generate different
applications. The FLASH architecture allows arbitrarily many alternative
implementations of its components to co-exist and interchange with each other.
A simple and elegant mechanism exists for customization of code functionality
without the need to modify the core implementation of the source. A built-in
unit test framework combined with regression tests that run nightly on multiple
platforms verify the code. 

This framework is very established within the astrophysics community and
represents the trusted basis for testing and comparing the modern higher-order
schemes provided by \FLEXI.

\FLASH already provides all modules necessary for the turbulence simulations
done as part of the thesis. A complete set of source files and setup calls
are provided in the digital appendix.

The demands for the grid/mesh module are minimal. Setting a uniform grid with
periodic boundary conditions is sufficient. Like most grid implementations the
domain is divided into blocks matching the number of computing units.
Unfortunately, the uniform grid module of \FLASH is quite unflexible with
regard to possible block configurations since they are tightly coupled with the
overall grid resolution or block size respectively.

There is broad range of hydrodynamic and magnet-hydrodynamic solvers available
but we will focus on three split schemes which were intentionally constructed
for supersonic turbulence simulations: PPM, Bouchut3 and Bouchut5. They
were already discussed briefly in \sec{flux-functions}.

The modules for polytropic cooling and turbulent forcing besides their main
purpose addiontally calculate interesting simulation data like total energy,
dissipated kinetic energy, etc.  and write them into plain-text data files.

For performance profiling a current timestamp gets written down at every
timestep. This very simple yet naive approach has the disadvantage that
input/output operations, especially saving large snapshots, contribute to the
overall account as well. Ephemeral bottle necks in network transfer rates of
the HPC infrastructure unfairly increase the total wallclock time.  This issue
is further discussed in the conclusion.

\paragraph{\FLEXI}
Cite from the homepage.

Flexi is developed by the team of the Numerics Research Group hosted at the
Institute of Aero- and Gasdynamics at the University of Stuttgart. We are
interested in efficient and robust numerical methods for applications in scale
resolving CFD and we apply these methods to a variety of of large scale
physical and industrial problems.

This modern framework did not originally provide or just rudimentarily offered
any facilities for polytropic cooling, turbulent forcing, bulk motion
correction and shock capturing.

Consequently, missing modules were ported from \FLASH and adapted accordingly
to fit into \FLEXI's infrastructure.  With regard to configuration and desired
effect on the physics simulation these additions were implemented as closely to
\FLASH as possible.
