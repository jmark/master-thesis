\subsection{Interpolation \& Data Transfer}
\label{sec:trafo}

\paragraph{Grid Spaces}
The term \emph{grid space} refers to the arrangement of nodes within an
element. Four types of node configurations are of importance here:
\emph{Face-centered grid} (FCG), \emph{body-centered grid} (BCG), \emph{Gauss
nodal grid} (GNG) and \emph{Gauss-Lobatto nodal grid} (LNG). Finite Volumes
live on a BCG while the Galerkin polyonomes are anchored on either GNG or LNG.
Gauss and Gauss-Lobatto nodes are the roots of special polynomials.  See
\fig{grid-spaces.png}. They minimize the errors of the \emph{Gauss quadrature}
(numerical integration) which is one of the corner stones in the development
Galerkin methods (cf. \sec{polyonome-ansatz}). Cf. 
\cite{gassner2011comparison} for a comparative study on the differences between
GNG and LNG.

\image{0.5}{grid-spaces.png}{Two finite elements containing
four cells/volumes. From top to bottom: Gauss-Lobatto nodes ($N_p$ =
3), Gauss nodes ($N_p$ = 3), body-centered and face-centered nodes. DG schemes
are defined on Gauss and Gauss-Lobatto nodes while FV methods anchor their
data at the center of the volumes/cells (body-centered). The exchange of
information (flux functions: cf. \sec{flux-functions}) happens at the
cell boundaries (face-centered).}

We only consider Gauss nodes, since the FV-DG mode switching (cf.
\sec{shock-capturing}) does only support Gauss nodes at the time of
writing.

In order to compare and share data between the FV and DG schemes one has to
translate between the different grids via Lagrange interpolation.  Although
FLEXI offers curvilinear grids we only need the Euclidean geometry. Thus whe
have a one-to-one relationship between the four grid spaces and are able to
directly lay them on top of each other as depicted in \Fig{grid-spaces.png}.

\paragraph{Interpolation Error} is in general inevitable but manageable.
See \fig{lagrange-interpolation-1d.png} for an interpolation of a trigonometric
function. The quality of an interpolation procedure is determined by an
error measure which we motivate in the following.

The Lagrange interpolating polyonome $L(x)$ of order $N$ is exact for a
polynomial function $f(x)$ of order $M \le N$.

\emph{Proof:} The remainder $R(x) = L(x) - f(x)$ can be expressed as
\begin{equation}
    R(x) = L(x) \; f^{(N+1)}|_{\zeta}, \;\;\;\; x_0 \le \zeta \le x_N,
\end{equation}
using Taylor series analysis (cf. \cite[p. 878]{abramowitz1964handbook}).
$x_0,...,x_N$ are the interpolation nodes and $f^{(N+1)}(x)$ is the (N+1)-th
derivative of $f(x)$.  Since $M \le N$, $f^{(N+1)}(x) = 0 \; \Rightarrow R(x) =
0,\; \forall x$. $\square$

An example for an exact Lagrange interpolation in two dimensions is shown
\Fig{plots/interpolation/2d-third-order}.
\image{1.0}{plots/interpolation/2d-third-order}{
Example of a two-dimensional exakt third-order Lagrange interpolation of a
second-order polynomial. The relative error $\errorrms$ = 0.000000\% (cf.
\eqn{error-rms}) is below machine precision.}

Instead of the remainder $R(x)$ we define a \emph{root-means-sqare
variance} as a kind of distance meassure between an arbitrary function $f(x)$ and
its interpolant $\widetilde{f}(x)$.
\begin{equation}
\label{eqn:error-rms}
    \errorrms = (f - \widetilde{f})_{\text{RMS}} = \sqrt{\frac{1}{N} \sum^N_i (f_i - \widetilde{f_i})^2}.
\end{equation}
The relative error is calculated by
\begin{equation}
    \errorrmsrel = \frac{\errorrms}{f_{\text{RMS}}},
\end{equation}
where $f_{\text{RMS}} = \sqrt{\frac{1}{N} \sum^N_i f^2_i}$. 

\remark One has to ensure that $f_i$ and $\widetilde{f_i}$ live in the same grid space.

% Another error estimate could be the absolut mean error:
% \begin{equation}
%     f_{\text{ABS}} = \frac{1}{N}\sum^N_i |f_i| 
% \end{equation}
% 
% \begin{equation}
%     \errorabs = (f - \widetilde{f})_{\text{ABS}} = \frac{1}{N}\sum^N_i |f_i - \widetilde{f_i}| 
% \end{equation}
% 
% \begin{equation}
%     \errorabsrel = \frac{\errorrms}{f_{\text{ABS}}} 
% \end{equation}
% In most cases the total absolute error yields smaller results, since runaways
% get weighted more in the rmse case. A proper handling of runaways especially in
% interpolation of shocks is still an open question.

% For analysis and defining initial conditions with DG based code interpolation
% techniques for translating correct state values between different grid spaces
% are necessary. See \sec{theory-interpolation} for theoretical disuccusion.

\paragraph{Transferring Data from \textsc{FLASH} to \textsc{FLEXI}}
For the decaying turbulence simulations (cf. \sec{decayturb}) a transfer of
initial states from FLASH to FLEXI is necessary. A brief overview of the
procedure and some results are discussed.

In order to give a sense of intuition, we imagine a Cartesian grid with 512
body-centered points in each of the three dimensions. Each data point
represents a finite-volume, read in from a FLASH checkpoint file. We subdivide
the contiguous block of $512^3$ into elements of $4^3 = 64$ nodes.  Within each
element the data gets transformed via three-dimensional third-order Lagrange
interpolation into Gauss or Gauss-Lobatta nodal space. Finally, the new data is
written to a FLEXI checkpoint file. The reverse procedure is available as well,
which is useful for analysis and visualization. 

% \remark Before the interpolation step the data has to translated from primitive
% to conservative variables (cf. \sec{sec:computational-frameworks}/FLEXI and 
% \sec{sec:governing-equation}/Primitive and Conservative Variables).

\Fig{plots/interpolation/flexi-column-density.png} and
\fig{plots/interpolation/density-ray.png} show practical results of a data
transfer from FLASH to FLEXI. They confirm the applicability of the transfer
procedure within acceptable error margins. Unavoidable is the truncation of
sharp spikes in the data. Evidently, critical information in the proxmity of
strong shocks gets lost. Considering this, it is advisable to apply a
Gaussian blur before interpolation step. This way, almost equal initial
states in FLASH and FLEXI are ensured.

\image{1.0}{plots/interpolation/flexi-column-density.png}{
Column densities plots of interpolated hypersonic (Mach 10) turbulence (see
\fig{plots/forcing/b5-full-turbulence}) from FLASH to FLEXI. Left side: The
density is interpolated via a third-order (four nodes per element) Lagrange
interpolation as described in the text. The relative interpolation error is
estimated as $\errorrms = 0.118763 \approx 12$\% (cf. \eqn{error-rms}).  Right
side: The interpolation can be improved by introducing a Gaussian blur before
the interpolation step: $\errorrms = 0.2808160 \approx 3$\%.
}
\image{0.8}{plots/interpolation/density-ray.png}{
Comparison of one-dimensional density profiles and residuals through a
hypersonic (Mach 10) turbulent box with original simulation data (flash) and
transferred interpolated data (flexi). The density ray is extracted from same
data already shown in \fig{plots/forcing/b5-full-turbulence} and
\fig{plots/interpolation/flexi-column-density.png}. The smoothing (before
the interpolation) was done via a Gaussian blur which minimizes the residual
considerably.}
