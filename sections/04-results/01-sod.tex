\subsection{\vip{Sod} Shock Tube Problem}
\label{sec:sod-shock-tube}
Before we can discuss the results of the turbulence simulations, we
need to validate the correctness of the solvers, considering the
modifications explained in the previous chapter. The standard test for the
correct treatment of shocks is the \vip{Sod} Shock Tube Problem in one
dimension. At first, we look at the classical adiabatic test case, then
introduce polytropic cooling (isothermal) and finally apply a strong shock
sitation by setting the initial left-side velocity to $u_L = 15$. All runs took
place with a conservative CFL number of $CFL = 0.4$.
If not indicated otherwise the initial conditions were set like follows
\begin{table}[H]
%\fontsize{3mm}{3mm}\selectfont
%\captionsetup{width=.6\textwidth}
\caption{Sod-Shock: Initial Condition}
\centering
\begin{tabular}{llc|lc}
\toprule
&
\multicolumn{2}{c}{Left Side} &
\multicolumn{2}{c}{Right Side} \\
\midrule
Name & Symbol & Value & Symbol & Value\\
\midrule
density         & $\dens_{L}$       & 1.0 & $\dens_{R}$ & 0.125 \\ 
velocity        & $\vels_{L}$       & 0.0 & $\vels_{R}$ & 0.0 \\ 
pressure        & $\pres_{L}$       & 1.0 & $\pres_{R}$ & 0.1 \\  
\bottomrule
\end{tabular}
\label{tab:initial-state-sod-shock}
\end{table}
We set a resolution of 512 cells for the domain $\Omega = [0,1]$, where the
discontinuity is at $x_0 = 0.5$.  512 cells resembles the same resolution
applied on the turbulence simulations. This way we get a comparable insight on
how well shocks can be resolved within a periodic box of equal size.
Table \tbl{sod-runs} outlines conducted experiments in tabular form.
\begin{table}[H]
%\fontsize{4mm}{4mm}\selectfont
%\captionsetup{width=.8\textwidth}
\caption{Overview of conducted runs for the Sod Shock Tube Problem}
\centering
\begin{tabular}{l|l}
\toprule
Setups & Adiabatic, Isothermal, Adiabatic w. Strong Shock,
Isothermal w. Strong Shock \\ 
\midrule
Solvers  & Bouchut 3, Bouchut 5, PPM, Euler FV, Euler Hybrid\footnotemark,\\
& Midp. FV, Midp. Hybrid, RK3 FV, RK3 Hybrid \\ 
\midrule
Resolution & 512 \\
\midrule
CFL    &  0.4 \\
\bottomrule
\end{tabular}
\label{tab:sod-runs}
\end{table}

This setup is not a convergence study or similar. The objective is to the test
wether FLEXI produces the same results as FLASH under identical initial
conditions.

\footnotetext{Euler Hybrid was only run for adiabatic and isothermal Sod Shock
since it is unstable.}

\subsubsection{Adiabatic \& Isothermal Shock}
In \fig{plots/sod/adiabatic/density.png} all utilized solvers satisfy the
expected density profile of the classical Sod-Shock problem with varying
precision. PPM yields the most accurate result while on the other hand
RK3-Hybrid smears out the discontinuities considerably. Looking closely at the
zoomed area all hybrid schemes (blue, yellow and green dashed lines) tend to
oscillate.  The inacceptable ringing of the Euler-Hybrid is a consequence of
the instability of the Euler time integration within the DG operator.
Consequently, Euler-Hybrid is disqualified and is discarded.  Same assertions
can be made for the isothermal case shown in
\fig{plots/sod/isothermal/density.png}.
\image{1.0}{plots/sod/adiabatic/density.png}{
Canonical density profile at the conventional time $t$ = 0.2 with anticipated
regions and discontinuities outlined in the plot. The formular of the analytical
analytical solution is written down in \cite{sod1978survey}.}
\image{1.0}{plots/sod/isothermal/density.png}{
Isothermal density profile at conventional time $t$ = 0.2 with expected
regions and discontinuities outlined in the plot. The contact discontinuity
gets suppressed due to polytropic cooling.}
%\image{1.0}{plots/sod/adiabatic/pressure.png}{Canonical pressure profile at conventional time $t$ = 0.2.}
%\image{1.0}{plots/sod/adiabatic/velocity.png}{Canonical velocity profile at conventional time $t$ = 0.2.}
%The pressure and density profiles \fig{plots/sod/adiabatic/density.png}
%and \fig{plots/sod/adiabatic/velocity.png} are shown here for completness.
%\image{1.0}{plots/sod/isothermal/velocity.png}{Foobar fo bar}

\subsubsection{Adiabatic \& Isothermal Sod with Strong Shock}
We compare how the solvers cope with a strong shock situation ($u_L = 15$)
first in the adiabatic and then in isothermal setting with polytropic cooling.

Considering \fig{plots/sod/adiabatic-supersonic/density} all solutions yield a
quite similar profile. The shock waves build up rapidly up to a certain
height, travel at a constant speed from left to right and widen with increasing
time. One can recognize two distinct parts. A fast traveling pedestal moving at
the speed of the initial velocity $u_{fast} = u_L = 15$ and a slower moving
peak at the speed of around two third of the fast wave $u_{slow} \approx
2/3\,u_{fast} = 10$. 
The solutions for the strong shock setup with cooling, see
\fig{plots/sod/isothermal-supersonic/density}, are quite different. Most
notably, the density peaks are of an order higher and the widths are
considerably narrower than before. There are no recognizable partitions into a
slow and fast moving wave. Additionally, the solvers do not agree on the shock
wave speed, with the Euler FV  being the slowest but highest rising one. We do
not have a definite explanation for this, but it should be safe to say that the
mean free path of a shock wave in a turbulence simulation does not exceed half
the size of the box. Hence, the influence of the velocity dispersion among the
solvers is considered neglectible.
%Taking a closer look at one of the shock waves, for example
%in \fig{plots/sod/adiabatic-supersonic/density-zoom}, one gets to the same conclusion
%as in the previous section with PPM yielding the best results.
\image{1.0}{plots/sod/adiabatic-supersonic/density}{
Adiabatic density snapshots depicting the movement of a strong shock wave from
left to right. Note: Only the right half of the domain is shown. The pedestal
of the rightmost wave ($t_d$ = 0.04) has gone out of visible range.}
%\image{1.0}{plots/sod/adiabatic-supersonic/density-zoom}{Zoom at the rightmost
%shock wave in \fig{plots/sod/adiabatic-supersonic/density.png}. The grid
%spacing of the x axis resemble the width of a finite volume.}
\image{1.0}{plots/sod/isothermal-supersonic/density}{Isothermal density
snapshots depicting the movement of a strong shock wave from left to right.
Note: Only the right half of the domain is shown.}


%% It probably makes no sense commenting about the details of the curve shapes,
%% see \fig{plots/sod/isothermal-supersonic/velocity-zoom}, since at least around
%% 16 cells (four elements of four nodes) are trying to resolve the shock wave.
%% This is an unsufficient resolution compared to the adiabatic case where a shock
%% wave is four to five times broader. Consequently, isothermal turbulence
%% simulations are severely under-resolved.
%\image{1.0}{plots/sod/isothermal-supersonic/velocity-zoom}{Zoom at the
%rightmost shock wave in \fig{plots/sod/isothermal-supersonic/density.png}. The
%grid spacing of the x axis resemble the width of a finite volume.}
%\image{1.0}{plots/sod/adiabatic-supersonic/pressure.png}{Foobar fo bar}
%\image{1.0}{plots/sod/adiabatic-supersonic/velocity.png}{Foobar fo bar}
%\image{1.0}{plots/sod/isothermal-supersonic/velocity.png}{Foobar fo bar}

\subsubsection{Summary}
We conducted four different Sod shock tube tests in order to validate the
correct treatment of shocks by the FLEXI code. The Euler Hybrid scheme rendered
invalid since it showed spurious oscillations. Special attention should be
given to the Euler FV since it drifted away somewhat from the other solvers
for the supersonic strong shock test (cf.
\fig{plots/sod/isothermal-supersonic/density}).
