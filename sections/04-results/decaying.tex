\subsection{Decaying Turbulence}
\subsubsection{Time Evolution}
\image{1.0}{plots/decayturb/evol-sonic-mach}{Time evolution of the
root-mean-square mach number. Additionally, the turbulent sonic mach number
evaluated from the width of the density PDF is shown for comparison. A detailed
discussion is provided in \sec{decayturb-PDF}.}

\image{1.0}{plots/decayturb/evol-timestep}{Time evolution of the timestep $dt$.
The CFL Numbers are listed in \tbl{tab:decayturb-cfl}.}
\begin{table}[H]
\fontsize{3mm}{3mm}\selectfont
\captionsetup{width=.8\textwidth}
\caption{Overview CFL Numbers}
\centering
\begin{tabular}{l| cccc cccc}
\toprule
Solver & \bouT & \bouF & \fppm & \eufv & \mpfv & \mphy & \rkfv & \rkhy \\
\midrule
CFL   & 0.8 & 0.8 & 0.1 & 0.4 & 0.8 & 0.9 & 0.9 & 1.2 \\
\bottomrule
\end{tabular}
\label{tab:decayturb-cfl}
\end{table}

\image{1.0}{plots/decayturb/evol-ener}{Time evolution of the energy in the
system. \remark The y-axis is in logarithmic scale. The total energy $\ener$ is
the sum of the kinetic $\ekin$ and internal energy $\eint$: $\ld(\ener) = \ld(\eint + \ekin)$.}

\image{1.0}{plots/decayturb/evol-eint}{Time evolution of the internal energy
$\eint$ under polytropic cooling. The quantifiable deviation of the internal
energy from the anticipated constant $\eint_{0} = 0.9$ is discussed in the
text.}

\image{1.0}{plots/decayturb/evol-ekin-diss}{Time evolution of the
kinetic energy dissipation $\ekindiss$.}

\image{1.0}{plots/decayturb/evol-enstrophy}{Time evolution of the
mean enstrophy $\enst$.}

\image{1.0}{plots/decayturb/evol-fv}{Time evolution of the fraction of
Finite-Volume Elements to the total number of elements. \remark The other
solvers would stay at 100\% since they operate solely with Finite-Volumes.}

\subsubsection{Powerspectra}
\image{1.0}{plots/decayturb/pws-ekin-dyntimes}{Powerspectra of the
mean kinetic energy field shown at three different stages of the
decaying turbulence.}

\image{1.0}{plots/decayturb/evol-ekin-small-scale}{Amount of small-scale
energies prevailing in the dissipative length scales over time.}

\image{1.0}{plots/decayturb/evol-slope-vw-velocity}{Time evolution of the
inertial range's slope of the volume-weighted velocity powerspectra. Examples
of them are presented in \fig{plots/stirturb/powerspectra/rmsv-vw-dyntimes.png}.}

\image{1.0}{plots/decayturb/evol-slope-mw-velocity}{Time evolution of the
inertial range's slope of the mass-weighted velocity powerspectra. Examples
of them are presented in \fig{plots/stirturb/powerspectra/rmsv-mw-dyntimes.png}.}

\subsubsection{Probability Distributions}
\label{decayturb-PDF}
\image{1.0}{plots/decayturb/pdf-vw-vels.png}{Volume-weighted velocity PDFs
moving from right to left since the root-mean-square velocity decreases with
time.}

\image{1.0}{plots/decayturb/pdf-vw-dens.png}{Narrowing volume-weighted density
PDFs as time increases. The black curve is the intial distribution for all
runs. Note how due to smoothing the density peaks got truncated.}

\subsubsection{Performance}
\image{1.0}{plots/decayturb/evol-cum-runtime}{Cumulative runtime of the
different runs. For example, the Midpoint Finite-Volume solver needs 30 min to
get half-way through the simulation at $t_d = 3.0$. The graph for PPM was
scaled down by a factor of 10. \remark The contribution of input-output operations and
unintended interuptions are already substracted out.}

\image{1.0}{plots/decayturb/evol-relative-runtime}{Time derivative of
cumulative runtime (see \fig{plots/decayturb/evol-cum-runtime}). For example,
the run with the RK3-Hybrid solver would have lasted roughly 3 hours and 36
minutes with the rate at beginning: \\ 0.6 h $\cdot$ 6 = 3.6 h. The graph for PPM
was scaled down by a factor of 10.}
\image{1.0}{plots/decayturb/runtime-vs-ekin-diss}{Relative Runtime vs Kinetic
Energy Dissipation Rate. The graph for PPM was scaled down by a factor of 10.}

