\subsection{\vip{Sod} Shock Tube Problem}

Some workds about sod shock. Derivation why in isothermal case the
contact disccontinuity disappears.

Before we can discuss the results we want to validate the solvers
considered in this work. At first, we look at the classical
adiabatic one-dimensional Sod-Shock, move on to the isothermal case
and finally introduce a strong shock sitation by setting the
initial left-side velocity to $u_L = 15$.

\subsubsection{Adiabatic \& Isothermal}
\image{1.0}{plots/sod/adiabatic/density.png}{Foobar fo bar}
\image{1.0}{plots/sod/adiabatic/pressure.png}{Foobar fo bar}
\image{1.0}{plots/sod/adiabatic/velocity.png}{Foobar fo bar}

\image{1.0}{plots/sod/isothermal/density.png}{Foobar fo bar}
\image{1.0}{plots/sod/isothermal/velocity.png}{Foobar fo bar}

\subsubsection{Adiabatic \& Isothermal with Strong Shock}
\image{1.0}{plots/sod/adiabatic-supersonic/density.png}{Foobar fo bar}
\image{1.0}{plots/sod/adiabatic-supersonic/density-zoom.png}{Foobar fo bar}
\image{1.0}{plots/sod/adiabatic-supersonic/pressure.png}{Foobar fo bar}
\image{1.0}{plots/sod/adiabatic-supersonic/velocity.png}{Foobar fo bar}

\image{1.0}{plots/sod/isothermal-supersonic/density.png}{Foobar fo bar}
\image{1.0}{plots/sod/isothermal-supersonic/velocity.png}{Foobar fo bar}
\image{1.0}{plots/sod/isothermal-supersonic/velocity-zoom.png}{Foobar fo bar}
