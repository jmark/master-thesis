\subsection{\vip{Sod} Shock Tube Problem}
Before we can reliably discuss the results of the turbulence simulations we
need to validate the correctness of the solvers especially with regards to the
modifications presented in the previous chapter. The standard test for the
correct treatment of shocks is the \vip{Sod} Shock Tube Problem in one
dimension. At first, we look at the classical adiabatic test case, then
introduce polytropic cooling (isothermal) and finally apply a strong shock
sitation by setting the initial left-side velocity to $u_L = 15$. All runs took
place with a conservative CFL number of $CFL = 0.4$. If not indicated otherwise
the initial condition is set like follows
\begin{table}[H]
%\fontsize{3mm}{3mm}\selectfont
%\captionsetup{width=.6\textwidth}
\caption{Sod-Shock: Initial Condition}
\centering
\begin{tabular}{llc|lc}
\toprule
&
\multicolumn{2}{c}{Left Side} &
\multicolumn{2}{c}{Right Side} \\
\midrule
Name & Symbol & Value & Symbol & Value\\
\midrule
density         & $\dens_{L}$       & 1.0 & $\dens_{R}$ & 0.125 \\ 
velocity        & $\vels_{L}$       & 0.0 & $\vels_{R}$ & 0.1 \\ 
pressure        & $\pres_{L}$       & 1.0 & $\pres_{R}$ & 0.0 \\  
\bottomrule
\end{tabular}
\label{tab:initial-state-sod-shock}
\end{table}
We suggest a resulution of 512 elements for the whole domain $\Omega = [0,1]$
since it resembles the same resolution applied on the turbulence simulations.
This way we get a matchable insight on how well shocks can be resolved within
the periodic box of the same length.

\subsubsection{Classical Adiabatic \& Isothermal Shock}
\image{1.0}{plots/sod/adiabatic/density.png}{
Canonical density profile at conventional time $t$ = 0.2 with anticipated
regions and discontinuities outlined in the plot.}
\image{1.0}{plots/sod/isothermal/density.png}{
Isothermal density profile at conventional time $t$ = 0.2 with expected
regions and discontinuities outlined in the plot. The contact discontinuity
gets suppressed due to polytropic cooling.}
In \fig{plots/sod/adiabatic/density.png} all utilized solvers satisfy the
expected density profile of the classical Sod-Shock problem with varying
precision. PPM yields the most accurate result while on the other hand
RK3-Hybrid smears out the discontinuities considerably. Looking closely at the
zoomed area all hybrid schemes (blue, yellow and green dashed lines) tend to
oscillate.  The inacceptable ringing of the Euler-Hybrid is a consequence of
the instability of the Euler Time Integration within the DG operator. TODO:
Reference. Consequently, Euler-Hybrid is disqualified and will be discarded in
future discussion. Same assertions can be made for the isothermal case shown in
\fig{plots/sod/isothermal/density.png}.

\image{1.0}{plots/sod/adiabatic/pressure.png}{Canonical pressure profile at conventional time $t$ = 0.2.}
\image{1.0}{plots/sod/adiabatic/velocity.png}{Canonical velocity profile at conventional time $t$ = 0.2.}
%The pressure and density profiles \fig{plots/sod/adiabatic/density.png}
%and \fig{plots/sod/adiabatic/velocity.png} are shown here for completness.
\image{1.0}{plots/sod/isothermal/velocity.png}{Foobar fo bar}

\subsubsection{Adiabatic \& Isothermal with Strong Shock}
\image{1.0}{plots/sod/adiabatic-supersonic/density.png}{
Adiabatic density snapshots depicting the movement of a strong shock wave from
left to right. Note: Only the right half of the domain is shown. The pedestal
of the rightmost wave ($t_d$ = 0.04) has gone out of visible range.}
\image{1.0}{plots/sod/adiabatic-supersonic/density-zoom.png}{Zoom at the
rightmost shock wave in \fig{plots/sod/adiabatic-supersonic/density.png}.}
\image{1.0}{plots/sod/adiabatic-supersonic/pressure.png}{Foobar fo bar}
\image{1.0}{plots/sod/adiabatic-supersonic/velocity.png}{Foobar fo bar}

\image{1.0}{plots/sod/isothermal-supersonic/density.png}{Foobar fo bar}
\image{1.0}{plots/sod/isothermal-supersonic/velocity.png}{Foobar fo bar}
\image{1.0}{plots/sod/isothermal-supersonic/velocity-zoom.png}{Foobar fo bar}
