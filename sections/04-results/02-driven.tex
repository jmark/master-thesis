\subsection{Driven Turbulence}
\label{sec:stirturb}

We conduct an isothermal driven supersonic turbulence simulation in a
three-dimensional periodic box of unit length with a resolution of 512 cells in
each direction. This kind of setup is very common in ISM simulations and has
been studied extensively in the past. See
\cite{schmidt2009numerical,kitsionas2009algorithmic,0004-637X-665-1-416,mac1998kinetic}.

Originating from the driver, kinetic energy enters the box on large scales
and leaves it as internal energy via polytropic cooling on small scales. This way a
stationary flow of momentum gets pumped through the system.  In conjunction
with \emph{bulk motion correction} supersonic turbulent flow emerges, which
presents a stress test for every numerical scheme.

The simulation consists of three phases. At the beginning of the run, turbulent
forcing (cf. \sec{turbuforcing}) stirs the medium up to a certain
root-mean-square velocity, which is equivalent to the sonic Mach number
$\mach_{\text{forcing}}$ (cf. \sec{governing-equations}/Choice of Parameters).
Prior tests have shown that FLEXI becomes unstable for driven turbulences with
Mach numbers beyond 3.  Consequently, we settled to $\mach_{\text{forcing}} =
2.5$, keeping a small safety gap.

One crossing time amounts to $T_d = L/\mach = 1/2.5 = 0.4$. When reaching
$\mach = 2.5$ as fast as possible, but slow enough not to over strain the solver, the
stirring module perpetually injects just enough energy to stay in proximity of
$\mach_{\text{forcing}}$. The turbulence is given four crossing times to fully
develop, after which forcing is deactivated.  We will call this the
\emph{turbulent phase}. During the \emph{decay phase} it takes another two
turning times to let the turbulence drop below the supersonic barrier
(cf. \cite{stone1998dissipation}, \cite{mac1998kinetic}).

Based on the considerations in \sec{governing-equations}, we initialize the
state as listed in \tbl{initial-state}. 
\begin{table}[H]
%\fontsize{3mm}{3mm}\selectfont
%\captionsetup{width=.6\textwidth}
\caption{Outline of initial condition for all driven turbulence simulations}
\centering
\begin{tabular}{llc|llc}
\toprule
\multicolumn{3}{c}{primitive variables} &
\multicolumn{3}{c}{conservative variables} \\
\midrule
Name & Symbol & Value & Name & Symbol & Value\\
\midrule
density         & $\rho_0$            & 1.0 & density         & $\rho_0$            & 1.0 \\ 
x-velocity      & $\vels_{x,0}$       & 0.0 & x-momentum      & $\mom_{x,0}$           & 0.0 \\ 
y-velocity      & $\vels_{y,0}$       & 0.0 & y-momentum      & $\mom_{y,0}$           & 0.0 \\ 
z-velocity      & $\vels_{z,0}$       & 0.0 & z-momentum      & $\mom_{z,0}$           & 0.0 \\  
pressure        & $\pres_{x,0}$       & 0.6 & total energy    & $\ener_{x,0}$       & 0.9 \\  
\bottomrule
\end{tabular}
\label{tab:initial-state}
\end{table}\remark Bouchut 3 and Bouchut 5 actually solve the ideal
magneto-hydrodynamics (MHD) equations. By setting the initial magnetic density
flux $\magdensv_0 = 0$, the MHD equations reduce to the compressible Euler
equations. 

Apart from the Mach number, the course of energy and energy dissipation
are measured. Additionally, we look at selected density distributions and
powerspectra during the turbulent and decay phase. Checkpoints for analysis and
plotting are saved ten times per crossing time. \Tbl{setup-stirturb} gives an
overview of the conducted experiments.
\begin{table}[H]
%\fontsize{3mm}{3mm}\selectfont
%\captionsetup{width=.8\textwidth}
\caption{Outline of Driven Turbulence Simulations}
\centering
\begin{tabular}{l| cccc cccc}
\toprule
Solver                      & B3 & B5 & PPM & EU-FV & MP-FV & MP-HY & RK3-FV & RK3-HY \\
\midrule
Resol. per Dim.             & 512 & 512 & 512 & 512 & 512 & 512 & 512 & 512 \\
CFL                         & 0.8 & 0.8 & 0.8 & 0.4 & 0.8 & 0.6 & 0.9 & 1.2 \\
$\mach_{\text{forcing}}$    & 2.5 & 2.5 & 2.5 & 2.5 & 2.5 & 2.5 & 2.5 & 2.5 \\
$t_{\text{forcing,stop}}$   & 4.0 & 4.0 & 4.0 & 4.0 & 4.0 & 4.0 & 4.0 & 4.0 \\
$t_{\text{sim. end}}$       & 6.0 & 6.0 & 6.0 & 6.0 & 6.0 & 6.0 & 6.0 & 6.0 \\
$N_{\text{checkpoints}}$    & 60 & 60 & 60 & 60 & 60 & 60 & 60 & 60 \\
\bottomrule
\end{tabular}
\label{tab:setup-stirturb}
\end{table}

\subsubsection{Mach Number Evolution}
\Fig{plots/stirturb/evolutions/sonic-mach.png} clearly depicts the stir-up,
turbulent and decay phases. After a steep stirring phase of roughly 0.1 turning
times the forcing routine keeps the solvers from decaying below $\mach = 2.5$.
The differences in rising times and saw-tooth shapes in the zoomed picture
(\fig{plots/stirturb/evolutions/sonic-mach.png}) are a result of
different timesteps.

The decay rates from $t_d = 4$ to 6 (decay phase) diverge considerably.
The best results with regards to retaining most kinetic energy, see also
\fig{plots/stirturb/evolutions/ener.png}, is the Euler FV followed by both RK3
solvers. PPM and Bouchut 3 and Bouchut 5 are in the middle. Both
Midpoint solvers seem to be most dissipative and end up at roughly $\mach =
0.75$ at $t_d = 6$.
\image{1.0}{plots/stirturb/evolutions/sonic-mach.png}{Time evolution of the
root-mean-square Mach number $\mach$. After a steep elevation to $\mach = 2.5$
the forcing routines keep the solvers from decaying by perpetually injecting
small amounts of kinetic energy. Taking a closer look (zoomed picture) a
distinct saw-tooth profile appears.}

\subsubsection{Column Density \& Velocity}
\label{sec:stirturb-column-plots}
\Fig{plots/stirturb/snapshots/forcing-stop-01.png} to
\fig{plots/stirturb/snapshots/forcing-stop-03.png} show the column Mach number
and the column density of all eight conducted runs after four crossing times
when the forcing is stopped and the phase of turbulent decay begins. At this
point in time all simulations are expected to show a fully developed
isotropic supersonic turbulence with a variety of large and small scale
features as well an established energy cascade.

In \fig{plots/stirturb/snapshots/forcing-stop-01.png} the three solvers from
FLASH are grouped together in one page. Obviously, they fulfill the
expectations and serve as reference for the solutions brought up by FLEXI.

First of all, the solutions from Euler FV and the RK3 schemes resemble an
isotropic supersonic turbulence as well, while the Midpoints seem to differ
considerably. A cascade from large scale structures down to smaller
scales is visible. However, there are some problems resolving
structures on smallest scales. The solutions appear smoother than in
\fig{plots/stirturb/snapshots/forcing-stop-01.png}. In contrast to the rest,
Midpoint FV (\fig{plots/stirturb/snapshots/forcing-stop-02.png}) has a very
distinct feature with a large blob of high mass concentration and
contiguous regions of void swept empty by a strong velocity field.
Apparently, the large scale modes of the driving force field (cf.
\sec{turbuforcing}) at that time drags the majority of the medium into one big
lump. 

The solutions during the decay phase at $t_d = 5$
(\fig{plots/stirturb/snapshots/decay-01.png},
\fig{plots/stirturb/snapshots/decay-02.png} and
\fig{plots/stirturb/snapshots/decay-03.png})
look much more alike. Most strikingly, Euler FV and both RK3 schemes have the
most energy left in the system proofed by the color intensity of the Mach
number plots. All density plots reveal an isotropic turbulence with large and
small scale features. As mentioned before, FLEXI's solvers again show fewer
features at smallest scales compared to FLASH.

\image{1.0}{plots/stirturb/snapshots/forcing-stop-01.png}{Turbulent phase: Column Sonic Mach Number and
column density along z-axis at the time when forcing is stopped: $t_d = 4.0$.}

\image{1.0}{plots/stirturb/snapshots/forcing-stop-02.png}{Turbulent Phase: Column Sonic Mach Number and
column density along z-axis at the time when forcing is stopped: $t_d = 4.0$.}

\image{1.0}{plots/stirturb/snapshots/forcing-stop-03.png}{Turbulent Phase: Column Sonic Mach Number and
column density along z-axis at the time when forcing is stopped: $t_d = 4.0$.}

\image{1.0}{plots/stirturb/snapshots/decay-01.png}{Decay phase: Column Sonic Mach Number and
column density along z-axis during decay phase: $t_d = 5.0$.}

\image{1.0}{plots/stirturb/snapshots/decay-02.png}{Decay phase: Column Sonic Mach Number and
column density along z-axis during decay phase: $t_d = 5.0$.}

\image{1.0}{plots/stirturb/snapshots/decay-03.png}{Decay phase: Column Sonic Mach Number and
column density along z-axis during decay phase: $t_d = 5.0$.}

\subsubsection{FV-DG Mode Switching}
\label{sec:stirturb-mode-switching}
Two of the conducted runs are hybrid schemes, where shock capturing routines
decide, when a DG element must switch to FV mode in order to handle a strong
shock condition correctly. \Sec{shock-capturing} describes the procedure in
detail. \Fig{plots/stirturb/evolutions/ratio-fv-dg.png} shows the relative
amount of FV elements over dynamic time. Similar to the Mach number plot
(\fig{plots/stirturb/evolutions/sonic-mach.png}), the three stages of turbulence
are reflected in this plot. Initially, all elements are in DG mode.  During the
stir-up phase the FV elements quickly dominate and settle down on a range
between 65\% and 75\%. After four crossing times, when forcing is
stopped, the overall shock situation in the turbulence is relieved and fewer FV
elements are needed. It finally reaches a balance of 50-50.  If the simulation
would go on, the amount of FV elements is expected to eventually fall down to 0\%.

The mode switching routines follow the Mach number with some latency. This make
sense since the amount of FV elements are kind of an indicator for the
intensity of the turbulence, which inherently responds slowly to external
changes. Consequently, even after stopping the driver at $t_d = 4$, it takes at
least half the crossing time till the ratio starts to noticeably decline.

Having only around 25\% DG elements (most of the time) means their influence on
the performance of the hybrid scheme is expected to be rather small.  Nevertheless,
\fig{plots/stirturb/evolutions/ratio-fv-dg.png} proofs that the mode switching
code operates as expected. Increasing the spatial resolution should yield
better ratios in favor of DG elements. The method is a proof-of-concept with
lots of room for improvement.

\image{1.0}{plots/stirturb/evolutions/ratio-fv-dg.png}{Time evolution of the
ratio of Finite-Volume Elements to the total number of elements. The mode
switching works both ways, otherwise the plots would reach 100\%. At $t_d = 4$
the driving stops and the amount of FV elements declines. \remark The other
solvers would stay at 100\% since they operate with Finite-Volumes only.}

\newpage

\subsubsection{Energy Dissipation}
\label{sec:stirturb-energy-dissipation}
In order to keep the temperature constant, polytropic cooling deprives the exact
amount energy, which got transferred from kinetic energy to internal energy. The
physical mechanics behind this have been explained in \sec{turbulence}/Energy
Cascade.

\image{1.0}{plots/stirturb/evolutions/ener.png}{Time evolution of the energy in
the system. The total energy $\ener$ is the sum of the kinetic $\ekin$ and
internal energy $\eint$: $\ener = \eint + \ekin$.  The internal energy is held
constant via polytropic cooling (cf.  \sec{computational-frameworks}).} Now we
are interested in how much kinetic energy is actually dissipated within a
certain amount of time. A naive approach would be to take the numerical time
derivative of the evolving kinetic energy
(\fig{plots/stirturb/evolutions/ekin-dissipation.png}). But it would just yield
sensible results for the decaying phase. Instead the energy loss right before
and after polytropic cooling is recorded and divided by the current timestep.
The resulting plot is shown in
\fig{plots/stirturb/evolutions/ekin-dissipation.png}.  The rate of internal
energy deprivation is equivalent to the rate of kinetic energy dissipation,
since their is only one energy source coming from turbulent forcing and one
energy sink via cooling. During the transition from the initially constant
state to turbulent state ($t_d \approx [0,0.3]$) the dissipation rate increases
rapidly and settles down to a mean rate of around $\avg{\edyndiss}$ = 6.5
energy units per dynamic timestep. While most solvers follow a similar pattern
up to 1.5 crossing times, a change of quality seams to appear. This observation
becomes even more clear in \fig{plots/stirturb/pdfs/evol-sonic-mach.png}.  The
conducted runs part into two groups. The first group contains the Euler FV and
both RK3 solvers. They show a balanced profile and happen to be the least
dissipative ones. The second group, PPM and both Midpoints plus both Bouchuts,
jump up and down in a incoherent manner. Especially, the Midpoint solvers give rise
to huge excesses, which also reflect in the cumulative energy accounting shown
in \fig{plots/stirturb/evolutions/ekin-dissipation-cumsum.png}. Furthermore,
it seems that the situation would have gotten worse if the solvers would not
have started to settle down due to driving stop at $t_d = 4$. During the decay
phase the variations in dissipation rates are much smaller than before and even
turn upside down at $t_d \approx 4.4$. An explanation of what causes the high
pitched energy dissipation rates for the Midpoint methods is further
investigated in the next sections.

\image{1.0}{plots/stirturb/evolutions/ekin-dissipation.png}{Time evolution of
the internal energy deprivation by measuring the energy loss right before
and after polytropic cooling and dividing by the current dynamic timestep.}
\image{1.0}{plots/stirturb/evolutions/ekin-dissipation-cumsum.png}{Cumulated
energy deprivation collected by the polytropic cooling module. The plots
represent the numerical integration (trapezoid rule) of the energy deprivation
rates in \fig{plots/stirturb/evolutions/ekin-dissipation.png}. By the end of
the simulation circa 10 times the average energy (cf.
\fig{plots/stirturb/evolutions/ener.png}) was pumped through the system.}

% To get a better understanding of why the Midpoint schemes yield so strong
% dissipation spikes we take a look at the evolution of maximum root-mean-square
% velocites over time (\fig{plots/stirturb/evolutions/rmsv-max.png}). The solvers
% from FLEXI show a strong relationship between dissipation rates (cf.
% \fig{plots/stirturb/evolutions/ekin-dissipation.png}) and maximum velocity.
% 
% \image{1.0}{plots/stirturb/evolutions/eint.png}{Time evolution of the internal
% energy $\eint$ under polytropic cooling. The quantifiable deviation of the
% internal energy from the anticipated constant $\eint_{0} = 0.9$ is discussed in
% the text.}

% --------------------------------------------------------------------------- %
% Influence of Timesteps on Enstrophy

% \image{1.0}{plots/stirturb/evolutions/enstrophy}{Time evolution of the
% mean enstrophy $\enst$ which is a measure of the average mass-weighted vorticity
% in the system (cf. \sec{turbulence}/Mean Values}. It shows how much energy

% --------------------------------------------------------------------------- %

\newpage

\subsubsection{Density \& Velocity Distributions}
\label{sec:stirturb-pdf}
\paragraph{Density Distribution}
Fully developed supersonic turbulences are expected to reveal a normal
density distribution (cf. \sec{theory-density-pdf}). Additionally, it gives
an insight into the distribution of mass from smallest to largest scales.
All log-log. scale density PDFs in \fig{plots/stirturb/pdfs/dens-vw-dyntimes.png}
show good indications for the presence of turbulence. The dotted lines are
log-normal fits of \eqn{density-pdf} in \sec{theory-density-pdf}. The notorious
underestimation of the fits on large scales is caused by the under-resolvement
of the setup. While during the turbulent phase
($t_d = [2,4]$) the solvers from FLASH have considerably more mass on small
scales, this is not the case anymore during the decay phase; exemplified by
$t_d = 4.7$ and $t_d = 6.0$.
\image{1.0}{plots/stirturb/pdfs/dens-vw-dyntimes.png}{Log-log. scale
volume-weighted density PDFs for the turbulent phase $t_d = [2,4]$
(time-average) and two stages of the decaying phase: $t_d = 4.7$ and $t_d =
6.0$. The faint grey area marks the margin of error of the time-averaging. As
the decay progresses, the width of the distributions decrease and the mean
values return to the initial density:
$\ld(\dens_0) = \ld(1) = 0$. The log-normal fit (dotted lines) were done with
\eqn{density-pdf-mach}.}
The width $\sigma_s$ of the density distribution in
\fig{plots/stirturb/pdfs/dens-vw-dyntimes.png} are related to the sonic Mach
number of the system (cf. \sec{theory-density-pdf}). Applying
\eqn{density-pdf-mach} and \eqn{density-pdf-mach-constant}, where
\begin{equation}
b = 1 + (D^{-1} - 1) \, \zeta = 1 + (3^{-1} - 1) \, 0.5 = 2/3,
\end{equation}
we can recover the Mach number of the turbulence. The result is shown in
\fig{plots/stirturb/pdfs/evol-sonic-mach.png}. The plot is similar in shape to
the energy dissipation in \fig{plots/stirturb/evolutions/ekin-dissipation.png}.
After two crossing times a change of quality seems to happen for some of the
solvers, accompanied by a lot of noise over time but better reconstruction of
the reference sonic Mach number of $\Mach = 2.5$. Again the Midpoint FV has a
huge spike near the end of the turbulent  phase ($t_d = 4$), which we will take
a closer look at further down. Euler FV and both RK3 schemes severely
underestimate the Mach number by 20\%. During the decay phase all fits yield
smaller values. It might be the case that in this phase the relation in
\eqn{density-pdf-mach} breaks down. \Tbl{pdf-dens-vw} lists the fitting results
of the time-averaged turbulent phase from 2 to 4 crossing times, where the
results seem most promising.

\image{1.0}{plots/stirturb/pdfs/evol-sonic-mach.png}{Time evolution of the
sonic mach number over time derived from the width of the volume-weighted
density PDFs via relation \eqn{density-pdf-mach} ($b = 2/3$). The dashed line
represents the root-mean-square velocity of the PPM solver and serves as a
reference. Due to the log-normal relationship, each individual fit yields very
low error margins. Hence, the error bars are to small to see.}
\begin{table}[H]
\fontsize{4mm}{4mm}\selectfont
\captionsetup{width=.95\textwidth}
\caption{Comparison of Gaussian Fitting Results of the Log-normal Density PDF
averaged over the turbulent phase: $t_d = [2,4]$. See relation
\eqn{density-pdf-mach}. $b = 2/3$. The PDF Mach number $\mach_{PDF}$ of the
Midpoint FV overshoots due to the influence of the runaway. Compare
\fig{plots/stirturb/pdfs/evol-sonic-mach.png}.}
\centering
\begin{tabular}{lcccc}
\toprule
Solver & Mean $s_0$ & Std. Deviation $\sigma_s$ & RMS Mach $\mach_{RMS}$ & PDF Mach $\mach_{PDF}$ \\
\midrule
\bouT & -0.31 $\pm$ 0.05 & 0.58 $\pm$ 0.06 & 2.63 $\pm$ 0.13 & 2.5 $\pm$ 0.4 \\
\bouF & -0.29 $\pm$ 0.06 & 0.57 $\pm$ 0.08 & 2.59 $\pm$ 0.09 & 2.4 $\pm$ 0.6 \\
\fppm & -0.32 $\pm$ 0.05 & 0.59 $\pm$ 0.06 & 2.62 $\pm$ 0.12 & 2.6 $\pm$ 0.5 \\
\eufv & -0.24 $\pm$ 0.03 & 0.47 $\pm$ 0.04 & 2.56 $\pm$ 0.04 & 1.8 $\pm$ 0.2 \\
\mpfv & -0.31 $\pm$ 0.04 & 0.60 $\pm$ 0.09 & 2.80 $\pm$ 0.40 & 2.7 $\pm$ 0.8 \\
\mphy & -0.32 $\pm$ 0.05 & 0.56 $\pm$ 0.07 & 2.81 $\pm$ 0.18 & 2.4 $\pm$ 0.5 \\
\rkfv & -0.26 $\pm$ 0.02 & 0.50 $\pm$ 0.03 & 2.61 $\pm$ 0.06 & 2.0 $\pm$ 0.2 \\
\rkhy & -0.26 $\pm$ 0.03 & 0.48 $\pm$ 0.04 & 2.55 $\pm$ 0.04 & 1.9 $\pm$ 0.3 \\
\bottomrule
\end{tabular}
\label{tab:pdf-dens-vw}
\end{table}

\paragraph{Velocity Distribution}
An exceeding dispersion of velocity PDFs indicates strong bulk motions within
the medium. Bulk motion (cf. \sec{turbulence}) distorts the turbulence by
drawing off kinetic energy from the turbulence, while at the same time the
root-mean-square velocity of the system remains constant. In the worst case the
energy cascade breaks down or does not even develop in the first place.
\Fig{plots/stirturb/pdfs/vels-vw-dyntimes.png} lacks any considerable
dispersion, which means the bulk motion correction (cf.
\sec{computational-frameworks}) is working correctly. More distinctive is
the bulge of high velocities, brought out by the Midpoint schemes,
accompanied by huge error margins. This means that during the turbulent phase
there must be extreme velocity fluctuations up to Mach 30 and beyond.
\Fig{plots/stirturb/evolutions/rmsv-max.png} affirms this observation.
Comparing this plot with the energy dissipation rates in
\fig{plots/stirturb/evolutions/ekin-dissipation.png}, there is a correlation
between maximum velocity bursts and high pitched dissipation. Obviously, this
applies only to the Midpoint schemes. The maximum velocity plot indicates only
a singular value, which does not necessarily represents a prevalent phenomenon in
the system. In conjunction with the velocity PDF however, this can be considered
true for the Midpoint schemes.
\image{1.0}{plots/stirturb/pdfs/vels-vw-dyntimes.png}{Volume-weighted velocity
PDFs computed from the checkpoint files for the turbulent phase $t_d = [2,4]$
(time-average) and two stages of the decaying phase: $t_d = 4.7$ and $t_d =
6.0$. The faint grey area marks the margin of error of the time-averaging.  The
Midpoint schemes yield strong fluctuation of high velocities during the
turbulent phase (cf. \fig{plots/stirturb/evolutions/rmsv-max.png}).}
\image{1.0}{plots/stirturb/evolutions/rmsv-max.png}{Maximum root-mean-square
velocity over time retrieved from the checkpoint files. The Midpoint schemes
show very strong maximum velocity bursts up to Mach 30 and beyond during the
turbulent phase.}

As already discussed under \sec{stirturb-column-plots}, the Midpoint FV method
shows a unique pattern of slow-moving mass lumps and regions of highly
accelerated but very thin gas. When mass-weighting the velocity distribution,
see \fig{plots/stirturb/pdfs/vels-mw-turb-phase.png}, we get another picture.
The bulge on the right side of \fig{plots/stirturb/pdfs/vels-vw-dyntimes.png}
completely disappears, which supports the observation of highly accelerated thin
gas in the Midpoint FV run. By contrast, Bouchut 3, Bouchut 5 and PPM show
a richer structure in the slow moving range.

%Now instead the impression that structures of smallest scales are absent for
%the FLEXI solvers becomes evident.

\image{1.0}{plots/stirturb/pdfs/vels-mw-turb-phase.png}{Log-log. scale
mass-weighted velocity PDF time-averaged over the turbulent phase ($t_d =
[2,4]$). The gray areas mark the margin of error. The solvers from FLASH have a
greater amount of slowly moving mass than the others. The bulge on the right
side in \fig{plots/stirturb/pdfs/vels-vw-dyntimes.png} is gone due to fewer
mass in the high-velocity range.}

% --------------------------------------------------------------------------- %

\subsubsection{Energy \& Velocity Powerspectra}
\label{sec:stirturb-pws}
Powerspectra give an insight into the distribution of a physical quantity from
large to small length scales. In this section we discuss the energy and
velocity powerspectra, which were introduced in \sec{theory-powerspectra}. The
powerspectrum is divided into three ranges. In the driving or large scale range
the energy injection from the turbulent driver takes place. It spans over the
first three to four modes.  From there the energy trickles down to smaller and
smaller scales, called energy cascade, which takes place in the inertial range.
Finally it reaches the dissipative range which begins in our case at roughly
$k_{diss} \approx 25$.  This is where the kinetic energy dissipates and gets
transformed into internal energy.  In the powerspectrum of the kinetic energy,
\fig{plots/stirturb/powerspectra/ekin-dyntimes.png}, the boundaries of the
mentioned ranges are drawn in. According to Parseval's theorem the normed area
under the kinetic energy powerspectrum is equal to the total squared
real-valued kinetic energy in the box. (cf. \sec{theory-powerspectra}).
\Tbl{stirturb-pws} proofs via numerical integration that this is indeed true
within the margins of error.  In agreement with the discussion of column
densities and the density distributions in the previous sections, the solutions
provided by FLASH's solvers contain more energy on small scales than those with
FLEXI. To quantify this observation the area under the dissipative range was
estimated and compared to the total squared kinetic energy.  Consulting the
right-most column in \tbl{stirturb-pws} the relative amount of energy on small
scales ranges from 2.4\% (RK3 FV) to 5.7\% (Bouchut 5).

\image{1.0}{plots/stirturb/powerspectra/ekin-dyntimes.png}{Powerspectra of the
volume-weighted mean kinetic energy field shown for the turbulent phase $t_d =
[2,4]$ (time-average) and two stages of the decay phase: $t_d = 4.5$ and $t_d =
6.0$.  When the total kinetic energy declines, the powerspectrum shifts
downwards. The inertial range starts at $k = 4$ since forcing is applied on
$k_{drive} = 1,2,3$. The beginning of the dissipative range is estimated to
$k_{diss} \approx 25$ when the spectra start the bend down. The study in
\cite{kitsionas2009algorithmic} came to the same conclusion for equal grid
resolution of $512^3$.}

Besides the log-normality of the density PDF (\sec{stirturb-pdf}) the
slopes of the inertial range of velocity powerspectra are well-established
indicators for the correct modelling of turbulences. According to the theory
(cf.  \sec{theory-powerspectra}) we expect slopes of $m_{vw} =
-2.1$ and $m_{mw} = -1.6667$ for the volume-weighted and mass-weighted
powerspectra of the velocity field, respectively.
\Fig{plots/stirturb/powerspectra/rmsv-vw-dyntimes.png} and
\fig{plots/stirturb/powerspectra/rmsv-mw-dyntimes.png} indeed reproduce
given slopes for the time-averaged powerspectrum of the turbulent phase.  The
spectra at the end of the simulation, when the turbulent decay progressed
considerably, are plotted as well. Obviously, the inertial range has leveled
down. This makes sense, considering the fact that the supply of kinetic energy
was cut off. Hence, the large scale structures gradually dissolve and
push their energy down the cascade. The change of slope can be seen in
\fig{plots/stirturb/powerspectra/evol-slopes-vw-velocity.png} very clearly. In
the beginning the inertial range is very steep. The turbulent driver is pumping
energy into the system on first three modes. As the simulation continues the
energy cascade builds up and is fully established after two crossing times.
When entering the decay phase ($t_d = 4$), the slope flattens out.
\Tbl{stirturb-pws} lists the results of the linear fits of the inertial range
as well as the numerically integrated areas under the time-averaged powerspectra
of the turbulent phase.
\image{1.0}{plots/stirturb/powerspectra/rmsv-vw-dyntimes.png}{Powerspectra of
the volume-weighted velocity field for the turbulent phase $t_d = [2,4]$
(time-average) and at the end of the decaying phase $t_d = 6.0$. 
The inertial range starts at $k = 4$ since forcing is applied on $k_{drive} =
1,2,3$. The black solid line marks the expected slope $m_{vw} = -19/9$ (cf.
\sec{theory-powerspectrum}). In agreement with \cite{kitsionas2009algorithmic}
(resolution: $512^3$), the beginning of the dissipative range is estimated to
$k_{diss} \approx 25$, where the spectra start the bend down.}
\image{1.0}{plots/stirturb/powerspectra/rmsv-mw-dyntimes.png}{Powerspectra of
the mass-weighted velocity field for the turbulent phase $t_d = [2,4]$
(time-average) and at the end of the decaying phase $t_d = 6.0$.
The black solid line marks the expected slope $m_{vw} = -5/3$ (cf.
\sec{theory-powerspectrum}). In agreement with \cite{kitsionas2009algorithmic}
(resolution: $512^3$), the beginning of the dissipative range is estimated to
$k_{diss} \approx 25$, where the spectra start the bend down.
}
\image{1.0}{plots/stirturb/powerspectra/evol-slopes-vw-velocity.png}{Time evolution of the
inertial range's slope of the volume-weighted velocity powerspectra. See 
\fig{plots/stirturb/powerspectra/rmsv-vw-dyntimes.png}.}

%\image{1.0}{plots/stirturb/powerspectra/evol-slopes-mw-velocity.png}{Time evolution of the
%inertial range's slope of the mass-weighted velocity powerspectra. Examples
%are presented in \fig{plots/stirturb/powerspectra/rmsv-vw-dyntimes.png}.}

%\image{1.0}{plots/stirturb/powerspectra/evol-ekin-small-scales.png}{Amount of
%small-scale energies prevailing in the dissipative length scales over time.}

\begin{table}[H]
\fontsize{3mm}{4mm}\selectfont
\captionsetup{width=.8\textwidth}
\caption{Comparison of the three types of powerspectra averaged over the turbulent phase: $t_d = [2,4]$}
\centering
\begin{tabular}{l|cc|cccc}
\toprule
Solver & Slope $m$ & Offset $n$ & Mean Squ. $\avg{\cdot^2}$ & Area $A$ & Diss. Area $A_{diss}$ & $A_{diss}/A$ [\%] \\
\midrule
\multicolumn{7}{c}{Volume-weighted Mean Kinetic Energy} \\
\midrule
\bouT & -1.16 $\pm$ 0.05 & 17.19 $\pm$ 0.24 & 39 $\pm$ 4 & 37 $\pm$ 8 & 2.00 $\pm$ 0.60 & 5.4 $\pm$ 1.7 \\
\bouF & -1.19 $\pm$ 0.10 & 17.22 $\pm$ 0.30 & 40 $\pm$ 5 & 37 $\pm$ 8 & 2.10 $\pm$ 0.70 & 5.7 $\pm$ 1.8 \\
\fppm & -1.14 $\pm$ 0.07 & 17.19 $\pm$ 0.24 & 38 $\pm$ 4 & 36 $\pm$ 8 & 1.37 $\pm$ 0.33 & 3.8 $\pm$ 0.9 \\
\eufv & -1.06 $\pm$ 0.04 & 17.11 $\pm$ 0.30 & 35 $\pm$ 5 & 34 $\pm$ 9 & 0.88 $\pm$ 0.35 & 2.6 $\pm$ 1.0 \\
\mpfv & -1.10 $\pm$ 0.05 & 17.18 $\pm$ 0.27 & 37 $\pm$ 6 & 35 $\pm$ 9 & 1.00 $\pm$ 0.50 & 3.0 $\pm$ 1.3 \\
\mphy & -1.06 $\pm$ 0.02 & 17.15 $\pm$ 0.17 & 36 $\pm$ 4 & 34 $\pm$ 7 & 1.12 $\pm$ 0.33 & 3.3 $\pm$ 1.0 \\
\rkfv & -1.07 $\pm$ 0.08 & 17.12 $\pm$ 0.31 & 33 $\pm$ 4 & 32 $\pm$ 8 & 0.75 $\pm$ 0.28 & 2.4 $\pm$ 0.9 \\
\rkhy & -1.17 $\pm$ 0.17 & 17.23 $\pm$ 0.34 & 33 $\pm$ 3 & 32 $\pm$ 6 & 0.81 $\pm$ 0.25 & 2.5 $\pm$ 0.8 \\
\midrule
\multicolumn{7}{c}{Volume-weighted Velocity} \\
\midrule
\bouT & -1.89 $\pm$ 0.13 & 16.67 $\pm$ 0.28 & 4.00 $\pm$ 0.40 & 4.0 $\pm$ 0.9 & 0.0152 $\pm$ 0.0034 & 0.38 $\pm$ 0.09 \\
\bouF & -1.87 $\pm$ 0.11 & 16.63 $\pm$ 0.23 & 3.86 $\pm$ 0.26 & 3.8 $\pm$ 0.7 & 0.0179 $\pm$ 0.0035 & 0.47 $\pm$ 0.09 \\
\fppm & -2.00 $\pm$ 0.15 & 16.76 $\pm$ 0.31 & 4.00 $\pm$ 0.40 & 4.0 $\pm$ 0.9 & 0.0143 $\pm$ 0.0029 & 0.36 $\pm$ 0.07 \\
\eufv & -2.08 $\pm$ 0.10 & 16.80 $\pm$ 0.15 & 3.86 $\pm$ 0.16 & 3.8 $\pm$ 0.6 & 0.0068 $\pm$ 0.0012 & 0.18 $\pm$ 0.03 \\
\mpfv & -2.09 $\pm$ 0.05 & 17.00 $\pm$ 0.70 & 6.00 $\pm$ 4.00 & 6.0 $\pm$ 0.5 & 0.0150 $\pm$ 0.0130 & 0.24 $\pm$ 0.22 \\
\mphy & -2.11 $\pm$ 0.03 & 16.94 $\pm$ 0.21 & 4.80 $\pm$ 0.90 & 4.8 $\pm$ 1.4 & 0.0102 $\pm$ 0.0030 & 0.21 $\pm$ 0.06 \\
\rkfv & -2.10 $\pm$ 0.08 & 16.87 $\pm$ 0.16 & 3.96 $\pm$ 0.17 & 3.9 $\pm$ 0.6 & 0.0074 $\pm$ 0.0015 & 0.19 $\pm$ 0.04 \\
\rkhy & -2.08 $\pm$ 0.04 & 16.85 $\pm$ 0.12 & 3.79 $\pm$ 0.14 & 3.9 $\pm$ 0.6 & 0.0073 $\pm$ 0.0015 & 0.19 $\pm$ 0.04 \\
\midrule
\multicolumn{7}{c}{Mass-weighted Velocity} \\
\midrule
\bouT & -1.56 $\pm$ 0.10 & 16.45 $\pm$ 0.19 & 3.158 $\pm$ 0.021 & 3.1 $\pm$ 0.4 & 0.054 $\pm$ 0.010 & 1.72 $\pm$ 0.33 \\
\bouF & -1.57 $\pm$ 0.13 & 16.45 $\pm$ 0.22 & 3.158 $\pm$ 0.024 & 3.1 $\pm$ 0.4 & 0.060 $\pm$ 0.011 & 1.90 $\pm$ 0.40 \\
\fppm & -1.59 $\pm$ 0.07 & 16.49 $\pm$ 0.17 & 3.166 $\pm$ 0.020 & 3.1 $\pm$ 0.5 & 0.043 $\pm$ 0.007 & 1.38 $\pm$ 0.24 \\
\eufv & -1.61 $\pm$ 0.07 & 16.48 $\pm$ 0.15 & 3.134 $\pm$ 0.004 & 3.1 $\pm$ 0.4 & 0.021 $\pm$ 0.004 & 0.69 $\pm$ 0.13 \\
\mpfv & -1.60 $\pm$ 0.04 & 16.52 $\pm$ 0.15 & 3.128 $\pm$ 0.006 & 3.0 $\pm$ 0.4 & 0.028 $\pm$ 0.009 & 0.93 $\pm$ 0.29 \\
\mphy & -1.60 $\pm$ 0.03 & 16.55 $\pm$ 0.11 & 3.133 $\pm$ 0.006 & 3.2 $\pm$ 0.5 & 0.029 $\pm$ 0.005 & 0.92 $\pm$ 0.17 \\
\rkfv & -1.61 $\pm$ 0.05 & 16.52 $\pm$ 0.12 & 3.137 $\pm$ 0.011 & 3.2 $\pm$ 0.4 & 0.022 $\pm$ 0.004 & 0.70 $\pm$ 0.13 \\
\rkhy & -1.62 $\pm$ 0.12 & 16.54 $\pm$ 0.19 & 3.138 $\pm$ 0.008 & 3.3 $\pm$ 0.5 & 0.022 $\pm$ 0.004 & 0.68 $\pm$ 0.13 \\
\bottomrule
\end{tabular}
\label{tab:stirturb-pws}
\end{table}

% --------------------------------------------------------------------------- %

\subsubsection{Summary}
In the driven supersonic turbulence setup we analyzed the various solutions of
FLASH's and FLEXI's solver over the course of 6 crossing times. All runs
consisted of three phases or stages. In the first phase the initially restful
medium was stirred up by turbulent forcing to the root-mean-square velocity
of $\mach$ = 2.5. After reaching this limit, the driver kept the kinetic
energy stable for 4 crossing times so that a turbulent energy cascade develops. 
After the turbulent phase, forcing was deactivated and the turbulence
decayed for another 2 crossing times.

By analyzing density/velocity distributions and powerspectra it was confirmed
that all solvers modeled a supersonic turbulence correctly. They confirmed a
log-normal density profile and their inertial ranges descended with the correct
slopes. However, there are differences regarding energy dissipation rates and
resolution of smallest scales revealing an inverse relationship. Suggested by
the cumulative energy budget in
\fig{plots/stirturb/evolutions/ekin-dissipation-cumsum.png}, Euler FV burnt the
least amount of kinetic energy contrasting PPM, Bouchut 3 and Bouchut 5, who
consumed roughly 15 to 20\% more. On the other hand the latter yield more
details on small scales, which becomes clearly evident in the mass-weighted
velocity distribution in \fig{plots/stirturb/pdfs/vels-mw-turb-phase.png}.
Estimating the energy in the dissipative range, see \tbl{stirturb-pws}, does
not contradict this observation. We will put this into perspective later on.

The behaviour of the Midpoint schemes falls out of place and should discussed
separately. It seems that the turbulent driver creates a force field in such a
unique way and over a longer period of time so that the majority of the mass is
dragged into a lump surrounded by highly accelerated regions of very thin
medium. See \fig{plots/stirturb/snapshots/forcing-stop-02.png} for a snapshot.
The dissolution of those extremely high Mach regions drives up the energy
dissipation substantially. See
\fig{plots/stirturb/evolutions/ekin-dissipation.png}.  This remarkable example
reveals the imprinting influence of the driver on the turbulence, which is
a problem in its own right.

In order to gain insights on how strong the influence of the turbulent driver
might be, we also conducted a decaying turbulence simulation with exact equal
initial conditions. The results are described in the next section.
