\subsection{Decaying Turbulence}
\label{sec:decayturb}

In the previous section we conducted a driven turbulence simulation where we
identified a significant influence of the turbulent driver on the modelled
turbulence. Consequently, we test each solver with a decaying turbulence setup
of identical initial conditions. A snapshot of a fully developed Mach-10
turbulence, generated by the \bouF solver provides the basis for the
simulations. The initial turbulence setup is left to decay six turning times
relative to $\mach = 10 \Rightarrow T_{turn} = 1/10 = 0.1$. In order to avoid
interpolation erros and unsolicited oscillations during transfer the initial
state was slightly smoothed beforehand via application of a Gaussian blur (cf.
\fig{plots/interpolation/flexi-column-density.png}). Apart from that the setup
is equivalent to the driven turbulence setup detailed at the beginning of
\sec{stirturb}. \Tbl{setup-decayturb} gives an overview of the conducted
experiments.
\begin{table}[H]
\fontsize{3mm}{3mm}\selectfont
%\captionsetup{width=.8\textwidth}
\caption{Outline of the Decaying Turbulence Setup. Note the small CFL number
for PPM which turned out to be necessary in order not to crash. There is a
slight variance among the initial Mach numbers/kinetic energies. They stem from
the fact that these values were measured at runtime when
subjected to minor numerical errors.}
\centering
\begin{tabular}{l| cccc cccc}
\toprule
Solver                          & B3 & B5 & PPM & EU-FV & MP-FV & MP-HY & RK3-FV & RK3-HY \\
\midrule
CFL number                            & 0.8 & 0.8 & 0.1 & 0.4 & 0.8 & 0.9 & 0.9 & 1.2 \\ 
initial mach number $\mach_0$   & 9.877 & 9.877 & 9.877 & 9.875 & 9.872 & 9.873 & 9.871 & 9.871 \\
initial kin. energy $\ekin_0$   & 48.78 & 48.78 & 48.78 & 48.76 & 48.72 & 48.73 & 48.72 & 48.72 \\
sim. end $t_{\text{sim. end}}$           & 6.0 & 6.0 & 6.0 & 6.0 & 6.0 & 6.0 & 6.0 & 6.0 \\
nr. checkpts. $N_{\text{checkpoints}}$        & 30 & 30 & 30 & 30 & 30 & 30 & 30 & 30 \\
\bottomrule
\end{tabular}
\label{tab:setup-decayturb}
\end{table}

\subsubsection{Mach Number Evolution}
In contrast to the different decay rates in
\fig{plots/stirturb/evolutions/sonic-mach.png} the declines of the Mach number
over time (\fig{plots/decayturb/evol-sonic-mach}) are unexpectedly alike.  The
decaying turbulences start at $\mach = 10$ and take six crossing times to drop
below the supersonic regime ($\mach = 1$). 
\image{1.0}{plots/decayturb/evol-sonic-mach}{Time evolution of the
root-mean-square mach number.}

\subsubsection{Column Density \& Velocity}
\label{sec:decayturb-column-plots}
Here we present a series of column density and velocity snapshots of all
conducted simulations at three successive stages of turbulent decay: $t_d =
[1,2,3]$ with $\mach = [4.2,2.4,1.9]$, respectively. See
\fig{plots/decayturb/decay-01-01.png} to \fig{plots/decayturb/decay-03-03.png}.
The exact point of time, the solver and the Mach number are attached to each
snapshot in the title and the figure caption.

First of all the large scale structures are congruent among all solvers
comparing the snapshots depicting equal points in time. As already discussed
in \sec{theory-powerspectrum} dissipation happens only at small scales.
And since the velocity field is not influenced by external forces the large
scale movements are conserved.

The originially smoothed initial state (cf.
\fig{plots/interpolation/flexi-column-density.png}) reestablished the shocks
which can be clearly seen by the sharp edges of the filamentry structures in
\fig{plots/decayturb/decay-01-01.png} to \fig{plots/decayturb/decay-01-03.png}.
As expected those structures gradually dissolve with increasing time which
leads to a transformation of kinetic energy into internal energy. It then
escapes the system via polytropic cooling. This process is indicated by
the increased darkening of the column Mach number plots as time progresses.

It is impossible to tell from the snapshots if there are differences in
the resolution of small scales. In analogy to the driven turbulence setup
density PDFs and powerspectra will give a better insight into this matter.

\image{1.0}{plots/decayturb/decay-01-01.png}{Decaying Turbulence from Mach 10:
Column Sonic Mach Number and column density along z-axis at $t_d = 1.0$ when
the root-mean-square mach number has fallen to $\mach = 4.2$}

\image{1.0}{plots/decayturb/decay-01-02.png}{Decaying Turbulence from Mach 10:
Column Sonic Mach Number and column density along z-axis at $t_d = 1.0$ when
the root-mean-square mach number has fallen to $\mach = 4.2$}

\image{1.0}{plots/decayturb/decay-01-03.png}{Decaying Turbulence from Mach 10:
Column Sonic Mach Number and column density along z-axis at $t_d = 1.0$ when
the root-mean-square mach number has fallen to $\mach = 4.2$}

\image{1.0}{plots/decayturb/decay-02-01.png}{Decaying Turbulence from Mach 10:
Column Sonic Mach Number and column density along z-axis at $t_d = 2.0$ when
the root-mean-square mach number has fallen to $\mach = 2.4$}

\image{1.0}{plots/decayturb/decay-02-02.png}{Decaying Turbulence from Mach 10:
Column Sonic Mach Number and column density along z-axis at $t_d = 2.0$ when
the root-mean-square mach number has fallen to $\mach = 2.4$}

\image{1.0}{plots/decayturb/decay-02-03.png}{Decaying Turbulence from Mach 10:
Column Sonic Mach Number and column density along z-axis at $t_d = 2.0$ when
the root-mean-square mach number has fallen to $\mach = 2.4$}

\image{1.0}{plots/decayturb/decay-03-01.png}{Decaying Turbulence from Mach 10:
Column Sonic Mach Number and column density along z-axis at $t_d = 3.0$ when
the root-mean-square mach number has fallen to $\mach = 1.9$}

\image{1.0}{plots/decayturb/decay-03-02.png}{Decaying Turbulence from Mach 10:
Column Sonic Mach Number and column density along z-axis at $t_d = 3.0$ when
the root-mean-square mach number has fallen to $\mach = 1.9$}

\image{1.0}{plots/decayturb/decay-03-03.png}{Decaying Turbulence from Mach 10:
Column Sonic Mach Number and column density along z-axis at $t_d = 3.0$ when
the root-mean-square mach number has fallen to $\mach = 1.9$}

\subsubsection{FV-DG Mode Switching}
In \fig{plots/decayturb/evol-fv} the initially smoothed turbulence immediately
shapes strong shocks everywhere, hence the sudden jump to over 95\% FV
elements.  After that the situation alleviates a bit and more DG elements can
be reintroduced. For not too long the second culmination of shocks appears and
the amount of FV elements rises again. From $t_d = 1$ onwards the ratio
declines in the same fashion as in
\fig{plots/stirturb/evolutions/ratio-fv-dg.png}.

\image{1.0}{plots/decayturb/evol-fv}{Time evolution of the fraction of
Finite-Volume Elements to the total number of elements. \remark The other
solvers would stay at 100\% since they operate solely with Finite-Volumes.}

\subsubsection{Energy Dissipation}
\label{sec:decayturb-ekin-diss}
The course of the total energy \fig{plots/decayturb/evol-ener} is plotted in
logarithmic scale so that the kinetic energy can be visually distinguished from
the total energy. Polytropic cooling keeps the system at constant temperature
hence the internal energy stays steady over time. There is no obvious 
difference among the solvers.
\image{1.0}{plots/decayturb/evol-ener}{Time evolution of the energy. The
logarithmic scaling helps to visually distinguish to course of total and
kinetic energy. The total energy $\ener$ is the sum of the kinetic
$\ekin$ and internal energy $\eint$: $\ld(\ener) = \ld(\eint + \ekin)$.}

% \image{1.0}{plots/decayturb/evol-eint}{Time evolution of the internal energy
% $\eint$ before applying polytropic cooling. The quantifiable deviation of the internal
% energy from the anticipated constant $\eint_{0} = 0.9$ is discussed in the
% text.}

The deprived internal energy is equivalent to the dissipated kinetic energy
(\fig{plots/decayturb/evol-ekin-diss}). See discussion under
\sec{stirturb-energy-dissipation}. Right after simulation has started
strong shocks emerge and the dissipation rate shoots up to its highest peak.
By applying numerical integration under the averaged plot of the solvers $\int
dt_d \; -\frac{d\ekin}{dt_d} = 48.251 \pm 0.007$  the initial kinetic energy is
recovered $\ekin_0 = 48.75 \pm 0.03$ besides a tiny fractioni of $\Delta \ekin
= 0.5 \pm 0.03$ still left in the system. This little calculation reveals two
insights. First, there are no noticable energy sources or sinks besides
polytropic cooling. Thus, the energy accounting is balanced. Second, the
variation in dissipated energy among solvers is neglectible. Conclusively,
there is no difference in dissipation with this setup which contradicts 
the results of the driven turbulence.
\image{1.0}{plots/decayturb/evol-ekin-diss}{Time evolution of the
kinetic energy dissipation in logarithmic scale $\ld(-\frac{d\ekin}{dt_d})$.
The averaged area (numerical integration) under the plots amounts to
$\int dt_d \; -\frac{d\ekin}{dt_d} = 48.251 \pm 0.007$ which is almost all of the
the initial kinetic energy: $\ekin_0 = 48.75 \pm 0.03$.}

% The enstrophy which is a meassure of dissipative small scale structures
% prevailing the system shows a similar picture as before but differs
% considerably among solvers.  The fewer amount of enstrophy with the \fppm can be
% explained by the very small timesteps necessary in order to be stable. Compare
% \fig{plots/decayturb/evol-timestep}. This relation has been observed before
% in previous similar runs and is a consequence of higher dissipation at least on
% small scales due to many more passes per physical time.
% \image{1.0}{plots/decayturb/evol-enstrophy}{Time evolution of the
% mean enstrophy $\enst$.}

\subsubsection{Density \& Velocity Distributions}
\label{sec:decayturb-PDF}
In agreement with \fig{plots/stirturb/pdfs/dens-vw-dyntimes.png} FLEXI's
solvers retain more mass on the outer scales than FLASH's. Fitting was done
with \eqn{density-pdf-mach} in the same manner as in the driven turbulene (cf.
\sec{stirturb-pdf}). From the width of the log-normal fits we try to reconstruct
the sonic Mach number at that time. The result is shown in
\fig{plots/decayturb/evol-sonic-mach-pdf.png}.

The volume-weighted velocity PDFs in \fig{plots/decayturb/pdf-vw-vels.png} do
not show anything suprising besides the fact that there is no significant bulk
motion (cf. \sec{sec:stirturb-pdf}).

\image{1.0}{plots/decayturb/pdf-vw-dens.png}{Log-log scale volume-weighted
density PDFs. As the decay progresses the widths of the distributions decrease
and the mean values return to the intial density: $\ld(\dens_0) = \ld(1) = 0$.
The solid black curve depicts the density distribution of the smoothed initial
state (cf. \fig{plots/interpolation/flexi-column-density.png}). The applied
Gaussian blur truncated the high velocity range. The log-normal fit (dotted
lines) were done with \eqn{density-pdf-mach}.}
\image{1.0}{plots/decayturb/evol-sonic-mach-pdf.png}{Time evolution of the
sonic mach number over time derived from the width of the volume-weighted
density PDFs of the decaying turbulence. The dashed line represents the
root-mean-square velocity of the PPM solver and serves as a reference. Due to
the lognormal relationship the fit yielded very low error margins. Hence, the
error bars are to small to see. The spike on the left is caused by the
originally smoothed initial state. The turbulence has to redevelop in order to
yield sensible results. However, the Mach number estimation is
systematically below the reference nevertheless.}
\image{1.0}{plots/decayturb/pdf-vw-vels.png}{Log-log. scale volume-weighted
velocity PDFs moving from right to left with increasing time due to kinetic
energy decline of the decaying turbulence. The median point of the initial PDF
at $t_d = 0$ is precisely over Mach 10: $\mu_0 = \ld(1) = 10$. A non-existent
dispersion of the PDFs over time indicates no bulk motion which is a good
sign (cf. \sec{stirturb-pdf}).}

\subsubsection{Energy \& Velocity Powerspectra}
In analogy with \sec{stirturb-pws} we will take a look at the kinetic
energy and the velocity powerspectra. In addition to the density PDF discussed
in the previous section they allow us to gain an insight into the resolution
of movement on small spatial scales. Since the kinetic energy declines over
time the powerspectra in \fig{plots/decayturb/pws-ekin-dyntimes} shift downwards.
At the end of the simulation $t_d = 6$ PPM retained the most energy in the
dissipation range. It is interesting to see that there still is an imprint left from
the turbulent driver who produced the original turbulence in the first place.
The buckling of the spectra within the large scale range is typical for
driven turbulences. During the evaluation of the column density/velocity snapshots
under \sec{decayturb-column-plots} we discussed the preservation of large scale
structures over long periods of time. This observation resurfaces here again.
\image{1.0}{plots/decayturb/pws-ekin-dyntimes}{Powerspectra of the
volume-weighted mean kinetic energy field shown for three stages of decay: $t_d
= 0$, $t_d = 3.0$ and $t_d = 6.0$.  Since the total kinetic energy declines over
time the powerspectra shift downwards. The strange looking spectrum from $t_d = 0$
is the result of the applied Gaussian blur on the initial state.}

\Fig{plots/decayturb/pws-vw-vels-dyntimes.png} and
\fig{plots/decayturb/pws-mw-vels-dyntimes.png} show that all simulations
maintain the energy cascade already presenent in the initial state. The related
powerspectra $t_d = 0$ are a bit distorted from the Gaussian blur but the
initial slope of the inertial range is very near the expectation marked by the
solid black line on top of the plot. As described in detail in
\sec{stirturb-pws} the slopes flatten with increasing time since the energy
cascade slowly breaks down with progressing the decay.

\image{1.0}{plots/decayturb/pws-vw-vels-dyntimes.png}{Powerspectra of the
volume-weighted velocity field for three stages of decay: $t_d = 0$, $t_d =
3.0$ and $t_d = 6.0$.  Since the total kinetic energy declines over time the
powerspectra shift downwards. The strange looking spectrum from $t_d = 0$ is
the result of the applied Gaussian blur on the initial state.}

\image{1.0}{plots/decayturb/pws-mw-vels-dyntimes.png}{Powerspectra of the
mass-weighted velocity field for for three stages of decay: $t_d = 0$, $t_d =
3.0$ and $t_d = 6.0$.  Since the total kinetic energy declines over time the
powerspectra shift downwards. The strange looking spectrum from $t_d = 0$ is
the result of the applied Gaussian blur on the initial state.}

% \image{1.0}{plots/decayturb/evol-slope-vw-velocity}{Time evolution of the
% inertial range's slope of the volume-weighted velocity powerspectra. Examples
% of them are presented in \fig{plots/stirturb/powerspectra/rmsv-vw-dyntimes.png}.}
% 
% \image{1.0}{plots/decayturb/evol-slope-mw-velocity}{Time evolution of the
% inertial range's slope of the mass-weighted velocity powerspectra. Examples
% of them are presented in \fig{plots/stirturb/powerspectra/rmsv-mw-dyntimes.png}.}

\subsubsection{Summary}
The decaying turbulence setup was conducted in analogy with the driven
setup under \sec{stirturb}. A fully developed Mach 10 turbulence was
let to decay by FLASH's and FLEXI's solvers. In a comparitive study
we evaluated colummn density/velocity snapshots at three different
stages of decay and underpinned the observations with density
distributions and powerspectra.

All schemes provided very similar solutions with minor differences on small
scales. See \fig{plots/decayturb/decay-01-01.png} to
\fig{plots/decayturb/decay-03-03.png}.  A variance in energy dissipation rates
(cf. \fig{plots/decayturb/evol-ekin-diss}) was not quantifiable.  In accordance
with the results of the decay phase of the driven turbulence (cf.
\sec{stirturb-pdf}) the relation between density PDF and sonic Mach number (cf.
\sec{theory-density-pdf}) systematically underestimates the correct
root-mean-square Mach number (cf.
\fig{plots/decayturb/evol-sonic-mach-pdf.png}). Apparently, the relation does
not hold for decaying turbulences. The accurate modelling of the energy cascade
was confirmed by the velocity powerspectra in
\fig{plots/decayturb/pws-vw-vels-dyntimes.png} and
\fig{plots/decayturb/pws-mw-vels-dyntimes.png}.

There is no clear winner who resolves small scale structures best.  On the
hand FLASH's solver contain more energy in the dissipative range of the kinetic
energy powerspectra (cf. \fig{plots/decayturb/pws-ekin-dyntimes}) throughout
the simulation. But on the other hand the solver from FLEXI show more amount of
mass on both ends of the density spectrum (cf.
\fig{plots/decayturb/pdf-vw-dens.png}). Though, one should not forget that
all plots are in log-log scale which means that subtle differenes get blown
up and pretend to be important.

The decaying turbulence setup was revealing in the sense that the influence of
turbulent forcing on dissipation rates is immense. Subsequent driven and decay
simulations with lower and higher Mach numbers come to the same conclusion.
