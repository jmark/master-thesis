\subsection{Driven Turbulence}

\subsubsection{Time Evolution}
\image{1.0}{plots/stirturb/evolutions/sonic-mach.png}{Time evolution of the
root-mean-square mach number. Additionally, the turbulent sonic mach number
evaluated from the width of the density PDF is shown for comparison.}
\image{1.0}{plots/stirturb/evolutions/timestep.png}{Time evolution of the timestep $dt$.
The CFL Numbers are listed in \tbl{tab:decayturb-cfl}.}
\begin{table}[H]
\fontsize{3mm}{3mm}\selectfont
\captionsetup{width=.8\textwidth}
\caption{CFL-Numbers}
\centering
\begin{tabular}{l| cccc cccc}
\toprule
Solver & \bouT & \bouF & \fppm & \eufv & \mpfv & \mphy & \rkfv & \rkhy \\
\midrule
CFL    & 0.8 & 0.8 & 0.8 & 0.4 & 0.8 & 0.6 & 0.9 & 1.2 \\
\bottomrule
\end{tabular}
\label{tab:decayturb-cfl}
\end{table}
\image{1.0}{plots/stirturb/evolutions/ener.png}{Time evolution of the energy
in the system. The total energy $\ener$ is the sum of the kinetic $\ekin$ and
internal energy $\eint$: $\ener = \eint + \ekin$.}

\image{1.0}{plots/stirturb/evolutions/eint.png}{Time evolution of the internal
energy $\eint$ under polytropic cooling. The quantifiable deviation of the
internal energy from the anticipated constant $\eint_{0} = 0.9$ is discussed in
the text.}

\image{1.0}{plots/stirturb/evolutions/ekin-dissipation.png}{Time evolution of
the kinetic energy dissipation $\ekindiss$.}

\image{1.0}{plots/stirturb/evolutions/ekin-dissipation-cumsum.png}{Cumulated dissipated
kinetic energy deprived out of the system by polytropic cooling.}

\image{1.0}{plots/stirturb/evolutions/enstrophy}{Time evolution of the
mean enstrophy $\enst$.}

\image{1.0}{plots/stirturb/evolutions/rmsv-max.png}{Maximum root-mean-square velocity
over time.}

\image{1.0}{plots/stirturb/evolutions/ratio-fv-dg.png}{Time evolution of the fraction of
Finite-Volume Elements to the total number of elements. \remark The other
solvers would stay at 100\% since they operate solely with Finite-Volumes.}

\subsubsection{Powerspectra}

\image{1.0}{plots/stirturb/powerspectra/rmsv-vw-dyntimes.png}{Powerspectra of
the volume-weighted velocity field for the turbulent phase $t_d = [2,4]$
(time-average) and at the end of the decaying phase $t_d = 6.0$.}

\image{1.0}{plots/stirturb/powerspectra/rmsv-mw-dyntimes.png}{Powerspectra of
the mass-weighted velocity field for the turbulent phase $t_d = [2,4]$
(time-average) and at the end of the decaying phase $t_d = 6.0$.}

\image{1.0}{plots/stirturb/powerspectra/evol-slopes-vw-velocity.png}{Time evolution of the
inertial range's slope of the volume-weighted velocity powerspectra. Examples
are presented in \fig{plots/stirturb/powerspectra/rmsv-vw-dyntimes.png}.}

\image{1.0}{plots/stirturb/powerspectra/evol-slopes-mw-velocity.png}{Time evolution of the
inertial range's slope of the mass-weighted velocity powerspectra. Examples
are presented in \fig{plots/stirturb/powerspectra/rmsv-vw-dyntimes.png}.}

\image{1.0}{plots/stirturb/powerspectra/ekin-dyntimes.png}{Powerspectra of the
volume-weighted mean kinetic energy field shown for the turbulent phase $t_d = [2,4]$
(time-average) and two stages of the decaying phase: $t_d = 4.5$ and $t_d = 6.0$.}

\image{1.0}{plots/stirturb/powerspectra/evol-ekin-small-scales.png}{Amount of
small-scale energies prevailing in the dissipative length scales over time.}

\begin{table}[H]
\fontsize{3mm}{4mm}\selectfont
\captionsetup{width=.8\textwidth}
\caption{Comparison of the three types of powerspectra averaged over the turbulent phase: $t_d = [2,4]$}
\centering
\begin{tabular}{l|cc|cccc}
\toprule
Solver & Slope $m$ & Offset $n$ & Mean Squ. $\avg{\cdot^2}$ & Area $A$ & Diss. Area $A_{diss}$ & $A_{diss}/A$ [\%] \\
\midrule
\multicolumn{7}{c}{Volume-weighted Mean Kinetic Energy} \\
\midrule
\bouT & -1.16 $\pm$ 0.05 & 17.19 $\pm$ 0.24 & 39 $\pm$ 4 & 37 $\pm$ 8 & 2.00 $\pm$ 0.60 & 5.4 $\pm$ 1.7 \\
\bouF & -1.19 $\pm$ 0.10 & 17.22 $\pm$ 0.30 & 40 $\pm$ 5 & 37 $\pm$ 8 & 2.10 $\pm$ 0.70 & 5.7 $\pm$ 1.8 \\
\fppm & -1.14 $\pm$ 0.07 & 17.19 $\pm$ 0.24 & 38 $\pm$ 4 & 36 $\pm$ 8 & 1.37 $\pm$ 0.33 & 3.8 $\pm$ 0.9 \\
\eufv & -1.06 $\pm$ 0.04 & 17.11 $\pm$ 0.30 & 35 $\pm$ 5 & 34 $\pm$ 9 & 0.88 $\pm$ 0.35 & 2.6 $\pm$ 1.0 \\
\mpfv & -1.10 $\pm$ 0.05 & 17.18 $\pm$ 0.27 & 37 $\pm$ 6 & 35 $\pm$ 9 & 1.00 $\pm$ 0.50 & 3.0 $\pm$ 1.3 \\
\mphy & -1.06 $\pm$ 0.02 & 17.15 $\pm$ 0.17 & 36 $\pm$ 4 & 34 $\pm$ 7 & 1.12 $\pm$ 0.33 & 3.3 $\pm$ 1.0 \\
\rkfv & -1.07 $\pm$ 0.08 & 17.12 $\pm$ 0.31 & 33 $\pm$ 4 & 32 $\pm$ 8 & 0.75 $\pm$ 0.28 & 2.4 $\pm$ 0.9 \\
\rkhy & -1.17 $\pm$ 0.17 & 17.23 $\pm$ 0.34 & 33 $\pm$ 3 & 32 $\pm$ 6 & 0.81 $\pm$ 0.25 & 2.5 $\pm$ 0.8 \\
\midrule
\multicolumn{7}{c}{Volume-weighted Velocity} \\
\midrule
\bouT & -1.89 $\pm$ 0.13 & 16.67 $\pm$ 0.28 & 4.00 $\pm$ 0.40 & 4.0 $\pm$ 0.9 & 0.0152 $\pm$ 0.0034 & 0.38 $\pm$ 0.09 \\
\bouF & -1.87 $\pm$ 0.11 & 16.63 $\pm$ 0.23 & 3.86 $\pm$ 0.26 & 3.8 $\pm$ 0.7 & 0.0179 $\pm$ 0.0035 & 0.47 $\pm$ 0.09 \\
\fppm & -2.00 $\pm$ 0.15 & 16.76 $\pm$ 0.31 & 4.00 $\pm$ 0.40 & 4.0 $\pm$ 0.9 & 0.0143 $\pm$ 0.0029 & 0.36 $\pm$ 0.07 \\
\eufv & -2.08 $\pm$ 0.10 & 16.80 $\pm$ 0.15 & 3.86 $\pm$ 0.16 & 3.8 $\pm$ 0.6 & 0.0068 $\pm$ 0.0012 & 0.18 $\pm$ 0.03 \\
\mpfv & -2.09 $\pm$ 0.05 & 17.00 $\pm$ 0.70 & 6.00 $\pm$ 4.00 & 6.0 $\pm$ 0.5 & 0.0150 $\pm$ 0.0130 & 0.24 $\pm$ 0.22 \\
\mphy & -2.11 $\pm$ 0.03 & 16.94 $\pm$ 0.21 & 4.80 $\pm$ 0.90 & 4.8 $\pm$ 1.4 & 0.0102 $\pm$ 0.0030 & 0.21 $\pm$ 0.06 \\
\rkfv & -2.10 $\pm$ 0.08 & 16.87 $\pm$ 0.16 & 3.96 $\pm$ 0.17 & 3.9 $\pm$ 0.6 & 0.0074 $\pm$ 0.0015 & 0.19 $\pm$ 0.04 \\
\rkhy & -2.08 $\pm$ 0.04 & 16.85 $\pm$ 0.12 & 3.79 $\pm$ 0.14 & 3.9 $\pm$ 0.6 & 0.0073 $\pm$ 0.0015 & 0.19 $\pm$ 0.04 \\
\midrule
\multicolumn{7}{c}{Mass-weighted Velocity} \\
\midrule
\bouT & -1.56 $\pm$ 0.10 & 16.45 $\pm$ 0.19 & 3.158 $\pm$ 0.021 & 3.1 $\pm$ 0.4 & 0.054 $\pm$ 0.010 & 1.72 $\pm$ 0.33 \\
\bouF & -1.57 $\pm$ 0.13 & 16.45 $\pm$ 0.22 & 3.158 $\pm$ 0.024 & 3.1 $\pm$ 0.4 & 0.060 $\pm$ 0.011 & 1.90 $\pm$ 0.40 \\
\fppm & -1.59 $\pm$ 0.07 & 16.49 $\pm$ 0.17 & 3.166 $\pm$ 0.020 & 3.1 $\pm$ 0.5 & 0.043 $\pm$ 0.007 & 1.38 $\pm$ 0.24 \\
\eufv & -1.61 $\pm$ 0.07 & 16.48 $\pm$ 0.15 & 3.134 $\pm$ 0.004 & 3.1 $\pm$ 0.4 & 0.021 $\pm$ 0.004 & 0.69 $\pm$ 0.13 \\
\mpfv & -1.60 $\pm$ 0.04 & 16.52 $\pm$ 0.15 & 3.128 $\pm$ 0.006 & 3.0 $\pm$ 0.4 & 0.028 $\pm$ 0.009 & 0.93 $\pm$ 0.29 \\
\mphy & -1.60 $\pm$ 0.03 & 16.55 $\pm$ 0.11 & 3.133 $\pm$ 0.006 & 3.2 $\pm$ 0.5 & 0.029 $\pm$ 0.005 & 0.92 $\pm$ 0.17 \\
\rkfv & -1.61 $\pm$ 0.05 & 16.52 $\pm$ 0.12 & 3.137 $\pm$ 0.011 & 3.2 $\pm$ 0.4 & 0.022 $\pm$ 0.004 & 0.70 $\pm$ 0.13 \\
\rkhy & -1.62 $\pm$ 0.12 & 16.54 $\pm$ 0.19 & 3.138 $\pm$ 0.008 & 3.3 $\pm$ 0.5 & 0.022 $\pm$ 0.004 & 0.68 $\pm$ 0.13 \\
\bottomrule
\end{tabular}
\label{tab:pws-rmsv-vw}
\end{table}

\subsubsection{Probability Distributions}
\image{1.0}{plots/stirturb/pdfs/vels-vw-dyntimes.png}{Volume-weighted velocity
PDFs for the turbulent phase $t_d = [2,4]$ (time-average) and two stages of the
decaying phase: $t_d = 4.7$ and $t_d = 6.0$.}
\image{1.0}{plots/stirturb/pdfs/dens-vw-dyntimes.png}{Volume-weighted density
PDFs for the turbulent phase $t_d = [2,4]$ (time-average) and two stages of the
decaying phase: $t_d = 4.7$ and $t_d = 6.0$.}
\image{1.0}{plots/stirturb/pdfs/evol-sonic-mach.png}{Time evolution of the
sonic mach number derived from the width of the volume-weighted density PDFs
over time.}

\begin{table}[H]
\fontsize{3mm}{4mm}\selectfont
\captionsetup{width=.8\textwidth}
\caption{Comparison of Gaussian Fitting Results of the Log-normal Density PDF
averaged over the turbulent phase: $t_d = [2,4]$}
\centering
\begin{tabular}{lcccc}
\toprule
Solver & Mean $s_0$ & Std. Deviation $\sigma_s$ & RMS Mach $\mach_{RMS}$ & PDF Mach $\mach_{PDF}$ \\
\midrule
\bouT & -0.31 $\pm$ 0.05 & 0.58 $\pm$ 0.06 & 2.63 $\pm$ 0.13 & 2.5 $\pm$ 0.4 \\
\bouF & -0.29 $\pm$ 0.06 & 0.57 $\pm$ 0.08 & 2.59 $\pm$ 0.09 & 2.4 $\pm$ 0.6 \\
\fppm & -0.32 $\pm$ 0.05 & 0.59 $\pm$ 0.06 & 2.62 $\pm$ 0.12 & 2.6 $\pm$ 0.5 \\
\eufv & -0.24 $\pm$ 0.03 & 0.47 $\pm$ 0.04 & 2.56 $\pm$ 0.04 & 1.8 $\pm$ 0.2 \\
\mpfv & -0.31 $\pm$ 0.04 & 0.60 $\pm$ 0.09 & 2.80 $\pm$ 0.40 & 2.7 $\pm$ 0.8 \\
\mphy & -0.32 $\pm$ 0.05 & 0.56 $\pm$ 0.07 & 2.81 $\pm$ 0.18 & 2.4 $\pm$ 0.5 \\
\rkfv & -0.26 $\pm$ 0.02 & 0.50 $\pm$ 0.03 & 2.61 $\pm$ 0.06 & 2.0 $\pm$ 0.2 \\
\rkhy & -0.26 $\pm$ 0.03 & 0.48 $\pm$ 0.04 & 2.55 $\pm$ 0.04 & 1.9 $\pm$ 0.3 \\
\bottomrule
\end{tabular}
\label{tab:pdf-dens-vw}
\end{table}

\subsubsection{Performance}

\image{1.0}{plots/stirturb/evolutions/runtime-cumsum.png}{Cumulative runtime of
the different runs. For example, the PPM solver needs two hours to get half-way
through the simulation at $t_d = 3.0$. \remark The contribution of input-output
operations and unintended interuptions are already substracted out.}

\image{1.0}{plots/stirturb/evolutions/runtime-relative.png}{Time derivative of
cumulative runtime (see \fig{plots/stirturb/evolutions/runtime-cumsum.png}).
For example, the run with the PPM solver would have lasted roughly 6 hours with
the rate at around $t_d = 3.0$: \\ 1 h $\cdot$ 6 = 6 h.}

\image{1.0}{plots/stirturb/runtime-vs-ekin-dissipation}{Relative Runtime vs Kinetic
Energy Dissipation Rate.}
