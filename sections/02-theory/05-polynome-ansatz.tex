\subsubsection{Discontinuous Galerkin Method}
\label{sec:polyonome-ansatz}

By refining the mesh the error in the numerical approximation of the physical
solution decays algebraically, that is, introducing more elements.
Traditionally, numerical steps in space or time are labeled with the letter
``h''. Hence, mesh refinement is also known as \emph{h-refinement}.


% More information can be found in Bernardo Cockburn George E. Karniadakis
% Chi-WangShu(Eds.) Discontinuous Galerkin Methods Theory, Computation and
% Applicationsi

% Source Spectral/hp Element Methods for CFD 1999

An alternative approach is to keep the number of subdomains fixed and increase
the order of the interpolating polynomials. This is called \emph{p-refinement}.
For infinitely smooth solutions p-refinement can lead to exponential convergent
rates!

DG methods have the following main advantages over finite volume methods:
\begin{itemize}
\item The order of accuracy only depends on the physical solution. By suitably
choosing the degree of the approximating polynomials they can obtain an
arbitrarily high order of accuracy.

\item They are highly parallelizable. Since the elements are discontinuous and
transformed to a reference element, the mass matrix (\eqn{mass-matrix}) becomes
block diagonal. The blocks must be inverted beforehand just once, since their
size is equal to the number of degrees of freedom. This massively improves
performance.

% \item DG methods are very well suited to handling complicated geometries
% and require an extremely simple treatment of the boundary conditions in
% order to achieve uniformly high-order accuracy.

% \item DG methods can easily handle adaptivity strategies since refinement or
% unrefinement of the grid can be achieved without taking into account
% the continuity restrictions typical of conforming finite element 
% methods. Moreover, the degree of the approximating polynomial can be easily
% changed from one element to the other. Adaptivity is of particular 
% importance in hyperbolic problems given the complexity of the structure of
% the discontinuities.

\item In essence, they provide fast convergence, small diffusion and dispersion
errors, better data volume-over-surface ratio for efficient parallel processing
and better input/output handling due to the smaller volume of data.
\end{itemize}

%% \image{0.6}{DG-work-over-periods.png}{Computational work (number of
%% floating-point operations) required to integrate a linear advection equation
%% for $M$ periods while maintaining a cumulative phase error of e = 10\%. Source:
%% \cite{},p.10}

It follows a brief derivation of the DG method.  At first the physical domain
$\Omega$ is divided into a \emph{mesh} of adjunct self-contained sub-domains
$\Omega_l$ ($l \in \mathbb{N}$) with well-defined boundaries. Every
element $\Omega_l$ gets transformed to the reference space $\hat{\Omega} =
[-1,1]^3$. For the sake of clarity, we assume a Cartesian grid, drop any
transformation prefactor and set $\Omega = \Omega_l = \hat{\Omega}$.

The unknown solution $\vec{U}$ is now replaced by polynomes of order $N_p$
constructed from linear combinations of orthogonal basis functions
$\vec{\Psi}^j$.
\begin{equation}
    U_{\vec{i}}(t,x,y,z) \approx p_{\vec{i}}(t,x,y,z) = \sum^{N_p}_{\vec{j}=0} U^{\vec{j}}_{\vec{i}}(t) \, \Psi^j(\vec{x}),
\end{equation}
where $\vec{i}$ and $\vec{j}$ are three-dimensional \emph{multi-indices}.
\begin{equation}
    \vec{i} = (i,j,k)^T, \;\; i,j,k \in \mathbb{N}_0
\end{equation}

The following treatment is analog for all five conservative variables of the
Euler equation, so we ignore the index $i$. Remembering the general weak
formulation, \eqn{weak-formulation}, of the solution integral,
\begin{align}
\label{eqn:weak-formulation-galerkin}
    &\int_\Omega \left(\sum^{N_p}_{j=0} (\partial_t U^j(t)) \, \Psi^j(\vec{x})\right)\,\phi(\vec{x}) d^3x 
        + \int_\Omega S(t) \, \phi(\vec{x}) d^3x = \nonumber \br
         &\int_{\partial\Omega} \left[F(t)\,\phi(\vec{x})\,n_x + G(t)\,\phi(\vec{x})\,n_z + H(t)\,\phi(\vec{x})\,n_z \right] d^2x, \br
        -&\int_\Omega \left[
              \left(\sum^{N_p}_{j=0} F^j(t)\,\Psi^j(\vec{x})\right)\,\partial_x \phi(\vec{x}) 
            + \left(\sum^{N_p}_{j=0} G^j(t)\,\Psi^j(\vec{x})\right)\,\partial_y \phi(\vec{x})
            + \left(\sum^{N_p}_{j=0} H^j(t)\,\Psi^j(\vec{x})\right)\,\partial_z \phi(\vec{x}) \right] d^3x, \nonumber
\end{align}
we put the time variable into the coefficients $S,F,G,H$ and the space
variables into the basis functions $\Psi$ and test functions $\phi$.

\paragraph{Lagrange Polynome}
If we associate the basis functions $\Psi^{\vec{J}} := L^{\vec{J}}$ and the
test functions $\phi := L^{\vec{I}}$ with \vip{Lagrange} polyonmes of equal
order $N_p$, we can formulate an interpolation and integration scheme
(\emph{collocation}) over the domain $\Omega$. The polynome in one dimension
reads as follows
\begin{equation}
l_j(x) = \prod^p_{k=0,k\neq j} \frac{x-x_k}{x_j-x_k}, \ j = 0,\dots,p,
\end{equation}
with the \vip{Kronecker} property $l_j(x_i) = \delta_{ij}$.
\Fig{lagrange-interpolation-1d.png} shows a third-order ($N_p = 3$)
Lagrange interpolation of a smooth function at four arbitrary
interpolation or \emph{anchor} nodes.
\image{0.8}{lagrange-interpolation-1d.png}{The one-dimensional Lagrange
polynome of third order ($N_p = 3$) needs four anchor nodes with their
associated values in order to approximate the exact function. Note the minor
deviation from the original. This is the interpolation error. 
Beyond the fringes the Lagrange polynome goes to infinity.}

%% Unfortunately, Lagrange polynomes go to
%% infinity going further away from the outer anchor nodes. This effect occurs in
%% the Gauss-Quadrature where the outermost anchor nodes do not lie on the
%% interval boundaries.

We get to three-dimensional formulation via the \emph{Tensor Product Ansatz}.
\newcommand{\LI}{L_{\vec{i}}(\vec{x})}
\begin{equation}
\label{eq-tensor-ansatz}
    \LI = L_{ijk}(x,y,z) = l_i(x) \cdot l_j(y) \cdot l_k(z)
\end{equation}
and write the three-dimensional interpolating polynome as:
\newcommand{\PI}{P^{\vec{i}}(t,\vec{x})}
\newcommand{\FI}{F_{\vec{i}}(t)}
\newcommand{\sumI}{\sum^{N_p}_{\vec{i}=0}}
\begin{equation}
\label{eq-tensor-ansatz}
    \PI = \sumI \FI \LI = \sum^{N_p}_{i,j,k = 0} f_{ijk}(t) \cdot l_i(x) \cdot l_j(y) \cdot l_k(z)
\end{equation}
Note that the time varying part of the polynome lives in the coefficients.
Keep this in mind.
\newcommand{\FJ}{F_{\vec{j}}(t)}
\newcommand{\GJ}{G_{\vec{j}}(t)}
\newcommand{\HJ}{H_{\vec{j}}(t)}
\newcommand{\STX}{S(t,\vec{x})}
\newcommand{\UJ}{ \dot{U}_{\vec{j}}(t)}
\newcommand{\LJ}{L_{\vec{j}}(\vec{x})}
\newcommand{\sumJ}{\sum^{N_p}_{\vec{j}=0}}
\newcommand{\nx}{n_x(\vec{x})}
\newcommand{\ny}{n_y(\vec{x})}
\newcommand{\nz}{n_z(\vec{x})}
Going back to \eqn{weak-formulation-galerkin} and replacing $\Psi^{\vec{j}}$
and $\phi$ with Lagrange polynomes $\LI$, we get an explicit ansatz, called
\emph{Galerkin polynome}.
\begin{align}
     &\int_\Omega \left(\sumJ \UJ \, \LJ \right) \, \LI d^3x + \int_\Omega \STX \, \LI d^3x = \nonumber \br
     &\int_{\partial\Omega} \bigg[ F(t) \, \nx + G(t) \, \ny + H(t)\, \nz \bigg]\,\LI\, d^2x \br
    -&\int_\Omega \left[
          \left(\sumJ \FJ \, \LJ \right) \, \partial_x \LI 
        + \left(\sumJ \HJ \, \LJ \right) \, \partial_y \LI
        + \left(\sumJ \GJ \, \LJ \right) \, \partial_z \LI \right] d^3x \nonumber
\end{align}

\newcommand{\sumK}{\sum^{N_p}_{\vec{k}=0}}
\newcommand{\wK}{\omega_{\vec{k}}}

\newcommand{\STXK}{S(t,\vec{x}_{\vec{k}})}
\newcommand{\LIK}{L_{\vec{i}}(\vec{x}_{\vec{k}})}
\newcommand{\LJK}{L_{\vec{j}}(\vec{x}_{\vec{k}})}

\newcommand{\LXIK}{L^{(x)}_{\vec{i}}(\vec{x}_{\vec{k}})}
\newcommand{\LYIK}{L^{(y)}_{\vec{i}}(\vec{x}_{\vec{k}})}
\newcommand{\LZIK}{L^{(z)}_{\vec{i}}(\vec{x}_{\vec{k}})}

Evaluating above formula at discrete nodes $\vec{x}_{\vec{i}}$ (cf.
\sec{trafo}) and introducing \emph{integration weigths}
\begin{equation}
\label{eqn:mass-matrix}
    \omega_{\vec{i}} = \int_{\Omega} \LI d^3 x
\end{equation}
we can discretize the continuous integrals.
\begin{align}
     &\sumK \left(\sumJ \UJ \, \LJK \right) \, \LIK \,\wK + \sumK \STXK \, \LIK \wK = \nonumber \br
     &\bigg[ F^*(t) + G^*(t) + H^*(t)\bigg] \, L_{\vec{j}}\bigg|_{\partial\Omega} \br
    -&\sumK \left[
          \left(\sumJ \FJ \, \LJK \right) \, \LXIK 
        + \left(\sumJ \HJ \, \LJK \right) \, \LYIK
        + \left(\sumJ \GJ \, \LJK \right) \, \LZIK \right] \wK \nonumber
\end{align}

The partial differential
$\partial_x$ turns into a discrete linear operator.
\newcommand{\DXKI}{D^{(x)}_{\vec{k}\vec{i}}}
\newcommand{\DYKI}{D^{(y)}_{\vec{k}\vec{i}}}
\newcommand{\DZKI}{D^{(z)}_{\vec{k}\vec{i}}}
\begin{equation}
    L^{(x)}_{\vec{j}}(\vec{x}) = \partial_x \LJ = (\partial_x l_{j_1}(x)) \cdot l_{j_2}(y) \cdot l_{j_3}(z)
\end{equation}
\begin{equation}
    D^{(x)}_{\vec{i}\vec{j}} = D^{(x)}_{i_1i_2i_3j_1j_2j_3} = l^{(x)}_{j_1}(x_{i_1})\cdot l_{j_2}(y_{i_2}) \cdot l_{j_3}(z_{i_3})
\end{equation}
The operators for the y- and z-dimension are constructed in analog manner.  The
surface term got replaced by flux functions $F^*$ (cf. \sec{flux-functions})
where the exchange of mass, momentum and energy across subdomain/element
boundaries takes place.

\paragraph{Weak Formulation}
Finally, we arrive at the semi-discrete \emph{weak formulation} of the
discontinuous Galerkin method which can be directly translated to computer
code.
\begin{align}
\label{eqn:semi-discrete-weak-formulation}
     &\sumK \left(\sumJ \UJ \, \LJK \right) \, \LIK \,\wK + \sumK \STXK \, \LIK \wK = \nonumber \br
     &\bigg[ F^*(t) + G^*(t) + H^*(t)\bigg]\, L_{\vec{j}}\bigg|_{\partial\Omega} \br
    -&\sumK \left[
          \left(\sumJ \FJ \, \LJK \right) \, \DXKI 
        + \left(\sumJ \HJ \, \LJK \right) \, \DYKI
        + \left(\sumJ \GJ \, \LJK \right) \, \DZKI \right] \wK \nonumber
\end{align}

\remark If the polynomial order is set to one ($N_p = 1$) the formulation reduces
to the first order finite volume method!
