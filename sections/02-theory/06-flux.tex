\subsubsection{Flux Functions}
\label{sec:flux-functions}
Finite Element methods must solve the Riemann problem (cf. \sec{theory-shocks})
at the element boundaries in order to exchange information.  The
\emph{Rankine–Hugoniot conditions} or \emph{shock jump conditions} require that
mass, momentum and energy must be conserved when crossing the boundaries. The
theoretical framework of the Rieamnn problem knows a procedure that provides an
exact solution to a given shock problem. Unfortunately, an exact Riemann solver
is in most cases hard to construct and even harder to compute due to bad
convergences rates.

Hence, a wide variety of approximate Riemann solvers or \emph{flux functions}
have been proposed that can be applied much more cheaply than the exact Riemann
solver and yet give results that in many cases are equally good when used in
high-resolution finite element methods. 

A detailed discussion of the theory of flux functions would go beyond the scope
of this thesis and has been done extensively elsewhere. For reasons given in
\sec{computational-frameworks}) we do/cannot not study the influence of flux
functions on the simulations separately. Hence, in this work, flux functions are
just considered an integral part of the numerical scheme.
