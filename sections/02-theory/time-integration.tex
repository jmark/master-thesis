\subsubsection{Time Integration}

At the end of \sec{polyonome-ansatz} in \eref{semi-discrete-weak-formulation} we arrived
at a semi-discrete weak formulation of the Euler equations. It resembles an ordinary
differential equation of the form
\begin{equation}
\frac{d}{dt}y = f(t,y).
\end{equation}
Defining initial values $y(t_0) = y_0$ this equation can be numerically solved
in the most naive way via the \vip{Euler} method. Chosing an appropiate timestep $\Delta t$
we can explicitly integrate from the current state $y^n$ to the future state $y^{n+1}$.
\begin{equation}
    y^{n+1} = y^{n} + \Delta t \cdot f(t^n,y^n)
\end{equation}
The $\Delta t$-convergence can be improved by introducing higher-order terms. A
widely used class of higher-order time integration schemes are the Runge-Kutta
methods (RK). The second-order RK, resp. \emph{midpoint} method, reads
\begin{equation}
    y^{n+1} = y^{n} + \Delta t \cdot f\left(t^n + \frac{\Delta t}{2},y^n + \frac{\Delta t}{2} \cdot f(t^n,y^n)\right)
\end{equation}

Next to the spatial resolution via the polyonme-ansatz, resp. h-p-convergence,
we will compare the influence of the time-integration in first, second and
third order.

\paragraph{Courant-Friedrichs-Lewy condition}
To keep a numerical algorithm stable the time step has to obey the
\emph{Courant-Friedrichs-Lewy condition} (CFL condition) which states that the
domain of dependence of $q_i^{n+1}$ of the algorithm at future time time
$t^{n+1}$ should include the true domain of dependence at time $t = t^n$ . Or in
other words: nothing is allowed to flow more than 1 grid spacing within one
time step. This means quantitatively
\begin{equation}
    \Delta t \le \frac{\Delta x}{\vels}
\end{equation}
Given a \emph{CFL} constant: $0 < C \le 1$
\begin{equation}
    \Delta t = C \cdot \min_{x}\left(\frac{\Delta x}{\vels(x)}\right)
\end{equation}
The CFL condition is a nessecary (but not sufficient) condition for the
stability of any explicit differencing method.

In a three-dimensional orthogonal domain the timestep reads
\begin{equation}
    \Delta t = C \cdot \min_{\vec{r}}\left(
        \frac{dx}{|v_x(\vec{r})|+c(\vec{r})},
        \frac{dy}{|v_y(\vec{r})|+c(\vec{r})},
        \frac{dz}{|v_z(\vec{r})|+c(\vec{r})}\right)
\end{equation}

\remark For supersonic regimes the sound speed $c$ can be neglected.
