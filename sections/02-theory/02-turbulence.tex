\subsection{Turbulence Statistics}
\label{sec:turbulence}

\emph{Turbulences} are common phenomena in nature. It names a flow regime in
fluid dynamics characterized by chaotic changes in pressure and flow velocity
and contrasts \emph{laminar} flow which occurs when a fluid flows in parallel
layers with no disruption between them.

In astrophysics turbulence is suspected to play a major role in star formation
within interstellar media (ISM). Cf. \cite{klessen2016physical}.  Hence, a
theoretical understanding of the underlying mechanics is crucial in order
correctly model turbulences in numerical simulations. While the problem for
incompressible media has been thoroughly studied in the past (\emph{Kolmogorov
scaling}), the additional dynamics introduced by compressibility are still an
active field of research.

It is important not to forget that real ISM turbulence is neither isothermal,
nor polytropic. The real ISM has a local temperature that is not a simple
function of density but results from the evolution of the thermal energy.
Gravity takes over control over the most dense regions. Magnetic fields lead to
effects like frozen filaments called \emph{Alfvén's frozen in theorem}. Cf.
\cite{roberts2007alfven}.

% A detailed discussion of turbulences can be found in:
% H K Versteeg and W Malalasekera: An Introduction to Computational Fluid Dynamics
% 
% AKlRA YOSHIZAWA: HydrodynaIllic and MagnetohydrodynaIllic Turbulent Flows Modelling and
% Statistical Theory

\paragraph{Reynolds Number}
An important dimensionless quantity in fluid mechanics used to help predict
flow patterns in different fluid flow situations is the \emph{Reynolds number}
$\reynold$. The Irish-British physicist \vip{Osborne Reynolds} (1842-1912)
discovered this fundemental relationship during his famous flow tube
experiments in the second half of the 19th century. It represents the ratio
between \emph{inertial} and \emph{viscous} forces and is calculated by
\begin{equation}
\label{eqn:reynold}
    \reynold = \frac{\text{inertial force}}{\text{viscous force}} = \frac{\dens_0\;\vels_0\;\clen}{\shearvisc_{vis}},
\end{equation}
where $\dens_0$, $\vels_0$, $\clen$, $\shearvisc_{visc}$ are the fluid density,
fluid velocity, characteristic length and dynamic viscosity, respectively.
Since the governing equations are defined to be the inviscid Euler equations
(cf. \sec{governing-equations}), in theory, there is no viscosity anywhere present.
\begin{equation}
\label{eqn:no-viscosity}
    \shearvisc_{visc} \equiv 0 \;\; \Longrightarrow \reynold = \infty
\end{equation}
This has far-reaching consequences because the Reynolds number predicts the
transition from \emph{laminar} to \emph{turbulent} flow. Laminar flow becomes
unstable when the inertial forces dominate over the viscous forces: $\reynold
\gg 1$. Consequently, the Euler equations lead to chaotic motion over all time and
length scales.

\paragraph{Reynolds Decomposition}
Turbulences are characterized by a random fluctuation of flow variables over time,
meassured at a fixed point in space. Their evolution can be described
with the \emph{Reynolds decomposition}.
\begin{equation}
    q(t) = \avg{q}_t + \tilde{q}(t), \ \ \avg{\tilde{q}} \equiv 0
\end{equation}
where 
\begin{equation}
    \avg{q}_t = \frac{\int_{t_0}^{t1} q(t) dt}{t_1 - t_0}
\end{equation}
is the time-average of the flow property $q(t)$.  Turbulent flows are globally
described in terms of their mean values of properties like density, velocity
and pressure. The velocity fluctuations are independent of the axis of
reference, i.e. invariant to translation, rotation and reflection; they are
\emph{isotropic}. \emph{Isotropic turbulence} is by its definition always
homogeneous. In such a situation, the gradient of the mean velocity does not
exists, the mean velocity is either zero or constant throughout.

As the Reynolds decomposition suggests it does not make sense to talk about a
turbulence at one specific point in time or space. Thus, we define important
average properties or \emph{mean values} which help to quantify the current
state of the turbulent system as a whole.

\paragraph{Root-mean-square Velocity}
is the mass-weighted average squared velocity of the system.
\begin{equation}
    \rmsv = \sqrt{\frac{\int_\Omega \dens \; \velv^2 \; d\Omega}{\int_\Omega \dens \; d\Omega}}
\end{equation}
We utilize mass-weighting since the conducted turbulence
simulations are highly compressible.

\paragraph{Sonic Mach Number}
is directly related to $\rmsv$ via
\begin{equation}
\mach = \frac{\rmsv}{\avg{\sndsped}}.
\end{equation}
From $\avg{c} = 1$, according to \eqn{parameter-choice-completion}, follows that
Mach number and root-mean-square velocity are equivalent and, if not stated
otherwise, used interchangeably.

\paragraph{Turning time}
represents a time span characterizing large scale turbulent motions. Under
\sec{governing-equations} a characteristic time scale $t_r =
\frac{l_r}{\vels_r}$ was introduced exhibiting the invariance of the Euler equations 
under unit transformations. Now it gets a practical meaning since we define a
\emph{turning time} that tells how long it takes for a turbulence vortex to
traverse the length $L$ of the physical domain.
\begin{equation}
\label{eqn:dyn-time}
    T_{turn} = \frac{L}{\vels_{rms}}
\end{equation}
Equivalent terms for the turning time are \emph{crossing time} or \emph{dynamic
time scale}.

\paragraph{Dynamic time}
is the scaled \emph{physical} time $t$.
\begin{equation}
    t_d = \frac{t}{T_{turn}}
\end{equation}
All time evolution plots presented in this thesis are scaled to dynamic
time which sets the time axis in relation to the analyzed turbulence. 

\paragraph{Bulk Motion}
is the integrated momentum over the whole domain.
\begin{equation}
\label{eqn:bulk-motion}
    \bulk = \int_{\Omega} \dens\,\velv \; d\Omega
\end{equation}
For turbulences in closed systems, say a periodic box
(three-dimensional torus), bulk motion distorts the measurement of
mean values like the sonic Mach number. In the worst case scenario the whole
fluid would coherently move in one direction pretending to be a turbulent flow
of a certain Mach number. Correction of bulk motion is a crucial subject
in turbulence simulations.

\paragraph{Total Kinetic Energy}
is important in the discussion of energy accounting, especially in closed
systems where the total energy $\ener = \eint + \ekin = \text{const}$ is
conserved.
\begin{equation}
\label{eqn:ekin}
    \ekin = \int_\Omega \frac{\dens}{2} \; \velv^2 \; d\Omega
\end{equation}

\paragraph{Total Kinetic Energy Dissipation Rate}
is the negated time derivative of the total kinetic energy.
\begin{equation}
\label{eqn:ekindisp}
    \ekindisp = -\frac{d\ekin}{dt} 
\end{equation}
It quantifies how fast kinetic energy gets transformed into internal
energy at a certain point in time. 
\begin{equation}
\label{eqn:ekindisp-zero}
    \frac{d\ener}{dt} = \frac{d\eint}{dt} + \frac{d\ekin}{dt} = 0 \;\;
        \Longleftrightarrow \;\; \frac{d\eint}{dt} = -\frac{d\ekin}{dt}
\end{equation}

\subsubsection{Energy Cascade \& Powerspectrum}
\label{sec:theory-powerspectra}
Real turbulences always have a three-dimensional character and they lead to
rotational flow structures, called turbulent eddies, with a wide range of
length and energy scales.

Fluid particles which were initially separated by a large distance can be
brought close together by eddying motions. Consequently, mass, heat and
momentum are very effectively exchanged. It is an established fact that this
property has a profound influence in birth of stars, solar systems and cosmic
structures.

\vip{Andrey Kolmogorov} (1903-1987) was a pioneering mathematician who provided
a statistical treatment of turbulent flows, the \emph{Kolmogorov picture}, with
methods from \emph{dimensional analysis}. Cf. \cite[p. 33ff]{yoshizawa2013hydrodynamic},
\cite[p. 26ff]{biskamp2003magnetohydrodynamic}.
The spatial wavenumber $k$ for an eddy of diameter $\lambda$ is defined as
\begin{equation}
\label{eqn:wave-number}
    k = \frac{2\pi}{\lambda}.
\end{equation}
Based on $k$ the eddies in a turbulent flow are arranged into a hierarchy or
more precisely, a \emph{spectrum}. The largest eddies interact and transfer
energy from the mean flow.  We call it the range of \emph{large scales}. Since
large eddies are of the same order of the characteristic length $L$ and
velocity scale $\rmsv$ the flow is inviscid and their dynamics are dominated by
inertial effects. At the \emph{inertial range} kinetic energy gets transferred
down to smaller eddies via \emph{vortex stretching}. An \emph{energy cascade}
from larger to smaller length scales takes shape. The so called
\emph{Kolmogorov microscales} (cf. \cite{kolmogorov1991dissipation}), denoting the smallest scales in the spectrum,
form the \emph{viscous} or \emph{dissipative sublayer range} where the energy
input from nonlinear interactions and the energy drain from viscous dissipation
are in exact balance. The Reynolds number is 1 (\eqn{reynold}) and the
laminar flow dominates.
\begin{align}
\label{eqn:kolmogorov-scales}
    \eta     &= \left(\frac{\visc^3}{\dens^3\,\varepsilon}\right)^{1/4} \;\;\;\; &\text{Kolmogorov length scale} \\[0.4cm]
    \tau     &= \left(\frac{\visc}{\dens\,\varepsilon}\right)^{1/2} \;\;\;\; &\text{Kolmogorov time scale} \\[0.4cm]
    \upsilon &= \left(\frac{\visc\,\varepsilon}{\dens}\right)^{1/4} \;\;\;\; &\text{Kolmogorov velocity scale}
\end{align}
where $\dens$ is the ambient density and $\varepsilon$ is the
kinetic energy dissipation rate defined in \eqn{ekindisp}.
\Fig{energy-cascade-schematic.png} gives an illustration of the energy
spectrum of a turbulent flow as just described.
\image{0.8}{energy-cascade-schematic.png}{Source: \cite[p.
8]{muller2003evolving}. Schematic of the Kolmogorov picture
of turbulence showing the spatial energy powerspectrum over $k$. The
mathematical definition is given by \eqn{pws-ekin}.  The physical meaning of
the marked ranges are detailed in the text.}

\paragraph{Numerical Viscosity} By the very definition of the Euler equations
there is no viscosity: $\visc \equiv 0$. This would mean turbulent flow all the
way down. The dissipation range is non existent. It is rather evident that no
physical
simulation, running on a machine with computing power and memory limits, is
able to model this flow. The profound consequence is that ISM simulations are,
no matter what, always underresolved.

Every numerical scheme introduces a so called \emph{numerical} viscosity
$\mu_{\text{n-visc}}$ that marks the very limit of length, time and velocity
scale this scheme is still capabable to resolve. The exact amount of
$\mu_{\text{n-visc}}$ is in general unknown but should be unique for every
numerical method. For grid based schemes it is possible to estimate a 
lower limit of the length scale $\eta$. The \emph{Nyquist–Shannon sampling
theorem} (cf. \cite{shannon1949communication}) states that:
\begin{quote}
If a function $x(t)$ contains no frequencies higher than $B$ Hertz, it is
completely determined by giving its ordinates at a series of points spaced
$1/(2B)$ seconds apart.
\end{quote}
Applying this theorem to a uniform grid with $N$ nodes, the smallest length
scale would then become $\eta = 2/N$. With \eqn{kolmogorov-scales} we arrive at
a relation between kinetic energy dissipation and numerical viscosity.
\begin{equation}
\label{eqn:relation-ekin-diss-numerical-visc}
    \mu_{\text{n-visc}} = (\eta^4 \; \varepsilon)^{1/3} 
        = 2^{4/3}\;\left(\frac{\varepsilon}{N^4}\right)^{1/3}
        \propto \varepsilon^{1/3}
\end{equation}
When the grid resolution $N$ is equal among different numerical schemes,
their capability of resolving smallest scales can be compared by looking at the
kinetic energy dissipation rate.

\remark The term \emph{dissipation} might mistakenly suggest a physical process
causing the loss in kinetic energy. That is not the case! \emph{Diffusion} is
more fitting. It sums the scattering effects that the coarseness of grids
imposes on small scale structures.

%see SCALING RELATIONS OF SUPERSONIC TURBULENCE IN STAR-FORMING MOLECULAR CLOUDS
%    - numerical validation of theory in above paper

\paragraph{Kinetic Energy Powerspectrum}
\newcommand{\pekin}{P_{\ekin}}

\Fig{energy-cascade-schematic.png} shows the schematic of the shell-averaged
kinetic energy spectrum $\overline{\pekin}$ of a fully developed three-dimensional
turbulence. The powerspectrum $\pekin$ is defined as
\begin{equation}
    \pekin = \hat{\ekin}(\vec{k}) \cdot \hat{\ekin}^{\dagger}(\vec{k}),
\end{equation}
where $\vec{k} = (k_1,k_2,k_3)^T$  is the spatial wave vector analog to
\eqn{wave-number}, $\hat{\ekin}(\vec{k})$ is the Fourier transformed kinetic
energy field $\ekin(\vec{x})$ and $\hat{\ekin}^{\dagger}(\vec{k})$ its complex
conjugate.  Taking the three-dimensional shell-average over $\pekin$
yields
\begin{equation}
\label{eqn:pws-ekin}
    \overline{\pekin} \; dk = 4\pi\,k^2 \; \hat{\ekin}(\vec{k}) \cdot \hat{\ekin}^{\dagger}(\vec{k}) \; dk
        = 4\pi\,k^2 \; \hat{\ekin} \cdot \hat{\ekin}^{\dagger} \; dk
.
\end{equation}
The pre-factor $4 \pi\,k^2$ is a contribution from the differential volume of a
thin sphere at radius $k = |\vec{k}| \ge 0$.  The area under
$\overline{\pekin}$
\begin{equation}
    A_{\ekin^2} = \frac{1}{V_{\Omega}} \; \int_0^\infty 4\pi\,k^2 \; \hat{\ekin} \cdot \hat{\ekin}^{\dagger} \; dk
\end{equation}
is equal to the total squared kinetic energy $\ekin^2$ of the system in
accordance with \vip{Parseval}'s theorem
\begin{equation}
    \int_{-\infty}^{\infty} |Y(x)|^2 \, dx = \frac{1}{2\pi} \; \int_{-\infty}^{\infty} |\hat{Y}(k)|^2 \,dk.
\end{equation}

\paragraph{Velocity Powerspectrum}
\label{sec:theory-powerspectrum}
\newcommand{\pvels}{P_{\vels}}
\newcommand{\pmvels}{P_{m\vels}}
It has been shown that the descent of the energy cascade among turbulences is
universal. Cf. \cite{federrath2013universality,she1994universal,boldyrev2002scaling,kitsionas2009algorithmic}.
By determining the slope of log-log scale velocity powerspectra one can
determine if the turbulence is modelled correctly.

For each velocity component we take the Fourier transform of the velocity field
$\velv = (\vels_1, \vels_2, \vels_3)^T$.
We denote these Fourier transforms as
$\hat{\velv} = (\hat{\vels}_1, \hat{\vels}_2, \hat{\vels}_3)^T$. Analog
to above we define the powerspectrum as
\begin{equation}
    \pvels = \frac{1}{2} \; \hat{\velv} \cdot \hat{\velv}^{\dagger}.
\end{equation}
Taking the three-dimensional shell-average we get
\begin{equation}
    \overline{\pvels}\,dk = 4 \pi\,k^2 \;\; \frac{1}{2} \; \hat{\velv} \cdot \hat{\velv}^{\dagger}\,dk,
\end{equation}
which we call the \emph{volume-weighted} velocity powerspectrum. By
mass-weighting the velocity beforehand, $\dens^{1/2}\,\velv$, we get the
\emph{mass-weighted} velocity powerspectrum.
\begin{equation}
    \overline{\pmvels}\,dk = 4 \pi\,k^2 \;\; \frac{1}{2} \; \widehat{(\dens^{1/2}\,\velv)} \cdot \widehat{(\dens^{1/2}\,\velv)}^{\dagger}\,dk
\end{equation}
According to \cite{kitsionas2009algorithmic} the slopes of the inertial range
for the volume-weighted and mass-weighted powerspectra should amount to
$m_{vw}$ = -19/9 and $m_{mw}$ = -5/3, respectively.

\subsubsection{Probability Distribution Functions}
\label{sec:theory-density-pdf}
Another tool to verify the properness of a turbulence simulation are density
and velocity distributions. According to various studies
\cite{federrath_klessen_schmidt_2008,hopkins_2013,1999intuconf218N,0004-637X-761-2-149},
a fully developed turbulence models yield a nearly log-normal density
distribution.
\begin{equation}
\label{eqn:density-pdf}
PDF_s(s) = \frac{1}{\sqrt{2\pi}\sigma_s} \exp\left[-\frac{(s-s_0)^2}{2\sigma_s^2}\right],
\end{equation}
where $s = \ln(\dens/\dens_0)$. The standard deviation $\sigma_s$ of
the distribution is related to the sonic Mach number $\mach$ of the
turbulence via (cf. \cite{federrath_klessen_schmidt_2008,0004-637X-761-2-149})
\begin{equation}
(2\,\sigma_s)^2 = \ln(1 + b^2\,\mach^2)
\end{equation}
Reording yields an explicit expression for the Mach number
\begin{equation}
\label{eqn:density-pdf-mach}
\mach_{PDF} = \frac{\sqrt{\exp((2\,\sigma_s)^2) - 1}}{b}
\end{equation}
The proportionality constant $b$ dependents on the ratio of compressible to
solenoidal forcing $\zeta \in [0,1]$ and the number of spatial
dimensions $D = 1,2,3$.
\begin{equation}
\label{eqn:density-pdf-mach-constant}
b = 1 + (D^{-1} - 1) \, \zeta
\end{equation}
