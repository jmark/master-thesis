\subsection{Finite Element Scheme}

Solving PDEs numerically means to discretize an original continuous problem.
Most approaches consist of four major generally indepentend parts.

\begin{description}
\item [Meshing] Divide the problem domain into adjunct self-contained
sub-domains, called elements or cells. Depending on the scheme and requirements
this step can happen periodically. Via \emph{Adaptive Mesh Refinement} small
scale phenomena within the simulation can be resolved where needed without
degrading the overall performance disproportionately.

\item [Reconstruction] Approximate the exact solution in every element by a
piecewise constant function or polynome of order $N_p$.

\item [Evolution] Based on the current set of variables the conservation
laws yield a new state which gets evolved one timestep into the future.

\item [Averaging/Propagation] Flux functions solve the \vip{Riemann} problem
and communicate the lately acquired state across boundaries and propagate the
new information throughout the cell.  \end{description}

The last three actions are commonly grouped together under the term \emph{REA
algorithm} which is typical for \vip{Godunov}-type methods.

Since following terms are mentioned on a regular basis througout this document
we give brief definitions.
\begin{description}
\item[Mesh] A mesh consists of either cells or elements. The mesh can be
structured or unstructured. It contains the necessary information where to find
cells/elements and what their (spatial) relationship to neighbors are.

\item[Grid] Regular/Irregular, grid spaces, array of points/nodes.

\item[Cell] The atomic container type of a grid. They contain the actual data
which can be a scalar, arrays of scalars, vectors, tensors, etc. What cells
distinguish from points is that they have an expanse. Hence, one must specificy
if the data is defined in the cell-center, at their corners or at their faces.

\item[Element] Elements are spatially extended objects like cells. However,
they group a list of points called \emph{nodes} on which the data is pinned on.
When an element interacts with the outside world it must extract the necessary
values from these nodes via polynomial interpolation.
\end{description}
