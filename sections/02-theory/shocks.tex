\subsubsection{Shocks}
Shocks, or in more technical terms singular compression waves, are escalating
highly localized spikes in density and pressure due to nonlinear dynamics
encoded in the Euler equation. When a shock wave is emerging the velocity
behind the wave front is higher than in front of it. The media gets highly
compressed untill an unphysical state is reached. The velocity characteristica
begin to cross. Nature solves this dilemma by introducing additional physics
like extreme heat radiation, explosions, bangs or detachement of media in case
of surface waves. Either way, it involves an entropy increase in the system.
A numerical solver has to capture this kind of physics in order to prevent
unphysical solutions.

Astrophysical flows often involve shock waves. Thus, it is important that
shocks are accurately described by the numerical code used for the simula-
simulation of the flow. The respective requirements on the code fall into two different
categories. In the first category of problems the structure of the shock front
itself is important, i.e., a high spatial resolution of the discontinuity is crucial.
Typical examples for this category of problems are hydrodynamic flows with
nuclear burning, where an insufficient spatial resolution can lead to quan-
quantitatively very inaccurate and in some cases to even qualitatively incorrect
solutions (see Sect. 4). In the second category of problems the time scales
of processes triggered by the shock wave are comparable or larger than the
hydrodynamic time scale. Then mainly an accurate description of the two
states on both sides of the shock is important, while the structure of the
discontinuity matters less.


Solving Riemann problem.

Due to the nonlinearity of the Euler equations 

\paragraph{Method of Characteristics}

The prototype for all ODEs of second order is the \vip{Burger}'s equation.
This is about the simplest model that includes the nonlinear and viscous
effects of fluid dynamics.

\begin{equation}
\label{eq:charact-x-t}
    q_t + q \, q_x = \epsilon \, q_{xx}
\end{equation}

We set $\epsilon = 0$.

The characteristics satisfy
\begin{equation}
    x'(t) = q(t,x(t))
\end{equation}
and along each characterstic $q$ is constant, since
\begin{align}
    \frac{d}{dt}q(t,x(t))   &= \partial_t q(t,x(t)) + \partial_x q(t,x(t)) x'(t) \\
                            &= q_t + q\,q_t \\
                            &= 0.
\end{align}
Moreover, since q is constant on each characteristic, the slope $x'(t)$ is constant
by \eref{charact-x-t} and so the characteristics are straight lines, determined by the initial
data.

\paragraph{Shock Formation}
For larger t the equation C.10) may not have a
unique solution. This happens when the characteristics cross, as will even-
eventually happen if qx(x,0) is negative at any point. At the time Tb where the
characteristics first cross, the function q(x, t) has an infinite slope — the wave
"breaks" and a shock forms. Beyond this point there is no classical solution of
the PDE, and the weak solution we wish to determine becomes discontinuous.

\begin{equation}
   \vels_{shock} = \frac{\vels_L + \vels_R}{2}
\end{equation}
is the shock speed, the speed at which the discontinuity travels.

When we set following initial condition:
\begin{equation}
    U(0,x) = 
    \begin{cases}
        1 \ \ \text{if} \ \ x < 0 \\
        0 \ \ \text{if} \ \ 0 \leq x < 1 \\
        2 \ \ \text{if} \ \ 1 \leq x < 2 \\
        0 \ \ \text{if} \ \ x > 2    
    \end{cases}
\end{equation}

\image{0.8}{burger-characteristics.png}{
Source: https://calculus7.org/2015/11/27/rarefaction-colliding-with-two-shocks/
}

The picture above shows multiple interesting features of shocks.

For times beyond the breaking time some of the characteristics have
crossed and so there are points x where there are three characteristics leading
back to t = 0. One can view the "solution" q at such a time as a triple-valued
function. However, the density of a gas cannot possibly be triple valued at a
point.


The equation C.6) is a model of C.7) valid only for small e and smooth q.
When it breaks down, we must return to C.7). If the
initial data is smooth and e very small, then before the wave begins to break
the eqxx term is negligible compared to the other terms and the solutions to
the two PDEs look nearly identical. However, as the wave begins to break,
the second derivative term qxx grows much faster than qx, and at some point
the eqxx term is comparable to the other terms and begins to play a role.
This term keeps the solution smooth for all time, preventing the breakdown
of solutions that occurs for the hyperbolic problem. This smooth solution has
a steep transition zone where the viscous term is important. As e —> 0 this
zone becomes sharper and approaches the discontinuous solution known as
a shock. It is this vanishing-viscosity solution that we hope to capture by
solving the inviscid equation.

Rarification wave is pulling both shocks together so the eventually collide

The speed of propagation can be determined by conservation. The relation
between the shock speed s and the states qi and qr is called the Rankine-
Hugoniot jump condition. For scalar problems:

\begin{equation}
\vels_{shock}\,(\vels_L - \vels_R) = f(\vels_L) - f(\vels_R)
\end{equation}

For systems of equations, qi - qr and f(qr) - f{qi) are both vectors while
s is still a scalar. Now we cannot always solve for s to make C.18) hold.
Instead, only certain jumps qi — qr are allowed, namely those for which the
vectors f(qi) — f{qr) and qi — qr are linearly dependent.
For a linear system with f(q) = Aq, C.18) gives

A(qi - qr) = s(qt - qr) , C.20)
i.e., qi — qr must be an eigenvector of the matrix A and s is the associated
eigenvalue. For a linear system, these eigenvalues are the characteristic speeds
of the system. Thus discontinuities can propagate only along characteristics,
just as for the scalar advection equation.

\paragraph{Entropy Increase}

The second law of thermodynamics requires that entropy must increase across a normal shock
wave.

\begin{equation}
    \Delta s = c_v \ln\left[\frac{\pres_2}{\pres_1}\left(\frac{\dens_1}{\dens_2}\right)\right]
\end{equation}

This discussion is important on defining Flux functions.

\paragraph{Sod Shock Tube Problem}

Some workds about sod shock. Derivation why in isothermal case the
contact disccontinuity disappears.
