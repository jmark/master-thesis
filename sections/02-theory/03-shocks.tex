\subsection{Supersonic Shocks}
\label{sec:theory-shocks}
Astrophysical flows often involve shock waves. Shocks, or in more technical
terms singular compression waves, are escalating highly localized spikes in
density and pressure due to nonlinear dynamics inherent to the Euler equation.
When a shock wave is emerging, the velocity behind the wave front is higher than
in front of it. The medium is highly compressed until an unphysical state is
reached. The velocity \emph{characteristics} begin to cross. Nature solves this
dilemma by introducing additional physics like extreme heat radiation,
explosions, bangs or detachement of fluid. Either way, it involves an increase
of entropy according to the second law of thermodynamics. Cf. \cite{Salas1996}. A
numerical solver has to capture this kind of physics in order to prevent
unphysical solutions. 

\paragraph{Method of Characteristics}
The prototype for second order, hence nonlinear, differential equations is the
\vip{Burger}'s equation.
\begin{equation}
\label{eqn:burger}
    \partial_t u + u \, \partial_x u = \epsilon \, \partial^2_x q_,
\end{equation}
where $u(x,t)$ represents the velocity at position $x$ at time $t$. \Eqn{burger}
becomes inviscid in the limit of $\epsilon \to 0$. The characteristic equations
are
\begin{equation}
\label{eqn:burger-characteristics}
    %\frac{dx}{dt} = u \;\;\;\text{and}\;\;\; \frac{du}{dt} = 0
    \frac{dx}{dt} = u \;\;\;\wedge\;\;\; \frac{du}{dt} = 0
\end{equation}
with their obvious solutions $x(t) = u\,t + C_1$ and $u(t) = C_2$. Since $C_2$
must be a function of $C_1$, we derive a general solution for \eqn{burger} in
the inviscid limit.
\begin{equation}
\label{eqn:burger-general-solution}
    u = C_2(C_1) \;\;\;\; \Longleftrightarrow \;\;\;\; u(x,t) = C_2(x - u\,t)
\end{equation}
Now, if we set the initial condition to $u(x,0) = 1 - \cos(x)$ and let it
evolve, we get unphysical multi-valued solutions at some point in time. See
\fig{plots/burgers/solution.png}. The reason for this behaviour is the
intersection of characteristics (cf. \fig{plots/burgers/characteristics.png}) due
to the nonlinearity of the inviscid Burgers' equation.
\image{0.8}{plots/burgers/solution.png}{Source: \cite{LeonBurgers}. Multiple
snapshots at successive timesteps of the initial condition $u(x,0) = 1 -
\cos(x)$ evolving under the inviscid Burger's equation (\eqn{burger}). It
eventually develops a shock and becomes multi-valued, which is unphysical.}
\image{0.8}{plots/burgers/characteristics.png}{Source: \cite{LeonBurgers}.
Intersecting characteristics (\eqn{burger-characteristics}) of the inviscid
Burger's equation (cf. \eqn{burger}) are symptoms of a shock wave. The left area
where lines diverge from each other is called \emph{rarefaction} fan.
Converging lines are part of the \emph{compression} fan. The region of
intersection marks the shock discontinuity, where the solution (cf.
\fig{plots/burgers/solution.png}) becomes multi-valued.
% source: http://www.eng.famu.fsu.edu/~dommelen/pdes/style\_a/burgers
}
\paragraph{Riemann Problem}
The correct modelling of shocks is summarized under the term \emph{Riemann
problem}. It provides the theoretical basis for the correct treatment of
discontinuities in solutions of nonlinear PDEs. Furthermore, the Riemann
problem is an integral part of finite element schemes who approximate the
exact solution by piecewise constant or polynomial functions as
Finite-Volume and discontinuous Galerkin methods, respectively, do. More
on this can be found in \sec{flux-functions}.

\paragraph{Gibb's Phenomenon}
An inherent downside of schemes involving polynomes is the \vip{Gibb's}
phenomenon. It states that polyonomes of higher order, trying to approximate
discontinuities, yield spurious oscillations. See \fig{plots/burgers/ringing.png}.
There are many approaches to get the ringing near shocks under control. Two
of them, \emph{Artificial Viscosity} and \emph{FV-DG mode switching}, are presented
in \sec{shock-capturing}.
\image{0.7}{plots/burgers/ringing.png}{Examples of spurious oscillation of the
discontinuous Galerkin method solving the one-dimensional advection equations
with a discontinuous initial condition. The domain is divided into 50 elements each
containing a third-order polymome. Gauss and Gauss-Lobatto are two types of
distributing interpolation nodes within each element (cf. \sec{trafo}).}
