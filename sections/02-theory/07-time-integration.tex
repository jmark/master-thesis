\subsubsection{Time Integration}
\label{sec:time-integration}

At the end of \sec{polyonome-ansatz}, in \eqn{semi-discrete-weak-formulation}, we
arrived at a semi-discrete weak formulation of the DG method. It is nothing
more than an ordinary differential equation of the form
\begin{equation}
\frac{d}{dt}y = f(t,y).
\end{equation}
Defining initial values $y(t_0) = y_0$, this equation can be numerically solved
in the most naive way via the \emph{explicit} \vip{Euler} (EU) method. Chosing
an appropiate timestep $\Delta t$ we can explicitly integrate from the current
state $y^n$ to the future state $y^{n+1}$.
\begin{equation}
    y^{n+1} = y^{n} + \Delta t \cdot f(t^n,y^n)
\end{equation}
However, if the $\Delta t$-convergence rate is worse than the h-p-refinement
(cf. \sec{polyonome-ansatz}) it would render the advantages of the Galerkin
method useless. A widely used class of higher-order time integration schemes
with good convergence properties are the explicit Runge-Kutta methods (RK). The
second-order RK, resp. \emph{midpoint} method (MP), reads
\begin{equation}
    y^{n+1} = y^{n} + \Delta t \cdot f\left(t^n + \frac{\Delta t}{2},y^n + \frac{\Delta t}{2} \cdot f(t^n,y^n)\right),
\end{equation}
evaluating the integrand two times per timestep. Introducing again another timestep,
the integration becomes of third order (RK3):
\begin{equation}
    y^{n+1} = y^{n} + \frac{1}{6}\;(k_1 + 4\,k_2 + k_3),
\end{equation}
where 
\begin{align}
    k_1 &= \Delta t \cdot f(t^n,y^n) \\[0.4cm]
    k_2 &= \Delta t \cdot f\left(t^n + \frac{\Delta t}{2},y^n + \frac{\Delta t}{2} \cdot f(t^n,y^n)\right) \\[0.4cm]
    k_3 &= \Delta t \cdot f\left(t^n + \Delta t,y^n - k_1 + 2\,k_2\right).
\end{align}

\paragraph{Courant-Friedrichs-Lewy condition}
To keep a numerical algorithm stable the time step has to obey, the
\emph{Courant-Friedrichs-Lewy condition} (CFL condition). It requires that the
domain of dependence of $q_i^{n+1}$ of the algorithm at future time time
$t^{n+1}$ should include the true domain of dependence at time $t = t^n$ . Or in
other words: nothing is allowed to flow more than one grid spacing $\Delta x$
within one time step. This means quantitatively
\begin{equation}
    \Delta t \le \frac{\Delta x}{\vels}
\end{equation}
Given a CFL number: $0 < CFL \le 1$
\begin{equation}
    \Delta t = CFL \cdot \min_{x}\left(\frac{\Delta x}{\vels(x)}\right)
\end{equation}
The CFL condition is a nessecary, but not sufficient, condition for the
stability of any explicit differencing method. With increasing order of the
Runge-Kutta scheme the stability increases though. RK3 even allows the CFL
number, in some cases, to be greater 1.

In a three-dimensional orthogonal domain the timestep reads
\begin{equation}
    \Delta t = C \cdot \min_{\vec{r}}\left(
        \frac{dx}{|v_x(\vec{r})|+c(\vec{r})},
        \frac{dy}{|v_y(\vec{r})|+c(\vec{r})},
        \frac{dz}{|v_z(\vec{r})|+c(\vec{r})}\right)
\end{equation}

For supersonic regimes the sound speed $c$ can be neglected.
