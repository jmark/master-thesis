\subsection{Governing equations}
\label{sec:governing-equations}

In astrophysics, the \emph{interstellar medium} (ISM) is the matter that exists
in the space between the star systems in a galaxy. It contains gas in ionic,
atomic, and molecular form, as well as dust and cosmic rays. ISM fills
interstellar space and blends smoothly into the surrounding intergalactic
space. The medium is composed primarily of atomic hydrogen followed by helium
with traces of carbon, oxygen, nitrogen and metals. Magnetic fields and
turbulent motions also provide pressure in the ISM, and are typically more
important dynamically than the thermal pressure.

Many theoretical models for ISM are basesd on the ideal
\emph{Magneto-Hydrodynamic equations} which are a blend of the \emph{compressible
Euler equations} and of the \emph{Maxwell equations}, describing the hydrodynamics
and (electro-)magnetodynamics, respectively.

In this thesis we solely focus on the hydrodynamics.

\subsubsection{Compressible Euler Equations}

In 1757 \vip{Leonhard Euler} (1707-1783) published a set of equations for
inviscid flow, known as the Euler equations. They are hyperbolic
conservation equations which model perfect fluids without any interaction of their
constituents.  Hence, we assume no
heat conduction ($\SET^{i0} = \SET^{0i} := 0, \;\;i = \{1,2,3\}$), no viscosity
($\SET^{ij} := p\,\mathbb{I}, \;\;i,j = \{1,2,3\}$) and no gravity $g = 0$. 
Within the comoving frame the
\emph{stress-energy tensor} $\SET \in \mathbb{R}^{4x4}$ then reads:
\begin{equation}
\SET^{\alpha\beta} = \text{diag}(\dens\sndsped^2,\pres,\pres,\pres) = \left(\dens + \frac{\pres}{\sndsped^2}\right) \vels^\alpha\vels^\beta + p\,\mathbb{G}^{\alpha\beta} \;\;\; \text{with}  \;\;\; \alpha, \beta \in \{0,1,2,3\},
\end{equation}
where $\rho$, $\pres$, $\vels$, $\sndsped$ and $\mathbb{G}$ are the
density, pressure, velocity, speed of sound and the \emph{metric tensor}.
In flat spacetime the metric tensor is set to $\mathbb{G} =
\text{diag}(-1,1,1,1)$. The total energy $\ener$ and the number of particles $n$ are
conserved.
\begin{align}
\partial_\beta \SET^{\alpha\beta}  &= 0\\
\partial_\alpha (n\,\vels^\beta) &= 0
\end{align}
Taking the non-relativistic limit, we arrive at the conservative differential
form of the Euler equations.
\begin{align}
\label{eqn:comp-euler-conservative}
\partial_t\dens + \nabla \cdot (\dens\,\velv)   &=  0 \;\; &\text{\emph{mass conservation}}\\[0.4cm]
\partial_t(\dens\,\velv) + \nabla\cdot(\dens\velv\velv^T) + \nabla \pres &= \vec{F}  \;\;&\text{\emph{momentum conservation}}\\[0.4cm]
\partial_t \Ener + \nabla \cdot (\velv\,(\ener + \pres)) &=  0,\;\;&\text{\emph{energy conservation}}
\end{align}

where $\partial_t$ is the partial derivative in time, $\nabla =
(\partial_x,\partial_y,\partial_z)^T$ is the \emph{Nabla-operator} and
$\velv = (\vels^x,\vels^y,\vels^z)^T$ is the velocity vector. The total energy $\Ener$ is
the sum of the internal energy $\Eint$ and the kinetic energy $\Ekin$.
\begin{equation}
\Ener = \Eint + \Ekin = \frac{\pres}{\gamma -1} + \frac{\dens}{2}\vels^2,
\end{equation}
with $\adiabconst$ being the \emph{adiabatic constant}.
The source term $\vec{F}$ (F for forcing) allows us to perpetually inject a
force field which gets important in the discussion of driven turbulence later
on. See \sec{turbuforcing}.

% \paragraph{Primitive and Consverative Variables}

\paragraph{Equation of State}
All simulations follow the \emph{ideal gas law}.
\begin{equation}
\label{eqn:ideal-gas-law}
\pres = \frac{\sndsped^2}{\gamma} \, \dens = R\,T\,\dens = \frac{R}{c_v} \Eint = (\gamma - 1) \Eint,
\end{equation}
where $R$ is the specific ideal gas constant, $T$ is the gas temperature and $c_v
= \frac{\gamma - 1}{R}$ is the specific heat capacity at constant volume. 
The  $\adiabconst$ is set to
\begin{equation}
\label{eqn:adiabconst}
    \adiabconst = \frac{\heatcapP}{\heatcapV} := \frac{5}{3},
\end{equation}
which represents a mono-atomic gas without interacting forces. $c_p$ is the
specific heat capacity at constant pressure.  During the numerical simulation
the equation of state is enforced via the \emph{polytropic process} (also
called \emph{polytropic cooling}) at
every timestep.
\begin{equation}
\label{eqn:polytropic-relation}
\pres = C_P \, \dens^\Gamma \;\;\;(C_P = \text{const}),
\end{equation}
where the \emph{polytropic exponent} is set to $\Gamma := 1$ which is
equivalent to an isothermal process. A thorough derivation can be found in
\cite[p. 2 ff.]{horedt2004polytropes}. 
The speed of sound $\sndsped$ is constant due to the polytropic
relation. Comparing \eqn{ideal-gas-law} with \eqn{polytropic-relation} we get
the squared \emph{isothermal} speed of sound.
\begin{equation}
    C_P = c^2 = \gamma \frac{\pres}{\dens} = \text{const.}
\end{equation}

% \paragraph{Prandtl Number}
% \begin{equation}
% \label{eqn:prandtl}
%     \prandtl = \frac{\shearvisc \; \heatcapP}{\heatconduct}
% \end{equation}

\paragraph{Dimensionless Euler Equations} We want to show that the Euler
equations are invariant to changes of units. This discussion is useful since
most numerical frameworks do not support physical units and rescaled phyiscal
quantities avoid truncation errors due to the limits of floating point
operations. For this, we choose a characteristic length $l_r$, a characteristic
velocity $\vels_r$ and a characterstic density $\dens_r$. Multiplying suitable 
combinations of these constants with the Euler equations yields
\begin{align}
\left[\partial_t\dens + \nabla \cdot (\dens\,\velv)\right] \cdot \frac{l_r}{\dens_r\,\vels_r} &=  0\\[0.4cm]
\left[\partial_t(\dens\,\velv) + \nabla\cdot(\dens\velv\velv^T) + \nabla \pres - \vec{F}\right] \cdot \frac{l_r}{\dens_r\,\vels_r^2} &= 0\\[0.4cm]
\left[\partial_t \Ener + \nabla \cdot (\velv\,(\ener + \pres))\right] \cdot \frac{l_r}{\dens_r\,\vels_r^3} &=  0
\end{align}
We simplify and get
\begin{align}
\partial_{\tilde{t}} \tilde{\dens} + \tilde{\nabla} \cdot (\tilde{\dens}\,\tilde{\velv})   &=  0\\[0.4cm]
\partial_{\tilde{t}}(\tilde{\dens} \,\tilde{\velv}) +      \tilde{\nabla}\cdot(\tilde{\dens}\tilde{\velv}\tilde{\velv}^T) + \tilde{\nabla} \tilde{\pres} - \tilde{\vec{F}}&= 0\\[0.4cm]
\partial_{\tilde{t}} \tilde{\Ener} + \tilde{\nabla} \cdot (\tilde{\velv}\,(\tilde{\ener} + \tilde{\pres})) &=  0,
\end{align}
where $t_r = \frac{l_r}{\vels_r}$ (characteristic time) and
\begin{equation}
\tilde{t} = \frac{t}{t_r},\ 
\tilde{\dens} = \frac{\dens}{\dens_r},\
\tilde{\velv} = \frac{\velv}{\vels_r},\ 
\tilde{\nabla} = l_r\,\nabla,\ 
\tilde{\Ener} = \frac{\Ener}{\dens_r\,\vels_r^2},\ 
\tilde{\pres} = \frac{\pres}{\dens_r\,\vels_r^2},\ 
\tilde{\vec{F}} = \vec{F} \, \frac{l_r}{\dens_r\,\vels_r^2}.
\end{equation}
Conseqently, the dimensionless Euler equations do not change under unit transformation. If not
stated otherwise we drop the tilde sign ($\tilde{\cdot}$) and assume always
dimensionless quantites from now on.

\paragraph{Choice of parameters} One consequence of dimensionless units is the
free choice of parameters. We exploit this feature in order to choose a
sensible set of parameters. Considering the Euler equations in conservative
form (\eqn{comp-euler-conservative}), their functions in space and time
\begin{equation}
\dens = \dens(t,x,y,z), \;\;(\dens\velv) = (\dens\velv)(t,x,y,z),\;\;\Ener = \Ener(t,x,y,z)
\end{equation}
are completed with
\begin{equation}
\label{eqn:parameter-choice-completion}
    R := 1, \langle\dens\rangle := 1,\langle\sndsped\rangle = c := 1,
\end{equation}
From that we derive
\begin{equation}
\langle\pres\rangle = \frac{\sndsped^2}{\gamma}\cdot\langle\dens\rangle = 0.6,\;\;
\langle\eint\rangle = \frac{\langle\pres\rangle}{\gamma -1} = 0.9,\;\;
\langle\temp\rangle = \frac{\avg{\sndsped}^2}{\gamma\,R} = 0.6,
\end{equation}
where $\avg{\cdot}$ is the \emph{volume-weighted average} or \emph{mean value}
over the domain $\Omega$.
\begin{equation}
\label{eqn:domain-average}
    \avg{q} = \frac{\int_\Omega\,q\,d\Omega}{\int_\Omega\,d\Omega}
\end{equation}
This set of parameters define the global state at all times.

\subsubsection{Weak Formulation}
A natural way to define a generalized solution of the Euler equations that does
not require differentiability is going back to the integral form of the
conservation law.

The basic idea is to take the PDE, multiply it by a smooth \emph{test
function}, integrate one or more times over some domain, and then use
integration-by-parts to move derivatives off the function $q$ and onto some
smooth test function $\phi$. The result is an equation involving fewer
derivatives on $q$, and hence requiring less smoothness.

In this section we want to derive the \emph{weak formulation} of the governing
equations. This establishes the basis for the polynomial formulation which is
the core idea of all DG methods.  First, the Euler equations get split up into
terms resembling the independent one temporal and three spatial
dimensions with respect to the linear differential operator.
\begin{equation}
\partial_t \vec{U} + \partial_x\vec{F}(\vec{U})+ \partial_y\vec{G}(\vec{U}) + \partial_z\vec{H}(\vec{U}) = \vec{S},
\end{equation}
where
\begin{align}
    \vec{U} &= (\dens, \dens\vels_1, \dens\vels_2, \dens\vels_3, \Ener)^T\br
    \vec{F}(\vec{U}) &= (\dens\vels_1, \dens\vels_1^2 + \pres, \dens\vels_1\vels_2, \dens\vels_1\vels_3, \vels_1(\Ener + \pres))^T\br
    \vec{G}(\vec{U}) &= (\dens\vels_2, \dens\vels_2\vels_1, \dens\vels_2^2 + \pres, \dens\vels_2\vels_3, \vels_2(\Ener + \pres))^T\br
    \vec{H}(\vec{U}) &= (\dens\vels_3, \dens\vels_3\vels_1, \dens\vels_3\vels_2, \dens\vels_3^2 + \pres, \vels_3(\Ener + \pres))^T\br
    \vec{S} &= (0,f_1,f_2,f_z,0)^T
\end{align}
Defining a vector-valued test function $\vec{\phi} = (0,..,0,\phi_i,0,...,0)^T$
($i\in{1,...,5}$), multiplying component-wise with above equation and
integrating over the domain $\Omega$, we get
\begin{equation}
    \int_\Omega \left( \partial_tU_i\,\phi^i 
        + \partial_x F_i(\vec{U})\,\phi^i 
        + \partial_y G_i(\vec{U})\,\phi^i 
        + \partial_z H_i(\vec{U})\,\phi^i + S_i \phi^i \right) d^3x = 0
\end{equation}
Integration-by-parts rearranges the integral into a \emph{source term},
\emph{surface term} and \emph{volume term}.
\begin{align}
\label{eqn:weak-formulation}
    \int_\Omega \partial_t U_i\,\phi^i d^3x = \;\;\;\;\;\;\;&\int_\Omega S_i \phi^i d^3x \br
    +& \int_{\partial\Omega} \left(F_i(\vec{U})\,\phi^i\,n_x + G_i(\vec{U})\,\phi^i\,n_z + H_i(\vec{U})\,\phi^i\,n_z \right) d^2x,\nonumber \br
    -& \int_\Omega \left(F_i(\vec{U})\,\partial_x \phi^i + G_i(\vec{U})\,\partial_y \phi^i + H_i(\vec{U})\,\partial_z \phi^i \right) d^3x
\end{align}
where $\vec{n} = (n_x,n_y,n_z)^T$ is the outward surface normal to $\partial\Omega$.

Unfortunately, weak solutions are not unique, and so an additional problem is
to identify which weak solution is the physically correct vanishing-viscosity
solution. Again, one would like to avoid working with the viscous equation
directly, but it turns out that there are other conditions one can impose on
weak solutions that are easier to check and will also pick out the correct
solution. These are called \emph{entropy conditions} by analogy with the gas
dynamics case, where a discontinuity is physically realistic only if the
entropy of the gas increases as it crosses the shock. So called \emph{entropy-stable}
numerical solver take this into account (cf. \cite{derigs2016novel}).
