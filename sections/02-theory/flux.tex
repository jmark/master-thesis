\subsubsection{Flux Functions}
\label{sec:flux-functions}
Defining consistent, accurate and stable flux functions is a huge numerical
discipline of itself and an active field of research.

\paragraph{Riemann Problem}
Many of the methods are based on solving the \vip{Riemann} problem between the
states $q_L$ and $q_R$ in order to define the numerical flux $F^*(q_L,q_R)$.
For one way to approach this, it is useful to view the data $Q^n$ at time $t_n$
as defining a piecewise constant function $q^n(t_n,x)$ which has the value
$Q^n_i$ for all $x$ in the interval $C_i$. Suppose we could solve the
conservation law exactly over the time interval $[t_n, t_{n+1}]$ with initial
data  $q^n(t_n,x)$. We call the resulting function $q_n(t,x)$ for $t_n < t <
t_{n+1}$. Then we might consider defining the numerical flux $F^{*n}_i$ by
\begin{equation}
    F^{*n}_i = \frac{1}{k} \int_{t_n}^{t_{n+1}} f(q^n(t,x_i))\,dt
\end{equation}
This integral is trivial to compute, at least provided the time step $k$ is
small enough, because of the fact that with piecewise constant initial data we
can find the exact solution easily by simply piecing together the solutions to
each Riemann problem defined by the jump at each interface.

The method above is known as \vip{Godunov}'s method, which was already mentioned
at the beginning of the section, as an approach to solving the Euler equations
of gas dynamics in the presence of shock waves.

A wide variety of approximate Riemann solvers have been proposed
that can be applied much more cheaply than the exact Riemann solver and yet
give results that in many cases are equally good when used in the Godunov or
high-resolution methods. In brief, we will look at some possibilities.

\paragraph{All-Shock Solver}
In solving a nonlinear Riemann problem we must worry about whether the wave in
each family should be a shock or rarefaction so that we know whether to use the
Hugoniot locus or integral curve correspondingly. One simplification that can
be made to the Riemann solver is to ignore the possibility of rarefaction waves
and simply find a Riemann solution in which each pair of states is
connected along the Hugoniot locus. The solution then consists entirely of
discontinuities that satisfies the Rankine-Hugoniot conditions and is a weak
solution of the conservation law.  This approach is discussed by Colella [54]
for gas dynamics. The all-shock solver is particularly valuable for problems
where the Riemann problem is harder to solve, such as in problems where a more
complicated equation of state than a gamma-law gas must be used. This occurs at
high temperatures, or in relativistic flow, for example.

A potential problem with this approach, of course, is that by using a solution
to the Riemann problem that does not satisfy the entropy condicondition, we
might obtain a numerical solution which does not approximate the correct weak
solution. Actually, however, in most cases (except for transonic rarefactions),
Godunov's method with the Riemann solver will typically work well even if the
correct solution involves rarefaction waves. This is because of the numerical
dissipation that is introduced in every step of Godunov's method through the
averaging process. Although the discontinuous solution used in the solution of
a particular Riemann problem may not be correct, by averaging this solution
over the grid cell at the end of the time step the discondiscontinuity is
smeared out and after many time steps a good approximation to the correct
rarefaction wave will be computed. The all-shock approximation will typically
be inadequate in the case of a transonic rarefaction. Such a discontinuity
tends to persist and results in the computation of an entropy-violating weak
solution. An \emph{entropy fix} is often used to eliminate this problem.

\paragraph {HLLE (Harten-Luv-Lax Entropy-fix) flux}
Another approach is to approximate the full Riemann solution by a single
intermediate state bounded by two waves moving at speeds $s_1$ and $s_2$. The
wave speeds should be some approximations to the minimum and maximum wave
speeds that would arise from this particular Riemann data and the intermediate
state can then be calculated by the condition of conservation. This
approach is developed by Harten, Lax, and van Leer originally [108] and
improved by Einfeldt [78].

\paragraph {PPM Solver}

\paragraph {Bouchut5 Solver}


\paragraph{Higher-order Flux}
Regardless of what Riemann solver is used, Godunov's method will be at best
first-order accurate on smooth solutions and generally gives very smeared
approximations to shock waves or other discontinuities. The key is to use a
better representation of the solution, say piecewise linear instead of the
piecewise constant representation used in Godunov's method, but to form
this reconstruction carefully by paying attention to how the data behaves
nearby. In smooth regions the finite-difference approximation to the slope
can be used to obtain better accuracy, but near a discontinuity the \emph{slope}
computed by subtracting two values of $Q$ and dividing by $h$ may be huge and
meaningless. Using it blindly in a difference approximation will introduce
oscillations into the numerical solution.

\paragraph{Limiter Function}
Near a discontinuity we may want to limit this slope, using
a value that is smaller in magnitude in order to avoid oscillations. Methods
based on this idea are known as slope-limiter methods. This approach was
introduced by van Leer in a series of papers [246] through [248], where he
developed the MUSCL scheme for nonlinear conservation laws (Monotonic
Upstream-centered Scheme for Conservation Laws). The same idea in the
context of flux limiting, reducing the magnitude of the numerical flux to
avoid oscillations, was introduced in the flux-corrected transport (FCT)
algorithms of Boris and Book [37]. We can view this as creating a hybrid algo-
algorithm that is second order accurate in smooth regions but which reduces to a
more robust first-order algorithm near discontinuities. This idea of hybridiza-
hybridization was also used in early work of Harten and Zwas [110]. An enormous va-
variety of methods based on these principles have been developed in the past
two decades. One of the algorithms of this type that is best known in the
astrophysics community is the piecewise parabolic method (PPM) of
Woodward and Colella [59], which uses a piecewise quadratic reconstruction,
with appropriate limiting.
