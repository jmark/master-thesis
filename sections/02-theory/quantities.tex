\subsubsection{Global Quantities}

\paragraph{Turbulent Intensity}

\begin{equation}
    \mach_T = \frac{\sqrt{2/3 \, \Ekin}}{\sndsped}
\end{equation}

\paragraph{Mach Number $\mach$}

\begin{equation}
\label{eq:mach}
    \mach = \sqrt{\frac{\int_\Omega \dens \; \velv^2 \; d\Omega}{\int_\Omega \dens \; d\Omega}}
\end{equation}

\paragraph{Total Kinetic Energy $\ekin$}

\begin{equation}
\label{eq:ekin}
    \ekin = \frac{\int_\Omega \frac{\dens}{2} \; \velv^2 \; d\Omega}{\int_\Omega \dens \; d\Omega}
\end{equation}

\paragraph{Kinetic Energy Dissipation Rate $\ekindisp$}

\begin{equation}
\label{eq:ekindisp}
    \ekindisp = -\frac{d\ekin}{dt} 
\end{equation}

\begin{equation}
\label{eq:straintensor}
    (\strainT)_{ij} = \Partial{\vels_i}{x_j} + \Partial{\vels_j}{x_i} - \lambda \delta_{ij} \Partial{\vels_k}{x_k}
\end{equation}

Bulk viscosity coefficient $\lambda := 2/3$

\begin{equation}
\label{eq:ekindisp1}
    \ekindisp_1 = 2\,\frac{\shearvisc}{\dens_0\,\Omega} \; \int_\Omega \strainT : \strainT \, d\Omega
\end{equation}

\begin{equation}
\label{eq:ekindisp2}
    \ekindisp_2 = 2\,\frac{\bulkvisc}{\dens_0\,\Omega} \; \int_\Omega (\nabla \cdot \velv)^2 \, d\Omega
\end{equation}

\begin{equation}
\label{eq:ekindisp3}
    \ekindisp_3 = -\frac{1}{\dens_0\,\Omega} \; \int_\Omega \pres \, \nabla \cdot \velv \, d\Omega
\end{equation}

\paragraph{Total Enstrophy $\enst$}

\begin{equation}
\label{eq:vort}
    \vort = \nabla \times \velv
\end{equation}

\begin{equation}
\label{eq:enst}
    \enst = \frac{\int_\Omega \frac{\dens}{2} \; \vort^2 d\Omega}{\int_\Omega \dens \; d\Omega}
\end{equation}

For subsonic mach numbers:
\begin{equation}
\label{eq:ekindisp2}
    \ekindisp \approx 2 \frac{\shearvisc}{\dens_0} \; \enst
\end{equation}

\paragraph{Reynolds Number}

\begin{equation}
\label{eq:reynold}
    \reynold = \frac{\dens_0\;\vol_0\;\clen}{\shearvisc}
\end{equation}

\paragraph{Adiabatic Constant}
\begin{equation}
\label{eq:adiabconst}
    \adiabconst = \frac{\heatcapP}{\heatcapV}
\end{equation}

\paragraph{Prandtl Number}
\begin{equation}
\label{eq:prandtl}
    \prandtl = \frac{\shearvisc \; \heatcapP}{\heatconduct}
\end{equation}

\paragraph{Kinetic Dissipation}
