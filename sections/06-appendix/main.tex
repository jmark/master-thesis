\section{Appendix}

% --------------------------------------------------------------------------- %
% Influence of Ratio FV/DG

\subsection{Runtime Performance}

\image{1.0}{plots/stirturb/evolutions/timestep.png}{Time evolution of the timestep $dt$.
The CFL Numbers are listed in \tbl{tab:decayturb-cfl}.}

\image{1.0}{plots/stirturb/evolutions/runtime-cumsum.png}{Cumulative runtime of
the different runs. For example, the PPM solver needs two hours to get half-way
through the simulation at $t_d = 3.0$. \remark The contribution of input-output
operations and unintended interuptions are already substracted out.}

\image{1.0}{plots/stirturb/evolutions/runtime-relative.png}{Time derivative of
cumulative runtime (see \fig{plots/stirturb/evolutions/runtime-cumsum.png}).
For example, the run with the PPM solver would have lasted roughly 6 hours with
the rate at around $t_d = 3.0$: \\ 1 h $\cdot$ 6 = 6 h.}

\image{1.0}{plots/stirturb/runtime-vs-ekin-dissipation}{Relative Runtime vs Kinetic
Energy Dissipation Rate.}

\image{1.0}{plots/decayturb/evol-timestep}{Time evolution of the timestep $dt$.
The CFL Numbers are listed in \tbl{tab:decayturb-cfl}.}

\image{1.0}{plots/decayturb/evol-cum-runtime}{Cumulative runtime of the
different runs. For example, the Midpoint Finite-Volume solver needs 30 min to
get half-way through the simulation at $t_d = 3.0$. The graph for \fppm was
scaled down by a factor of 10. \remark The contribution of input-output operations and
unintended interuptions are already substracted out.}

\image{1.0}{plots/decayturb/evol-relative-runtime}{Time derivative of
cumulative runtime (see \fig{plots/decayturb/evol-cum-runtime}). For example,
the run with the RK3-Hybrid solver would have lasted roughly 3 hours and 36
minutes with the rate at beginning: \\ 0.6 h $\cdot$ 6 = 3.6 h. The graph for PPM
was scaled down by a factor of 10.}

\image{1.0}{plots/decayturb/runtime-vs-ekin-diss}{Relative Runtime vs Kinetic
Energy Dissipation Rate. The graph for PPM was scaled down by a factor of 10.}
