\section*{Acknowledgment}

First and foremeost I want to express my debt of gratitude and deep respect to
my supervisor, Prof. Dr. Stefanie Walch-Gassner, for inviting me to her
workgroup and for her excellent guidance.  Her invaluable advice, far-reaching
insights and open mind were much appreciated.  She granted me a lot of freedom
to pursue my own interests and always had a friendly ear for any thought or
idea. Even when they were stupid, retrospectively.

Likewise I want to thank Prof. Dr.-Ing. Gregor Gassner who encouraged me to
write this thesis in the first place.

Many thanks to PD Dr. Volker Ossenkopf for co-reviewing my work despite his
busy schedule.

\vspace{0.5cm}

I thank my family for their love, support and unlimited patience.

\vspace{3cm}


\begin{tabbing}
\hspace{1cm}    \= \textbf{Examinee} \hspace{4cm}  \= Johannes Markert \\
                \>                                 \> I. Physikalisches Institut \\
                \>                                 \> Universität zu Köln \\
                \>                                 \> markert@ph1.uni-koeln.de  \\\\

                \> \textbf{First Supervisor}       \> Professor Dr. Stefanie Walch-Gassner \\
                \>                                 \> I. Physikalisches Institut \\
                \>                                 \> Universität zu Köln \\
                \>                                 \> walch@ph1.uni-koeln.de  \\\\

                \> \textbf{Second Supervisor}      \> PD Dr. Volker Ossenkopf \\
                \>                                 \> I. Physikalisches Institut \\
                \>                                 \> Universität zu Köln \\
                \>                                 \> ossk@ph1.uni-koeln.de
\end{tabbing}

\newpage

\section*{Declaration of Authorship}

I hereby certify that this thesis has been composed by me and is based on my
own work, unless stated otherwise. No other person’s work has been used without
due acknowledgement in this thesis. All references and verbatim extracts have
been quoted, and all sources of information, including graphs and data sets,
have been specifically acknowledged. This work was neither published nor sent,
in part or whole, to any other examination committee.

\vspace{3cm}

\hspace{0.7cm}\begin{tabular}{ll}
\makebox[2.5in]{\hrulefill}\hspace{2cm}    & \makebox[2.5in]{\hrulefill}\\
\hspace{0.1cm} \emph{Location, Date}       & \hspace{0.11cm}\emph{Signature}
\end{tabular}

\newpage

\section*{Abstract/Zusammenfassung}
\paragraph{Abstract}
The goal of this thesis was to investigate if it is feasible to apply a
recent third-order discontinuous Galerkin method (FLEXI) to astrophysical
turbulence simulations and how they compare to established first/second order
finite volume methods (FLASH). The objective is that DG methods are less
dissipative and better resolve small scale structures.

The primary focus was to accurately model isothermal turbulences up to Mach 10
which are, for example, important for simulating star formation rates in
molecular clouds.

As a first step FLEXI was augmented in order to support isothermal turbulence
setups in the same way as FLASH does. Modules for polytropic cooling, turbulent
forcing, bulk motion correction and shock capturing were implemented and
integrated into FLEXI (cf. \sec{computational-frameworks}). These
modifications were then tested and verified with the Sod Shock Tube problem
(cf. \sec{sod-shock-tube}).

Since pure Galerkin schemes cannot cope with discontinuities, shock
capturing routines try to dampen spurious oscillations and as a last resort,
locally switch to second-order finite volumes. In theory, this hybrid scheme
should yield a convergence rate somewhere between 2 to 3, depending on the
ratio of DG elements to FV elements (cf. \sec{shock-capturing}).

The next step was to conduct driven turbulence simulations of Mach 2.5 with
three explicit time integration methods of ascending order (first order Euler,
second order Midpoint and third-order Runge-Kutta) each in combination with
hybrid and FV-only mode. The same simulation was also performed with three
solvers from FLASH: PPM, Bouchut3 and Bouchut5. They serve as reference
(cf. \sec{stirturb}).

\paragraph{Zusammenfassung}

The goal of this thesis was to investigate if it is feasible to apply a
recent third-order discontinuous Galerkin method (FLEXI) to astrophysical
turbulence simulations and how they compare to established first/second order
finite volume methods (FLASH). The objective is that DG methods are less
dissipative and better resolve small scale structures.

The primary focus was to accurately model isothermal turbulences up to Mach 10
which are, for example, important for simulating star formation rates in
molecular clouds.

As a first step FLEXI was augmented in order to support isothermal turbulence
setups in the same way as FLASH does. Modules for polytropic cooling, turbulent
forcing, bulk motion correction and shock capturing were implemented and
integrated into FLEXI (cf. \sec{computational-frameworks}). These
modifications were then tested and verified with the Sod Shock Tube problem
(cf. \sec{sod-shock-tube}).

Since pure Galerkin schemes cannot cope with discontinuities, shock
capturing routines try to dampen spurious oscillations and as a last resort,
locally switch to second-order finite volumes. In theory, this hybrid scheme
should yield a convergence rate somewhere between 2 to 3, depending on the
ratio of DG elements to FV elements (cf. \sec{shock-capturing}).

The next step was to conduct driven turbulence simulations of Mach 2.5 with
three explicit time integration methods of ascending order (first order Euler,
second order Midpoint and third-order Runge-Kutta) each in combination with
hybrid and FV-only mode. The same simulation was also performed with three
solvers from FLASH: PPM, Bouchut3 and Bouchut5. They serve as reference
(cf. \sec{stirturb}).

\newpage

\section*{List of Figures}

\section*{List of Symbols and Acronyms}

\newpage


