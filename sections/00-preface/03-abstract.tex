\section*{Abstract/Zusammenfassung}
\paragraph{Abstract}
Finite Volume methods (FV) are an established numerical scheme for
astrophysical turbulence simulations of molecular clouds and interstellar media
(ISM). ISM simulations are characterized by supersonic turbulent flows who 
impose very high demands on the numerical scheme with regards to robustness and
accuracy. The widely adopted fluid simulation framework FLASH, developed at the
\vip{Flash Center for Computational Science} (University of Chicago) provides
specialized FV solver, meeting the requirements imposed by supersonic flows.

Astrophysical turbulences show a rich variety of structure over huge time and
length scales. They play a major role in several processes such as the
formation of dense structures and stars as well as the stability of molecular
clouds. The desire to study and understand the underlying physics has driven
generations of scientists to develop more advanced numerical schemes with
higher resolution capabilities. However, FV methods provide limited space for
improvement due to their lack in degrees-of-freedom.

By contrast to FV schems, modern higher-order discontinuous Galerkin methods
promise to provide faster convergence rates, smaller diffusion and dispersion
errors, better data volume-over-surface ratio for efficient parallel processing
and better input/output handling due to the smaller volume of data. Their
drawback is the impracticality in resolving shocks. In this work, we propose a
hybrid scheme which combines the accuracy of DG with the stability of FV
schemes. In order to provide a proof-of-concept, the DG implementation FLEXI,
developed at the \vip{Institute of Aerodynamics and Gas Dynamics} (University
of Stuttgard), is augmented to support supersonic isothermal turbulence
simulations, governed by the compressible Euler equations.

A comparative study of driven and decaying turbulence setups ensures that the
hybrid scheme models turbulent flows correctly. Widely adopted methods of
turbulence statistics, such as density/velociy PDFs and powerspectra, give an
insight into the distribution of mass, velocity and energy from large to small
length scales. Well-established astrophysics solvers integrated into FLASH,
namely PPM, Bouchut3 and Bouchut5, serve as a reference.

\newpage
\paragraph{Zusammenfassung}
Finite Volume Verfahren (FV) sind ein etablierter numerischer Lösungsansatz für
astrophysikalische Turbulenzsimulationen von Molekülwolken und interstellaren
Medien (ISM). ISM simulationen sind charakterisiert durch turbulente Strömungen
im Überschallbereich. Das stellt höchste Anforderungen an das numerische
Verfahren hinsichtlich Robustheit und Genauigkeit. Die weithin genutzte
Strömungssimulationssoftware FLASH, entwickelt am \vip{Flash Center for
Computational Science} (Universität Chicago) bietet spezialisierte FV-Löser,
die den Anforderungen supersonischer Strömungen gerecht werden

Astrophysikalischen Strömungen zeigen eine reiche Vielfalt an Strukturen über
große Zeiträume und räumliche Ausdehnungen hinweg. Sie spielen eine wichtige
Rolle in vielen Prozessen, wie zum Beispiel die Formung von hochdichten
Strukturen und Sternen oder in der Frage über die Langlebigkeit von
Molekülwolken. Den Drang, die dahinterliegende Physik zu studieren und zu
verstehen, hat Generationen von Wissenschaftlern angetrieben, immer bessere und
genauere numerischen Verfahren zu entwickeln. Allerdings bieten FV-Verfahren
auf Grund ihrer eingeschränkten Flexibilität wenig Spielraum für Verbesserungen.

Im Gegensatz zu FV-Verfahren, versprechen modere schrittweise glatte
Galerkin-Methoden schnellere Konvergenz, weniger Diffusion und Dispersion,
günstigere Volumen-zu-Oberflächen-Verhältnisse, gut geeignet zur parallelen
Verarbeitung, und bessere Lese-/Schreibraten durch geringeres Datenaufkommen.
Allerdings sind sie nicht zur Darstellung von Schockwellen geeignet. Im Rahmen
dieser Arbeit, stellen wir ein Hybrid-Löser vor, welches die Präzsision von
DG- mit der Stabilität von FV-Verfahren vereint. Damit wir die Machbarkeit
dieses Konzepts zeigen können, erweitern wir die DG-Implementierung FLEXI,
entwickelt am \vip{Institut für Luft- und Gasdynamik} (Universität Stuttgart),
um die Möglichkeit supersonische isothermische Turbulenzen zu simulieren.
Die Dynamiken werden durch die kompressiblen Euler-Gleichungen bestimmt.

Vergleichende Studien von getriebener und zerfallender Turbulenz stellen die
korrekte Darstellung von Turbulenzen durch den Hybrid-Löser sicher.  Gängige
Methoden der Turbulenzstatistik, wie Dichte- und Geschwindigkeitsverteilungen
sowie Energiespektren, geben Auskunft über die Verteilung von Masse,
Geschwindigkeit und Energie auf großen und kleinen Längenskalen. Etablierte, in
FLASH integrierte, numerische Löser für die theoretische Astrophysik dienen als
Referenz. Zu nennen sind PPM, Bouchut3 und Bouchut5.
