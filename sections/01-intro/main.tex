\section{Introduction}

Astrophysical turbulence simulations of interstellar gases a play vital role in
the understanding and modelling of star formation (cf.
\cite{mac2004turbulence}). A multitude of numerical schemes and computational
fluid dynamics software has been developed, each with their own merits and
drawbacks.

Finding the best numerical solver for a specific problem domain is a vibrant
field of research as of today, even though the first pioneers began to conduct
simple physics simulations many decades ago. A lot has changed since these
days. The present explosion in computing power paralleled by a striking
inflation of operational costs passes on a mighty tool to the astrophysics
community, which opens the door to numerical experiments unfeasible ten years
ago.

Alongside the advent of new technical possibilities, modern numerical schemes
appear which are better suited for utilizing latest CPU architectures to gain
better performance. Especially, the aeronautics and car industries push forward
the development of new simulation software. They are interested in modelling
flight characteristics for new airplane designs, the aerodynamic drag of cars
or want to increase the efficiency of combustion engines. Consequently,
innovative numerical schemes should be highly flexible with regards to meshing
complex geometries, very accurate, massively parallelizable and blazingly fast.
The demands for the underlying physical model are rather moderate. The
Navier-Stokes equations suffice on most occasions. One of the latest specimen
of their kind is the high-order accurate CFD software FLEXI\footnotemark[1]
developed by teams at the \vip{Mathematical Institute} (Univ. of Cologne) and
the \vip{Institute of Aerodynamics and Gas Dynamics} (Univ. of Stuttgart).
Special credit goes to Prof. Dr.-Ing. Gregor Gassner, who is the initiator of
the project, and his former supervisor Prof.  Dr. Claus-Dieter Munz. Their idea
is to introduce piecewise polynomial functions of higher order. The
underpinning scheme is called \emph{discontinuous Galerkin method}. The
implementation details of FLEXI are discussed in
\cite{hindenlang2014mesh,hindenlang2012explicit,gassner2009discontinuous}.

In contrast to air flow models, the demands for turbulence simulations of
astrophysical gases are quite the opposite. Free space does not imply complex
geometries. The physical model becomes very complex, when electro-magnetic
fields, radiation, chemistry and gravity are introduced to the governing
equations.  Furthermore, a huge range of scales in time, space, energy, density
and pressure must be covered as well. As if this is not enough, high Mach
numbers in the flow give rise to strong shock conditions or discontinuities,
which challenge the stability and accuracy of every numerical solver. Up to
now, the astrophysics community considers finite-volume schemes of first/second
order to be the only viable method for conducting this kind of simulations.
FLASH\footnotemark[2], originally developed by researchers at the University of
Chicago, is the name for a well established nuclear and astrophysics simulation
software with \emph{finite volumes} as their foundational scheme. FLASH is, for
example, discussed in \cite{fryxell2001adaptive,lee2011progress,walch2015silcc}.

This thesis is a first attempt to introduce higher order Galerkin methods to
astrophysical turbulence simulations. The statistical evaluation of simple
turbulence setups, run by FLASH and FLEXI, are expected to give an insight into
the potential, Galerkin schemes might offer to the astrophysics community.

\footnotetext[1]{https://www.flexi-project.org}
\footnotetext[2]{http://flash.uchicago.edu/site/flashcode/}

\newpage
