\section{Introduction}

Turbulent gases are everywhere and play a major role in nature, science and
engineering. Consequently, the desire to model resp. simulate fluent media as
close to reality as possible has driven many generations of scientists to
develop better algorithms.

%% \image{0.7}{TGV-quiver.png}{Initial Velocity Field: Slice through periodic box.}

There are three distinct streams of numerical solution techniques: finite difference,
finite element and spectral methods. Finite Volume Methods are a specialization of
finite difference methods wheras DG methods belong the the family of finite element
methods.

A numerical solving algorithm consists of three parts:

- Integration of the governing equations of fluid flow over the control volume

- Discretization of the integral into a system of algebraic equations

- iterative time stepping method


Advantage of polynomial methods: exact interpolation -> arbitrary high information
density ... space savings

Since discontinuous Galerkin (DG) methods assume 
discontinuous approximate solutions, they can be considered as generalizations
of finite volume methods.

Owing to their finite element nature, the DG methods have the following
main advantages over classical finite volume and finite difference methods:
— The actual order of accuracy of DG methods solely depends on the 
exact solution; DG methods of arbitrarily high formal order of accuracy
can be obtained by suitably choosing the degree of the approximating
polynomials.
— DG methods are highly parallelizable. Since the elements are 
discontinuous, the mass matrix is block diagonal and since the size of the blocks
is equal to the number of degrees of freedom inside the corresponding
elements, the blocks can be inverted by hand (or by using a symbolic
manipulator) once and for all.
— DG methods are very well suited to handling complicated geometries
and require an extremely simple treatment of the boundary conditions in
order to achieve uniformly high-order accuracy.
— DG methods can easily handle adaptivity strategies since refinement or
unrefinement of the grid can be achieved without taking into account
the continuity restrictions typical of conforming finite element 
methods. Moreover, the degree of the approximating polynomial can be easily
changed from one element to the other. Adaptivity is of particular 
importance in hyperbolic problems given the complexity of the structure of
the discontinuities.

More information can be found in 
Bernardo Cockburn
George E. Karniadakis
Chi-WangShu(Eds.)
Discontinuous
Galerkin Methods
Theory, Computation
and Applicationsi


They provide fast convergence, small
diffusion and dispersion errors, easier implementation of the inf-sup condition for
incompressible Navier-Stokes, better data volume-over-surface ratio for efficient
parallel processing, and better input/output handling due to the smaller volume
of data.

\image{0.6}{DG-work-over-periods.png}{Computational work (number of
floating-point operations) required to integrate a linear advection equation
for $M$ periods while maintaining a cumulative phase error of e = 10\%. Source: \cite{},p.10
}

\paragraph{hp Convergence}
The mathematical theory of finite elements in the 1970s has established rigor-
ously the convergence of the h-version of the finite element. The error in the
numerical solution decays algebraically by refining the mesh, that is, introduc-
ing more elements while keeping the (low) order of the interpolating polynomial
fixed. An alternative approach is to keep the number of subdomains fixed and
increase the order of the interpolating polynomials in order to reduce the error in
the numerical solution. This is called p-type refinement and is typical of polyno-
mial spectral methods [101]. For infinitely smooth solutions p-refinement usually
leads to an exponential decay of the numerical error.

Source Spectral/hp Element Methods for CFD 1999



