\section{Introduction}

Astrophysical turbulence simulations of interstellar gases play vital role in
the understanding and modelling of star formation. A multitude of numerical
schemes and computational fluid dynamics software has been developed each
with their own merits and drawbacks.

Finding the best numerical solver for a specific problem domain is a vibrant
field of research as of today even though the first pioneers began to conduct
simple physics simulations many decades ago. A lot has changed since these
days. The present explosion in computing power paralleled by a striking
inflation of operational costs passes on a mighty tool to the astrophysics
community which opens the door to numerical experiments unfeasible ten years
ago.

Alongside the advent of new technical possibilities modern numerical schemes
appear which are better suited for utilizing lates CPU architectures to gain
better performance. Especially, the aeronautics and car industries push forward
the development of new simulation software. They are interested in modelling
flight characteristics for new airplane designs, the airodynamic drag of cars
or want to increase the efficiency of combustion engines. Consequently,
innovative numerical schemes should be highly flexible with regards to meshing
complex geometries, very accurate, massively parallelizable and blazingly fast.
The demands for underlying physical model are moderate. The Navier-Stokes
equations suffice on most occasions. One of the latest specimen of their kind
is the high-order accurate CFD software FLEXI developed by a team lead by
Prof. Claus-Dieter Munz in the Aeronautics Institut in Stuttgard, Germany.
The basic idea is to introduce piecewise polynomial functions of order higher
order. The scheme is called Discontinuous Galerkin method.

In contrast to aeronautical models the demands for turbulence simulations of
astrophysical gases are quite the opposite. In free space there are no complex
geometries at least not in proximity of massive objects like black holes. The
physical model become extremely complex when electro-magnetic fields,
radiation, chemistry and gravity are introduced to the governing equations.
Furthermore a huge range of scales in time, space, energy, density and pressure
must be covered, too. As if this is not enough high Mach numbers in the gas
give rise to strong shock conditions and discontinuties which challenge the
stability and accuracy of every numerical solver. Up to now the astrophysics
community considered finite-volume schemes of first/second order to be the only
viable method for conducting their simulations. FLASH originally developed by
researchers at the University of Chicago is the name for a well established
nuclear and astrophysics simulation software with finite-volumes as their
foundational scheme.

This thesis is a first attempt to introduce higher order Galerkin methods to
astrophysical turbulence simulations. The statistical evaluation of simple
turbulence setups run by FLASH and FLEXI are expected to give an insight into
the potential Galerkin schemes might offer to the astrophysics community.

\newpage
